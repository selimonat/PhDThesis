\label{part_sg_mm}


\subsection{Participants} 


EEG was recorded from 30 healthy university students, who were first
screened for normal sight and hearing using standard tests (Landolt C chart
and calibrated PC-based audiometer) and gave their informed written consent
to participate in the study. Of these, 6 subjects were discarded prior to
data analysis due to noisy data. The remaining 24 subjects (14 female, 23
right-handed) were aged between 19 and 33 years (mean: 23, standard
deviation: 3). The experiment was conducted in accordance with the
Declaration of Helsinki.

\subsection{Experimental Design}

In order to investigate multisensory integration processing, it is possible
to contrast responses to unimodal and bimodal stimuli, or alternatively, to
compare congruent and incongruent bimodal stimulation. To address both of
these contrasts, we presented auditory and visual stimuli in four
stimulation conditions: bimodal congruent, bimodal incongruent, unimodal
visual, and unimodal auditory.  

\begin{SCfigure}[][!htb]
\includegraphics[width=0.5\textwidth]{part_sg/figure/Figure1_ExpParadigm.png}
\caption[Schematic Diagram of Both Experimental
Paradigms.]{\protect\textbf{Schematic diagram of both experimental paradigms.} For both paradigms trial
onset was self-paced, and subjects indicated their readiness by button
press. Each trial began after a random delay with the appearance of a red
fixation cross, which subjects were asked to fixate until the end of the
trial. \textbf{(a)} Bimodal trials consisted of simultaneously presented
visual and auditory stimuli, and were either congruent (stimuli followed
the same trajectory) or incongruent (different trajectories). \textbf{(b)}
Single modality trials contained two sequentially presented stimuli,
separated by a random interval (500 -1500 ms). A pair could entail stimuli
from one or both modalities, again either congruent or incongruent. After
stimulus presentation, subjects were required to decide whether the trial
was congruent or incongruent, answering via button press.
}
\label{expparadigm}\end{SCfigure} 



To engage their attention, participants were required to perform a
congruency judgement task. In the bimodal case, subjects were asked to
judge the temporal congruency of simultaneously presented auditory and
visual stimulus components, thus ensuring that attention would be equally
allocated to both modalities. To similarly engage subjects' attention
during unimodal presentations, we adjusted the task to a sequential
comparison of two consecutively presented unimodal stimuli, meaning that
each such trial contained two stimuli. A sequential pair could consist of
any combination of the two modalities (V-A, A-V, V-V or A-A). In subsequent
EEG analysis, however, each stimulus of these sequentially presented pairs
was treated as an independent unimodal stimulus in either the unimodal
visual or auditory condition. 



We chose a self-paced paradigm in which subjects triggered trial onset via
button press (see Fig. \ref{expparadigm}). In doing so, we aimed to
maximise participants' concentration and minimise eye-movements during
stimulus presentation. Each trial began with the presentation of a red
fixation cross. After a short, random delay, the stimulus appeared on the
screen and/or was heard through headphones. In the case of bimodal stimuli,
auditory and visual components were simultaneously presented, after which
subjects completed the task by pressing one of two buttons (denoting
"congruent" and "incongruent"). In the case of sequential trials, two
stimuli were shown consecutively, separated by a short, random time
interval (500-1500 ms), and subjects completed the task after the second
stimulus. For purposes of behavioural analysis, task performance was
evaluated on a trial basis.



\subsection{Stimuli}

Stimuli consisted of a rotating Gabor patch, a frequency-modulated tone, or
both presented simultaneously, and lasted for 6 seconds. Gabor patches were
grayscale, had a sinusoidal frequency of 0.33 cycles/$^\circ$, subtended
approximately 4.5$^\circ$ of visual angle (FWHM) and were centrally
presented on a gray background identical to the mean luminance of the
patch. Auditory stimuli had constant amplitude, and were ramped with a 10
ms half-cosine window to avoid clicks or other artifacts. Movies of visual
stimuli were sampled at a frame rate of 25 Hz, and uncompressed visual and
auditory stimuli were merged into an avi file for bimodal presentation
using MEncoder (MPlayer version 1.Orc1, www.mplayerhq.hu).



Both unimodal stimulus components varied along only one feature dimension:
orientation of Gabor patches spanned $180^\circ$ centred around the
horizontal axis, and the frequency of tones was modulated between 200 and
300 Hz. This ongoing feature modulation was determined by trajectories that
changed smoothly but irregularly over time (see Fig. \ref{xcorrpower}).
Visual and auditory feature dimensions were chosen based on a behavioural
pilot study (see Section \ref{app_eeg} for more detail on how we have
selected which feature to modulate during the main experiment) 32
trajectories were created in Fourier space by assembling a Gaussian power
spectrum with a cut-off frequency of 1 Hz (resulting in a cut-off frequency
of $\approx$4 Hz for stimulus speed) and random phase information. Each
trajectory was constructed as part of a set of 4 mutually orthogonal
trajectories, and in total 8 sets were created using an iterative
optimisation procedure. The Fourier representation was then transformed to
the time domain using the inverse fast Fourier transform. Stimuli were
presented in congruent and incongruent conditions, where visual and
auditory stimuli were modulated by identical or orthogonal trajectories
(see below). This allowed us to precisely control for the degree of
incongruence. 



Stimuli were organised into four blocks, each consisting of 24 congruent
audiovisual, 24 incongruent audiovisual, 24 visual, and 24 auditory stimuli
(yielding 48 simultaneous and 24 sequential trials). Each block was created
from two sets of orthogonal trajectories, with each trajectory occurring
three times in each of the four conditions. Trajectories were balanced over
conditions and presentation order of paired stimuli. Block order and
stimulus order within blocks was randomised over subjects. Depending on
time constraints and subject alertness, participants completed either 3 or
4 blocks (equivalent to 288 or 384 stimuli).



\subsection{Presentation \& Recording}

Subjects were comfortably seated approximately 1 m from the presentation
screen in a dimly-lit recording room. Stimuli were presented on a 21"
monitor (Samsung SyncMaster 1100 DF) with a refresh rate of 100 Hz. The
presentation software (Neurobehavioural Systems, version 10.3) was run on a
Macintosh Pro (Apple Computers). Sounds were delivered via in-ear
headphones (EAR-Tone 3A, Etymotic Research Inc.) at a comfortable volume
chosen by the subject. During trials, subjects were asked to remain still
and keep their gaze fixated on a centred, red fixation cross. 



EEG was recorded from 28 sintered Ag/AgCl ring electrodes mounted in a cap
(EasyCap GmbH, Herrsching-Breitbrunn, Germany) according to the standard
10/20 system, referenced to linked mastoids. Four additional electrodes
were used to capture vertical and horizontal electro-oculograms (bipolar
montages). The upper threshold for impedances was 5 kOhm. Two of the 30
standard electrodes (FT8 and FT7) were exchanged for diodes to capture
visual and auditory stimulus onset. Signal amplification, filtering and
digitisation were carried out by a 32-channel amplifier system (Synamps1,
Neuroscan, Compudedics, TX, USA). The data were digitized at 500 Hz and
online bandpass-filtered within 0.3 and 100 Hz to prevent aliasing. The
digital signal was recorded by a PC-workstation (Intel Pentium 4, 2.41 GHz,
1GB RAM).



The diode signals revealed that the visual stimulus preceded the auditory
by 25 ms \textemdash a small delay compared to the cut-off frequency of the
trajectories \textemdash due to the stimulation software used.


\subsection{Behavioural Analysis}

Behavioural analysis served two purposes: first, to determine whether
subjects were attentive during the experiment; second, to ensure that
subjects could perceptually distinguish bimodal congruent from bimodal
incongruent stimuli. Task performance was evaluated in terms of trials, and
simultaneous and sequential stimulus presentations were dealt with
separately.



To determine whether subjects performed above chance level, an exact
Binomial test was performed for each subject. We furthermore used signal
detection methods to quantify subjects' sensitivity and bias in
discriminating between congruent and incongruent bimodal stimuli (see
\cite{Wickens2002a} for more detail). The sensitivity measure \textit{d'}
was computed from hit (correct identification of congruent stimulus) and
false alarm rates.  Assuming that both rates come from standard normal
distributions, \textit{d'} estimates the distance between the two
distributions in units of standard deviations \textemdash the further apart
the two distributions, the easier the congruency can be detected, and the
higher the \textit{d'}. The response bias estimate ($\log{\beta}$) informs
us whether the observer favours a 'congruent' or 'incongruent' response,
with a negative value indicating a bias to congruence, and a positive value
indicating an incongruence bias.

\subsection{Pre-processing \& Artefact Removal}

All data analysis was carried out using MatLab (The MathWorks, Natick, MA),
using the EEGLab toolbox \citep{delorme2004a} and custom-made functions.
Before artefact removal, all data were bandpass filtered between 0.5 and
100 Hz and epoched. Eye-movement artefacts were eliminated using
Independent Component Analysis \citep{Jung2000a}. Other muscle artefacts
were detected by visual inspection, and whole trials were rejected. All
trials were baseline corrected relative to 500 ms of pre-stimulus activity.

\subsection{Stimulus Locking}

Two approaches were used to quantify stimulus locking, both involving
cross-correlations between EEG activity and the magnitude of change of the
stimulus (see Fig. \ref{xcorrpower}). All correlation coefficients were
normalised so that the values lie between -1 and 1, with 0 indicating the
absence of a correlation. Correlation coefficients are calculated for all
temporal offsets between stimulus and response. Significant coefficients at
positive time-lags indicate that the stimulus leads the cortical responses,
that is the brain response is locked to the stimulus dynamics. Due to the
finite autocorrelation of the stimulus, a locking might even occur at
negative lags. 



Our first analysis approach (spectrotemporal analysis) was concerned with
the question of whether the time course of stimulus-induced power changes
in the EEG correlates with changes in stimulus speed . This approach
allowed us to investigate whether the power in any frequency band of the
measured EEG signals locks to the dynamic stimulus. 

A second analysis approach (waveform analysis) was used to examine whether
the EEG waveform itself locks to the temporal profile of the stimulus. This
approach was used to investigate whether there is a stable
phase-relationship between evoked EEG response and stimulus. A significant
correlation coefficient indicates that the phase of the EEG signal locks to
the ongoing stimulus. In contrast to the spectrotemporal approach, this
analysis only reveals locking of frequencies common to both stimulus and
EEG. A detailed description of each approach is given below.

\begin{figure}[!htb] \centerline{
\includegraphics[width=\textwidth]{part_sg/figure/Figure2_AnalysisMethod}}
\caption[Two Correlation Analysis Approaches.]{\protect\textbf{Two Correlation Analysis Approaches.} Spectrotemporal and waveform
correlation methods are illustrated on the left and right, respectively.
Exemplary stimulus trajectory and single-trial EEG waveforms are shown in
the centre, with a gray bar indicating the temporal range used in both
analysis approaches. In the case of spectrotemporal analysis, each trial is
transformed into a time-frequency representation (see text for details) and
then correlated with the speed profile of the corresponding stimulus. The
waveform analysis method involves creating an ERP from all trials
corresponding to a single stimulus, and then correlating this waveform
directly with the stimulus speed. Peaks at positive time lags indicate that
the EEG power or waveform follows the temporal dynamics of the stimulus, in
other words that the EEG has locked to the stimulus. In both approaches,
correlograms are then averaged over stimuli. Grand averages are created by
taking the median over subjects.
} \label{xcorrpower}\end{figure} 



In the spectrotemporal analysis approach, spectral power changes in
response to visual speed were determined on a trial-by-trial basis (Fig.
\ref{xcorrpower}, left panels). For each trial, we estimated the Short-time
Fourier Transform using a window length of 80 ms shifted with an overlap of
78 ms. Each data segment was windowed (Hamming window) and zero-padded (to
256 samples $\approx$512 ms). Thus, we achieved a nominal temporal
resolution of 2 ms and frequency resolution of approximately 2 Hz. The
amplitude of the Short-time Fourier Transform was squared to obtain
spectral power, and then z-transformed with respect to baseline power. Due
to the power increase evoked by stimulus onset, we removed the first 500 ms
of each trial which is uninformative for the present purpose and
investigated only the stationary response. The power of individual
frequencies was then either summed across our frequency range of interest
(20-35 Hz) corresponding to previous results \citep{kayser2004b}, or kept
at its full resolution. Each frequency band was then separately
cross-correlated with the speed of the stimulus, given by the absolute
values of the trajectory's first derivative. The resulting
cross-correlations were then averaged using Fisher's z-transform over all
trials within each condition, yielding one cross-correlogram per subject,
condition, frequency band and channel. Bimodal incongruent trials resulted
in two cross-correlograms due to the difference in visual and auditory
stimulation trajectories. 



The waveform analysis approach evaluated whether stimulus speed is
reflected in the EEG waveform by cross-correlating stimulus speed with
stimulus-specific ERPs. An ERP was computed by averaging the EEG over those
trials in which the same stimulus (i.e. identical trajectories) was shown
for the modality in question (Fig. \ref{xcorrpower}, right panels). Due to the
limited number of trajectory repetitions throughout the experiment, ERPs
were constructed from a maximum of 9 waveforms (please note that the number
of trials is substantially larger). Stimulus-specific ERPs were then
cross-correlated with the speed profile of the corresponding trajectory.
The stimulus average for each subject, condition, and channel was obtained,
again using Fisher's z-transform. As above, the grand average was
calculated using the median.



Our first research question asked whether there is any evidence for
stimulus-locking to visual and auditory stimulus dynamics, in other words
how the cortex responds to the presence of a time-varying stimulus. To
address this, for each modality we averaged over all conditions containing
input to that modality, e.g. unimodal visual and both bimodal conditions
were averaged to investigate visual stimulus-locking. This increased the
amount of data used in our analysis, thus stabilising effects that were
otherwise small and less consistent between subjects. In addition, we can
thus estimate the modality-specific, automatic, bottom-up response
irrespective of task factors or information present in other modalities.
After evaluating the presence or absence of stimulus locking, the original
conditions were used to explore the second question, namely whether locking
is modulated by multisensory interaction. Finally, to address the third
research question, we estimated changes in stimulus locking over the
duration of a trial. This was done using a time-dependent cross-correlation
function. Cross-correlations were computed within a 500 ms shifting time
window (250 ms overlap), thus yielding 23 cross-correlograms per trial.
Averaging over trials, conditions and subjects was performed as above.

\subsection{Evaluating Statistical Significance}

Significance of the resulting correlation coefficients was tested using
bootstrap techniques. To test for significant peaks within a subject and
for a given condition, we computed 1000 cross-correlations composed of
averaged cross-correlations between non-matching EEG data and stimuli. From
this distribution, the 95\% and 99\% confidence intervals were estimated.
To test significance of the grand average, a distribution of 1000 medians
was generated to determine confidence intervals.



To test for significant condition differences, permutation testing was
used. For each subject, 1000 surrogate conditions were created by pooling
and randomly redrawing trials from both conditions without replacement. A
distribution of condition differences was obtained by subtracting the
resulting 1000 grand averages (median) of both conditions.


