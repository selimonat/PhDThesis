



Here we first evaluate subjects' task performance in order to determine
their alertness during the experiment and ability to distinguish
audiovisual congruent from incongruent stimuli. Next we address the issue
of whether stimulus locking can be found in the human brain. Finally, we
compare congruent and incongruent bimodal conditions to investigate whether
stimulus locking is involved in the crossmodal processing of temporal
information. 

\subsection{Task Performance}

All subjects performed well above chance in the task ($p \textless 0.01$,
exact binomial test), with an average performance of 76 \% ($\pm$8\%
standard deviation). This result indicates that subjects were attentive
during the experiment and understood the task. Dividing data according to
simultaneous audiovisual and sequential single modality trials, subjects'
performance was better for the simultaneous (87 $\pm$11\%) than the
sequential paradigm (65 $\pm$7\%). All subjects performed above chance in
simultaneous trials (23 with $p \textless 0.01$, 1 with $p \textless
0.05$).  In sequential trials the majority of subjects remained
significantly above chance (14 with $p \textless 0.01$, 5 with $p \textless
0.05$), while 5 of the 24 were not able to significantly discriminate
congruent from incongruent trials. The differences in performance are most
likely related to the difficulty of the task \textemdash it is intuitively
more difficult to compare two temporally extended patterns when they are
serially presented than when they are simultaneously available.



In the case of bimodal trials, we further investigated subjects' ability to
detect congruence and whether any bias was involved in their congruency
judgement. This was necessary before proceeding with the contrast between
EEG responses to bimodal congruent and incongruent stimuli. 22 out of 24
subjects had \textit{d'} estimates greater than 1, with 19 of these
exceeding a \textit{d'} of 2. The subject average was 2.92, indicating that
the difference between congruent and incongruent bimodal stimuli was
clearly perceptually detectable. Interestingly, almost all subjects (21 out
of 24) had a negative $\log{\beta}$ estimate revealing a bias to respond
'congruent' with a mean $\log{\beta}$ of -0.8 ($ \pm$1 standard deviation).
Maximum and minimum bias estimates were 1.65 and -2.6132, respectively. 



As mentioned above, each stimulus of sequential pairs is treated as an
independent stimulus in either the unimodal visual or unimodal auditory
condition. Thus, we are not concerned with the behavioural results for
those conditions. In all further analysis, all stimuli are used to
calculate results, regardless of whether they appeared in correctly or
incorrectly answered trials. 



\subsection{Stimulus Locking: Can We Find it in Human EEG?}

We first had to establish whether EEG entrains to the dynamics of the
presented stimuli. As mentioned in the Methods section, we quantify the
locking for each individual input signal by correlating it with the
measured EEG from any trial in which the modality of interest was
stimulated with that input, thus averaging over conditions. Separate
conditions are compared in the next section.



\subsection{Visual Locking}

The spectrotemporal analysis approach quantifies the amount of
correspondence between the magnitude of stimulus change and the
spectrotemporal power of the EEG within a given frequency range. To
evaluate whether our data show stimulus locking, we first concentrate on
the 20-35 Hz frequency band reported by \cite{kayser2004b}. Here we examine
the effect at the population level, averaging first over all visual
conditions (bimodal congruent and incongruent and unimodal visual) and then
over subjects. As can be seen in Fig. \ref{visuallocking}A, we find
stimulus locking to visual stimuli within this frequency range at occipital
electrodes (O1, OZ, O2), peaking at a lag of 92 ms with a maximal
correlation coefficient of 0.01.

\begin{SCfigure}[][!htb]
\includegraphics[width=0.5\textwidth]{part_sg/figure/Figure3_XcorrPower.png}
\caption[Stimulus Locking of EEG Power to Visual
Stimuli.]{\protect\textbf{Stimulus Locking of EEG Power to Visual Stimuli.} \textbf{(A)} Each plot shows the
topographic distribution of the grand average of the spectrotemporal
analysis correlation results (averaged over bimodal congruent, incongruent
and unimodal visual conditions and over all subjects) within the 20-35 Hz
band, at selected time lags beginning at 0 ms and ending at 200 ms lag.
Dots represent locations of labelled electrodes, with locations below head
centre drawn outside the cartoon head. Colour codes for the magnitude of
the correlation coefficient, with warm colors indicating positive and cold
colors negative correlation, and values are linearly interpolated between
recording sites for visualisation purposes. \textbf{(B)} Grand average
spectrotemporal analysis correlograms (as in a) for individual frequencies
are shown for electrode OZ. Each row represents the results for a single
frequency, with frequencies given on the y-axis and correlation lags on the
x-axis. Colour codes for magnitude of correlation coefficient
}\label{visuallocking}
\end{SCfigure} 

 

Looking now at the entire range of frequencies that are available, again at
the population level, we see at occipital electrodes that in addition to
this positive correlation between visual stimulus dynamics and EEG power
($r = 0.015$ at OZ), there is also a negative correlation found in lower
frequencies at a later correlation lag ($r =-0.03$ at OZ, see Fig.
\ref{visuallocking}B).  Although these effect sizes are small, they are
clearly different to baseline, and significance testing using permutation
tests reveals that these peaks are indeed highly significant ($p \textless
0.01$). Thus, we report two ranges of stimulus-locking: a positive peak
centred at approximately 27 Hz occurring at 80 ms lag after visual
stimulation; and a strong negative correlation between 8 and 20 Hz
beginning at approximately 180 ms lag. The difference in direction of these
correlations suggests that there are at least two frequency-specific
stimulus-locking phenomena in response to visual stimulation.


These results are supported by an analysis of individual subjects. In 10
out of 24 subjects, a significant peak was found in the beta range for at
least one occipital recording site. The clearest beta correlation was not
found in the same occipital channel for all participants, with some
displaying a strongly lateralized effect. The anti-correlation found in
lower frequencies showed less variability across subjects, and EEG power of
20 out of 24 subjects was found to significantly anti-correlate with visual
stimuli at an occipital electrode. 



Visual stimulus speed is reflected not only in the power profile of single
EEG trials but also in the EEG waveform itself, as can be seen from the
waveform analysis results (Fig. \ref{waveform}A). The grand average cross-correlogram
between stimulus speed and the stimulus-specific ERP waveform, averaged
over all visual conditions, reveals peaks at various recording sites across
the scalp. The strongest effects are found at occipital sites at 78 ms lag
with a magnitude of correlation of 0.12. An extensive cluster of electrodes
in the centro-parietal region shows a more complex pattern of locking, with
a first positive peak at approximately -60 ms lag followed by a negative
peak at around 180 ms and a second positive peak at approximately 390 ms
lag. The strongest effect within this cluster is seen at CPZ ($r = 0.09$).
Given that EEG at these sites follows the stimulus with differing delays,
occipital and centro-parietal clusters are likely to reflect distinct
underlying cortical sources. 

\begin{figure}[!htb]\centerline{
\includegraphics[width=\textwidth]{part_sg/figure/Figure4_XcorrERP.png}}
\caption[EEG Waveforms Lock to Visual Input at Multiple Sites.]
{\protect\textbf{EEG Waveforms Lock to Visual Input at Multiple Sites.} \textbf{(A)} Headplots show the
topographic distribution of the grand average correlation coefficients
between ERP waveform (averaged over bimodal congruent and incongruent and
unimodal visual conditions) and visual stimulus speed at selected time lags
between 0 and 400 ms. Correlation magnitude is color coded as in Fig.
\ref{xcorrpower}. \textbf{(B)} Condition comparison for electrodes OZ
(upper panel) and CPZ (lower panel). Waveform analysis correlograms for
bimodal congruent (BC, green line), bimodal incongruent (BI, red) and
unimodal visual (UV, dashed blue) are shown with time lags on the x- and
correlation coefficients on the y-axis. Time lags showing significant
differences between bimodal incongruent and bimodal congruent conditions
are highlighted in light gray for p\textless0.05 and darker gray for
p\textless0.01
}\label{waveform} \end{figure} 



Contrasting the two analysis approaches, we see that the waveform analysis
provides a stronger, consistent measure of stimulus locking. For almost all
subjects (23 of 24), EEG recorded from OZ correlates with stimulus speed
($p \textless 0.01$ for 22 subjects, $p \textless 0.05$ for one subject).
Significant peaks occur between zero and 300 ms lag (median r: 0.1372 for
significant subjects, 0.1353 for all subjects) for the average of all
visual conditions. Stimulus locking in the centro-parietal region is
similarly stable, with 20 subjects showing a significant effect (19 with p
\textless 0.01, 1 with $p \textless 0.05$) at CPZ between 200 ms and 460 ms
lag, (median r: 0.1229 for significant subjects, 0.1100 for all subjects).
Thus, phase-locking, as captured by waveform analysis, is more reliable
than the power-locking quantified by our spectrotemporal approach, making
it more useful for investigating \linebreak 

condition differences.



\subsection{Auditory Locking}

As a next step, we asked whether stimulus locking can also be found in the
auditory system. Here, we didn't have an initial frequency band or site to
guide our investigation. Representative results for both analysis
approaches are depicted in Fig. \ref{audiolock} for electrode CZ (chosen because its
auditory ERP showed the largest evoked potential of all channels shortly
(~100 ms) after stimulus onset) for the unimodal auditory condition.
Although the spectrotemporal analysis results for this recording site
suggest there may be stimulus-locking between 45 and 55 Hz, this
correlation is not salient with respect to other lags and frequencies, and
is furthermore not found at neighbouring channels nor is it consistent
across auditory conditions. Waveform analysis cross-correlograms also show
no salient peaks, with very low correlation coefficients over all time
lags. As in the visual case, we also looked at the average over all
auditory conditions, since we assumed that averaging over more trials might
uncover small effects. However, the obtained results are similarly
uninformative about auditory stimulus-locking. In addition, no consistent
effects were seen on the individual subject level for either analysis
approach.


\begin{SCfigure}[][!htb]
\includegraphics[width=0.5\textwidth]{part_sg/figure/Figure5_AuditoryResults.png}
\caption[No Stimulus Locking to Auditory
Input.]{\protect\textbf{No Stimulus Locking to Auditory Input.} Exemplary unimodal auditory results
are depicted for electrode CZ. \textbf{(A)} Spectrotemporal analysis
cross-correlograms are shown in rows for individual frequency bands
(y-axis). Colours represent correlation coefficients. \textbf{(B)} The waveform
analysis cross-correlation is represented with time lags on the x- and
correlation coefficients on the y-axis
}\label{audiolock} \end{SCfigure} 



\subsection{Is there Evidence for Multisensory Interactions?}

To address the question of whether stimulus locking is subject to
multisensory interactions, we contrasted different stimulation conditions.
As we did not find any evidence for auditory stimulus locking, we
concentrate in the following on visual stimulus locking and how it is
modulated by the presence of matching or mismatching auditory input. 

Despite the absence of auditory stimulus locking, the waveform analysis
results revealed that incongruent auditory information modulates visual
stimulus locking. This crossmodal effect is indicated by diminished peak
correlation coefficients in bimodal trials containing incongruent auditory
input, compared to bimodal congruent trials. Occipitally, the maximum
correlation of the bimodal incongruent grand average ($r = 0.0612$) is
almost 25\% smaller than the peak of the bimodal congruent condition ($r =
0.0795$).  Time lags between 98 ms and 168 ms have statistically less
locking in incongruent than congruent bimodal trials ($p \textless 0.05$).
The unimodal visual condition, in contrast, shows the same correlation as
the bimodal congruent condition ($p \textgreater 0.05$). For
centro-parietal EEG, the biggest condition differences are found for the
first of the two peaks mentioned above. For time bins between -90 and -30,
and -10 and 72 ms of lag, bimodal incongruent coefficients are
significantly smaller than for the bimodal congruent or unimodal visual
conditions ($p \textless 0.05$, see Fig. \ref{waveform}B). Likewise, the
magnitude of the subsequent anti-correlation is lower for bimodal
incongruent than both other conditions for lags between 168 and 198 ms (p
\textless 0.05). It seems then that incoherent auditory input suppresses
visual stimulus locking during multimodal stimulation.


\begin{SCfigure}[][!htb]
\includegraphics[width=0.5\textwidth]{part_sg/figure/Figure6_XcorrERPTimeWin_III.png}
\caption[Temporal Progression of Stimulus
Locking.]{\protect\textbf{Temporal Progression of Stimulus Locking.} \textbf{(A)} Each row shows one grand
average waveform analysis cross-correlogram (waveforms calculated from
average of bimodal congruent and incongruent and unimodal visual
conditions), which was computed within a 500 ms shifting time window with
250 ms overlap. From bottom to top, the window is shifted from stimulus
onset to the end of stimulus presentation. Correlation coefficients are
color-coded. \textbf{(B)} The maximum waveform analysis correlation
coefficients (y-axis) are plotted as a function of time for OZ (left panel)
and CPZ (right panel). Peaks were determined within a region of interest
(0-250 ms lag) for each time window from the grand average of each
condition. The resulting data points are shown in light-colored lines for
bimodal congruent (BC, green), bimodal incongruent (BI, red) and unimodal
visual (UV, dashed blue). The according linear fits for each condition are
shown in darker thick lines in the corresponding color scheme. Stars at
line ends indicate significant linear trends (p\textless0.05) within a
condition, and starred brackets indicate significantly different slopes
between two conditions
}\label{timelocking} \end{SCfigure} 



Applying the spectrotemporal analysis method, no condition differences were
found for the stimulus-locking effects mentioned above. However, the
unimodal visual, bimodal congruent and bimodal incongruent condition all
showed significant peaks in the frequency ranges and time-lags reported for
the average condition reported above: peak cross-correlation values for the
postive peak within the limits of 0-150ms lag and 20-30 Hz were 0.018,
0.020, and 0.17, respectively; for the negative peak they were -0.035,
-0.034, and -0.035 within lag limits 150-500 ms and 8-20 Hz. Thus,
spectrotemporal stimulus locking to visual stimulus dynamics seems to be a
bottom-up driven cortical response independent of input to other
modalities.

\subsection{Do the Effects Change over Time?}

To examine the progression of stimulus locking over the duration of
stimulus presentation, we evaluated cross-correlations as a function of
stimulus time. To this end, correlation coefficients were determined within
a shifting time window (see methods in Section \ref{part_sg_mm}). Before
investigating unimodal visual, bimodal congruent and bimodal incongruent
conditions separately, we again examined their condition average.

Fig. \ref{timelocking}A shows the waveform analysis results for selected occipital and
central electrodes. At occipital recording sites, stimulus locking
increases with progressing stimulus time \textemdash immediately after
stimulus onset, correlation coefficients are very small and only start to
increase after approximately 2 seconds of visual stimulation. In contrast,
the magnitude of stimulus locking at central sites is not dependent on
time. Locking is roughly constant, with no apparent systematic change.
Overall, we see different time courses of visual stimulus locking at
occipital and centro-parietal electrodes.

To quantify the temporal progression and obtain a better comparison across
conditions, we extracted the maximum correlation coefficient for each time
window. These time series were then fitted using linear regression. Results
are shown for bimodal congruent and incongruent and unimodal visual
conditions at selected occipital and central electrodes (Fig. \ref{timelocking}B). At the
occipital site, there is a significant linear increase for the bimodal
congruent (regression coefficient = 0.0027, $r^2$ = .5; $p \textless 0.01$)
and unimodal visual condition (regression coefficient = 0.0034, $r^2$ =
.75; $p \textless 0.01$). In contrast, locking in the bimodal incongruent
condition does not follow any linear trend (regression coefficient =
0.0007, $r^2$ = .12 ; $p \textgreater 0.05$). As expected from these data,
a comparison of bimodal congruent and incongruent slopes yields highly
significant results ($p \textless 0.01$), while there is no statistical
difference between bimodal congruent and unimodal visual stimuli. At our
representative central site, there are no increases in locking over time in
any condition ($p \textgreater 0.05$). There are, however, condition
differences regarding the magnitude of stimulus locking: although all
conditions show significant stimulus locking ($p \textless 0.01$), peak
correlations are significantly higher for bimodal congruent than
incongruent and unimodal visual stimuli ($p \textless 0.01$). In summary,
our results suggest two distinct multisensory interactions at work. At
central recording sites we see a sensitivity to congruence between auditory
and visual input streams, and at occipital sites a further time-dependent
facilitation of visual stimulus-locking is evident that is suppressed when
conflicting auditory information is present.



In the case of the spectrotemporal analysis approach, we first defined
frequency bands of interest, guided by the visual stimulus locking results
reported above (8-20 Hz and 20-30 Hz). The sum of spectrotemporal power in
these bands was then correlated with stimulus speed using the shifting time
window method just described, and peak correlation values (minimum in the
case of the anti-correlation, and maximum for the correlation) were
extracted from lag limits defined from the earlier results (200-300 ms and
0-150 ms, respectively). Finally, these time series were fitted using
linear regression. No linear trend was seen for the 8-20 Hz band at
occipital sites, but effect sizes were in the same range for bimodal
congruent, incongruent, and unimodal visual conditions. In the case of the
beta correlation, no linear trend was found for bimodal congruent and
incongruent conditions at occipital sites ($p \textgreater 0.05$); however,
locking of spectrotemporal power to unimodal visual stimuli was seen to
increase over time in OZ ($p \textless 0.05$, $r^2$ = 0.27) and O1 (p
\textless 0.01, $r^2$ = 0.47).

