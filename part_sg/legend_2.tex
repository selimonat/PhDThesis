\textbf{Two Correlation Analysis Approaches.} Spectrotemporal and waveform
correlation methods are illustrated on the left and right, respectively.
Exemplary stimulus trajectory and single-trial EEG waveforms are shown in
the centre, with a gray bar indicating the temporal range used in both
analysis approaches. In the case of spectrotemporal analysis, each trial is
transformed into a time-frequency representation (see text for details) and
then correlated with the speed profile of the corresponding stimulus. The
waveform analysis method involves creating an ERP from all trials
corresponding to a single stimulus, and then correlating this waveform
directly with the stimulus speed. Peaks at positive time lags indicate that
the EEG power or waveform follows the temporal dynamics of the stimulus, in
other words that the EEG has locked to the stimulus. In both approaches,
correlograms are then averaged over stimuli. Grand averages are created by
taking the median over subjects.
