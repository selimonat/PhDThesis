In the previous Chapter (Section \ref{oi_motionlock}), we demonstrated that
the dynamic natural stimulation has a major effect on the temporal profile
of the cortical responses recorded in cats. By comparing the evoked optical
activity waveform to the motion profile of dynamic natural movies we have
shown that a good deal of the variations in brain activity is common to
both of these signals. Moreover we provided evidence that the cortical
machinery processing these dynamic input is able to integrate distant
contextual information (Section \ref{oi_local_fac}) when the context is
presented within the same sensory modality. In order to more fully
understand the motion locking phenomenon \citep{kayser2004b} and to test
its generalizability to humans, we designed a new set of experiments and
benefited from the unequalized temporal resolution of EEG.
Electroencephalogram is a well-established recording technique and offers
recording of high frequency components (up to 100 Hz in our case).

To stimulate dynamically the visual system, we used a single Gabor wavelet.
The choice of the stimuli was motivated by its well-parametrized nature. We
modulated one of its parameters (for example its orientation) across time
in such a way that the temporal trajectory was similar to the motion
signals of our natural movies. Using this dynamic stimulation we
investigated the motion-locking phenomenon at different frequency bands.
Additionally, to test the effect of contextual information, we complemented
the visual input with an extra-modal dynamic sensory signal. To this end, a
frequency modulated auditory signal was used. The modulation of the
frequency was either the same or different from the motion of the visual
stimulation that was simultaneously presented. This resulted in a dynamic
multimodal Gabor wavelet that was perceptually either congruent or
incongruent across different modalities. 

Our results showed that global brain activity, as recorded by EEG, locked
to the dynamic stimulation at temporal frequencies way above those present
in the dynamic stimulation. Moreover we demonstrated that extramodal
auditory input has modulatory effects on the potentials recorded from
electrodes reflecting presumably the activity of early visual areas. 

These observations extend the results presented in the last Chapter: First
of all we show that the motion locking phenomenon is not specific to the
cat visual cortex but it can also be observed in humans. Moreover we show
that the connectivity of the central nervous system subtends the
integration of contextual information even when it is presented in a
distributed way across different sensory modalities.
