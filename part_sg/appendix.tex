
\section{Behavioural Pilot Study} 

\label{app_eeg} 



A behavioural pilot experiment was conducted to determine whether and how
stimulus parameters affect the perceptual detectability of audiovisual
congruence. 



We tested different visual and auditory feature dimensions, as well as
slower and faster modulation frequencies. Otherwise, audiovisual stimuli
were constructed as described for the main experiment. The Gabor patch
could be modulated in one of 6 dimensions: size (100-200 pixels), contrast
(0.1-0.5), orientation (180 $^{\circ}$ around horizontal axis), frequency
($\approx$0.2-1 cycles/$^{\circ}$), phase (0-360 $^{\circ}$) or color
saturation (0.1-0.5 along the red-green axis in DKL color space). The tone
was either amplitude- (0.1-1) or frequency- (Carrier Frequency $\pm$ 40 Hz)
modulated.  In the latter case, the carrier frequency was chosen randomly
from a uniform distribution (100-500 Hz). Modulation frequencies were bound
by either 0.7, 1 or 1.3 Hz. The combinations of parameter triplets (visual,
auditory, cutoff) were fully balanced. 

Additionally, each combination was shown an equal number of times in
congruent and incongruent stimuli. Movies were created out of 6
trajectories per modulation frequency, resulting in a total number of 432
movies.  Button presses were recorded from 9 naive subjects.  The whole
stimulus set was divided and balanced between pairs of subjects, i.e. each
subject saw 216 movies of a single balanced set. Paradigm, trial
organization and instruction were identical to the main experiment
(simultaneous trials). Unimodal stimuli were not shown. 



On average, subjects responded correctly on 74,9 \% of all trials (standard
deviation: 9.3 \%). The analysis of responses indicated that each subject
performed well above chance (exact Binomial test, $p \textless 0.01$). A
comparison of different visual features, auditory features and modulation
frequencies revealed no significant differences ($\chi^{2}$, $p \textless
0.05$).



We conclude that subjects are capable of distinguishing audiovisual
congruent from incongruent stimuli. Furthermore, sensitivity to temporal
congruence does not seem to be dependent on the specific low-level features
through which the temporal structure is conveyed.


