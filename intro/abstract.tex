We can affirm to apprehend a system in its totality only when we know how
it behaves under its natural operating conditions. However, in the face of
the complexity of the world, science can only evolve by simplifications,
which paradoxically hide a good deal of the very mechanisms we are
interested in. On the other hand, scientific enterprise is very tightly
related to the advances in technology and the latter inevitably influences
the manner in which the scientific experiments are conducted. Due to this
factor, experimental conditions which would have been impossible to bring
into laboratory not more than 20 years ago, are today within our reach.
This thesis investigates neuronal integrative processes by using a variety
of theoretical and experimental techniques wherein the approximation of
ecologically relevant conditions within the laboratory is the common
denominator. The working hypothesis of this thesis is that neurons and
neuronal systems, in the sensory and higher cortices, are specifically
adapted, as a result of evolutionary processes, to the sensory signals most
likely to be received under ecologically relevant conditions. In order to
conduct the present study along this line, we first recorded movies with
the help of two microcameras carried by cats exploring a natural
environment. This resulted in a database of binocular natural movies that
was used in our theoretical and experimental studies. 

In a theoretical study, we aimed to understand the principles of binocular
disparity encoding in terms of spatio-temporal statistical properties of
natural movies in conjunction with simple mathematical expressions
governing the activity levels of simulated neurons. In an unsupervised
learning scheme, we used the binocular movies as input to a neuronal
network and obtained receptive fields that represent these movies optimally
with respect to the temporal stability criterion. Many distinctive aspects
of the binocular coding in complex cells, such as the phase and position
encoding of disparity and the existence of unbalanced ocular contributions,
were seen to emerge as the result of this optimization process. Therefore
we conclude that the encoding of binocular disparity by complex cells can
be understood in terms of an optimization process that regulates activities
of neurons receiving ecologically relevant information. 

Next we aimed to physiologically characterize the responses of the visual
cortex to ecologically relevant stimuli in its full complexity and compare
these to the responses evoked by artificial, conventional laboratory
stimuli. To achieve this, a state-of-the-art recording method,
voltage-sensitive dye imaging was used. This method captures the
spatio-temporal activity patterns within the millisecond range across large
cortical portions spanning over many pinwheels and orientation columns. It
is therefore very well suited to provide a faithful picture of the cortical
state in its full complexity. Drifting bar stimuli evoked two major sets of
components, one coding for the position and the other for the orientation of the
grating. Responses to natural stimuli involved more complex dynamics,
which were locked to the motion present in the natural movies. In response
to drifting gratings, the cortical state was initially dominated by a
strong excitatory wave. This initial spatially widespread hyper-excitatory
state had a detrimental effect on feature selectivity. In contrast, natural
movies only rarely induced such high activity levels and the onset of
inhibition cut short a further increase in activation level. An increase of
30 \% of the movie contrast was estimated to be necessary in order to
produce activity levels comparable to gratings. These results show that the
operating regime within which the natural movies are processed differs
remarkably. Moreover, it remains to be established to what extent the
cortical state under artificial conditions represents a valid state to make
inferences concerning operationally more relevant input.

The primary visual cortex contains a dense web of neuronal connections
linking distant neurons. However the flow of information within this local
network is to a large extent unknown under natural stimulation conditions.
To functionally characterize these long-range intra-areal interactions, we
presented natural movies also locally through either one or two apertures
and analyzed the effects of the distant visual stimulation on the local
activity levels. The distant patch had a net facilitatory effect on the
local activity levels. Furthermore, the degree of the facilitation was
dependent on the congruency between the two simultaneously presented movie
patches. Taken together, our results indicate that the ecologically
relevant stimuli are processed within a distinct operating regime
characterized by moderate levels of excitation and/or high levels of
inhibition, where facilitatory cooperative interactions form the basis of
integrative processes. 

To gather better insights into the motion locking phenomenon and test the
generalizability of the local cooperative processes toward larger scale
interactions, we resorted to the unequalized temporal resolution of EEG and
conducted a multimodal study. Inspired from the temporal properties of our
natural movies, we designed a dynamic multimodal stimulus that was either
congruent or incongruent across visual and auditory modalities. In the
visual areas, the dynamic stimulation unfolded neuronal oscillations with
frequencies well above the frequency spectrum content of the stimuli and
the strength of these oscillations was coupled to the stimuli's motion
profile. Furthermore, the coupling was found to be stronger in the case
where the auditory and visual streams were congruent. These results show
that the motion locking, which was so far observed in cats, is a phenomenon
that also exists in humans. Moreover, the presence of long-range multimodal
interactions indicates that, in addition to local intra-areal mechanisms
ensuring the integration of local information, the central nervous system
embodies an architecture that enables also the integration of information
on much larger scales spread across different modalities.

Any characterization of integrative phenomena at the neuronal level needs
to be supplemented by its effects at the behavioral level. We therefore
tested whether we could find any evidence of integration of different
sources of information at the behavioral level using natural stimuli. To
this end, we presented to human subjects images of natural scenes and
evaluated the effect of simultaneously played localized natural sounds
on their eye movements. The behavior during multimodal conditions was well
approximated by a linear combination of the behavior under unimodal
conditions. This is a strong indication that both streams of information
are integrated in a joint multimodal saliency map before the final motor
command is produced. 

\nopagebreak[4]{
The results presented here validate the possibility and the utility of
using natural stimuli in experimental settings. It is clear that the
ecological relevance of the experimental conditions are crucial in order to
elucidate complex neuronal mechanisms resulting from evolutionary
processes. In the future, having better insights on the nervous system can
only be possible when the complexity of our experiments will match to the
complexity of the mechanisms we are interested in.}
