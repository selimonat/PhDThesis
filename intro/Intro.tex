%\epigraphhead{
%\epigraph{So happiness is not made by a chemical. That would be the same as
%treating a violin sonata as nothing but rubbing horse hair on strings of
%cat gut in order to make a wooden box resonate. Violin makers have to know
%their materials to make one, and physicians have to know about the brain
%chemicals in order to treat patients, when the chemistry of brains has gone
%wrong, but they can't give us a pill to make us happy.} {
%\cite{freeman1997a}} } \undodrop

\section{Context}

One of the most surprising facts about life is its diversity. Besides the
biological diversity of species which is in the focus of classic and modern
biology across centuries, another aspect that is much less investigated is
just as worth being surprised: \textit{The diversity of the experienced world}.
The way living organisms experience the world is also subject to
extreme diversity. Naturally here, there is no possibility for any claim of this
order to be supported by a rigorous scientific proof because on the one
hand, we have no practical access to how other animals may be experiencing
the external world and on the other hand, we simply have no idea of how and
why we, as humans, do experience the world as we do \citep{oregan2001a}.
However, given the enormous diversity of animals and their sensory organs,
one may make the claim that the world as it would potentially be
experienced by living beings is infinitely complex \citep{freeman2000a}.
As a matter of fact we, animals, do not experience this infinity but only a
part of it. And interestingly different animals do it differently. What is
the origin of this diversity?  More precisely, what determines that a
sensory system should be sensitive to a given set of aspects of the
environment and ignore, to a good extent, others? 



One answer, which is certainly the Modern Age's most prevailing one, would
be that these diverse selectivities result from the clockworks of evolution
shaping different sensory organs and epithelia in order to
\textit{represent} different and specific aspects of external world by
means of genetic and/or epigenetic mechanisms. Each unique sensory domain
could then be seen as different pair of goggles directed to different parts
of the same reality, which is the environment. Said another way, this view
has a specific answer to the popular chicken-egg problem: Which one, the
sensory system responsible for sensation or the environment, did come
first?  According to this view, the environment should have been there
before the sensory system evolved to process and represent the incoming
information by virtue of natural selection \citep{varela1992a}.  Therefore
this account of sensory diversity presupposes that the world has
\textit{pregiven} properties which are accessible to the organisms
directly. It assigns a totally passive observer status to the organism with
the absolute capacity of having access to different properties of the
environment. 


But here we should stop and ask what an incoming signal is . Does the
environment let itself be experienced by observers without any
constraints? The answer is most certainly negative given the fact that the
input to the nervous system depends not only on the environment but also on
the animal's more general properties such as its body shape, locomotion,
speed, head shape, inter-ocular distances, dexterity among many more
traits.  The obvious relationship of the body shape and multitude of
potentials it generates or restricts, suggests that the problem of sensory
diversity is not a simple problem which could be explained by a
representation of the external world and further attunement of these
representations. One should recognize the primacy of animals being animated
beings. As a fact of definition, animals are self-animated beings and the
capacity of self motion is their unshared trait among all living beings.
Importantly, the nervous system is an invention of the Animal Kingdom: that
is to say that the link between action and sensation is created via the
nervous tissue.

Strikingly, a representationist account of sensory diversity does not make
recourse to this defining property of animal nature. Hence it ignores to a
large extent their dynamic nature. It simply focuses on the sensation and
creates a simple picture of the relationship between the environment and
the organism. Such a view reduces the animal to a disembodied agent where
it simply receives signals from the external world, to which he has full
access from the beginning. However considering the organism in its embodied
context, we accept the fact that what reaches central sensory neurons
primarily depends on, the environment, and also importantly, how motor
neurons make use of the very specific body, which in turn depends on the
very shape of the body and the manner in which it interacts with the
environment.  Therefore it becomes increasingly harder to defend the
position of a pregiven world that is represented by the organism without
taking into account the shape and design of the animal. The shape and
design of the animal specifies the sensory motor space of the animal; these
are the aspects which an \textit{embodied} science of mind can not
disregard \citep{clark1999a}.  Let's illustrate this view with an excellent
passage of \textit{The Embodied Mind} \citep{varela1992a}: 
 
 
 \begin{quote} 
 
	It is well known that honey bees are trichromats whose spectral
	sensitivity is shifted toward the ultraviolet. It is also well known
	that flowers have contrasting reflectance patterns in ultraviolet
	light. Consider now our "chicken-and-egg" question [...]: Which came
	first, the world (ultraviolet reflectance) or the image (ultraviolet
	sensitive vision). Most of us would probably answer with little
	hesitation, The world (ultraviolet reflectance). It is therefore
	interesting to observe that the colors of flowers appear to have
	\textit{coevolved} with the ultraviolet sensitive, trichromatic vision
	of bees.

	Why should such coevolution occur? One the one hand, flowers attract
	pollinators by their food content and so must be both conspicuous and
	yet different from flowers of the other species. On the other hand,
	bees gather food from flowers and so need to recognize flowers from
	distance. These two broad and reciprocal constraints appear to have
	shaped a history of coupling in which plant features and the
	sensorimotor capacities of bees coevolved. It is this coupling, then,
	that is responsible for both the ultraviolet vision of bees and the
	ultraviolet reflectance patterns of flowers. Such coevolution therefore
	provides an excellent example of how environmental regularities are not
	pregiven but are rather \textit{enacted} or brought forth by a history
	of coupling."

 \end{quote}

%[structural coupling]

This example emphasizes an important concept of the so-called
\textit{enactivist} point of view: \textit{structural coupling}. In this
example, bees and flowers are two components of a system. These, by virtue
of their coupling codetermine the fate of each other. In this paradigm, the
idea of a world with pregiven attributes that the agent needed to represent
does not occupy a central position. Admittedly the agent needs a sensorial
system that reacts to the external world, but the principle characteristics
of the sensory-motor system are not organized for the representation of the
external world. Both the environment and the agent are embodied within a
dynamic system that enacts the observed visual sensitivities of bees. Bees
do not need to know the exact properties of the external world. They don't
need to create a sensory system which faithfully represents the world. It
suffices for the subject to distinguish the relevant objects from the
background of meaningless stimulation. This example shows that the
\textit{representationism} (at least at its strong definition) is not a
necessary component of the biological and sensorial diversity generation.
From this perspective, the fact that different animals have different
sensitivities with respect to the environment results from their specific
history of structural coupling, which endows them with basic capacities for
solving problems related to their survival.


\subsection{Representationism in Cognitive Science}

Interestingly, representationism (takes the label of \textit{Cognitivism}
in the field of Cognitive Science) translates one-to-one to the field of
Cognitive Science and constitutes the major paradigm and working
hypothesis. Here also, representationism is based on the very assumption
that the world has pregiven properties which are then represented by the
mind in order to solve different tasks or to produce intelligent behavior.
Accordingly, any intelligent behaviour finds its sources on the
computations which are done on the symbolic level based on the
representations of an external world. Clearly today, representationist
accounts of the intelligent behaviour can be found in many different
contexts. It wouldn't be erroneous to assert that this view occupies the
defining zeitgeist in Cognitive Science.



Let's illustrate this paradigmatic view with a real-world example and take
the task that an outfield baseball player has to solve. The question of
interest here is to understand how outfield players in baseball do know
where to run, starting from the point where the ball hits the bat. Given
the high speed levels the ball reaches and the large distances between the
ball and the outfield player, it is reasonable to assume that the visual
cues, such as for example stereoscopic depth, are not reliable. Therefore
the question of how this is achieved raises interest. 


Classically this task was thought to be solved by humans by realizing
unconscious complex mathematical computations in order to derive the
trajectory of the ball based on variables extracted from the external world
such as for example its speed, acceleration \citep{mcleod1993a}.
\cite{mcbeath1995a} provided a different account that relies on the idea
that when outfield players start running to catch the ball, they choose a
trajectory for their own body which linearizes the trajectory of the ball
according to the optical information they receive. This is an important
switch in our conceptualization of the intelligent behaviour as it removes
any need to extract parameters from the external world, and thus, the
realization of complex mathematical computations on them. The problem
raised by fact that we are not consciously aware of these computations (see
Mind-Mind problem \citep{Jackendoff1987a}) is also eliminated. Rather it
relies on sensory-motor coupling and puts the body into a central position
by involving the active motor behaviour of the subject: the player using
his sensory-motor capacity coupled dynamically to the external world can
exhibit the required intelligent behaviour without actually representing
any external parameters.


It is generally within the representationist paradigm that Computer
Scientists "inspired" by biological vision are interested in reconstructing
the three-dimensional world based on two-dimensional, and thus ambiguous,
retinal images. For example, a chair may have extremely different
two-dimensional projections onto a two dimensional surface depending on the
angle of view. Here it is of extreme interest to create a visual algorithm
which is able to understand the content of images. Although this task is
extremely easy for the primate's nervous system, computer algorithms are
rather poor in their recognition performance. 

A typical Computer Vision approach for solving this task would be to design
an algorithm for recognizing objects based on their two-dimensional images.
A dataset consisting of images taken from a camera, which passively creates
images, would be the major source of information in order to solve this
task. A classifier trained on these data would learn how to categorize
different images depending on the presence or absence of the target object.
Clearly as shown in this example, the representationist approach ignores
the situatedness of biological intelligence, its dependence on a body and
also on a history of sensory-motor coupling which underpins the performance
of living system, therefore it reduces the visual system to a passively
observing camera. As criticized in \textit{Action in Perception}
\citep{noe2004a}, there is simply no fundamental reason to think that the
data which carries the information about the external world is contained
exclusively in the static two-dimensional retinal images. However, the
outlined approach in computer science is completely static, passively
observing the outside world and radically disembodied. It wouldn't be
erroneous to assert that since the approach outlined by David Marr in
\textit{Vision} \citep{marr1982a}, the paradigm in Computer Vision, one of
the most prominent fields in Cognitive Science, has not made significant
progress toward incorporating embodied aspects of biological vision. 


In experimental fields of Cognitive Science however the importance of the
active bodily participation was very early recognized (even though here
also it never became the major paradigm). One of the classic behavioral
experiments which underlines the primacy of an active bodily involvement
for the development of perceptual skills is the work of \cite{held1963a}.
They showed that a well functioning central nervous system requires the
active motor involvement of the animal during its developmental stages.  In
their now classical experiments, they deprived a group of cats (cats are
mammals commonly used in physiological experiments investigating sensory
systems) of the active motor component of their behavior while they were
exposed to the same visual input patterns as another group of cats which
were not constrained in their active behavior (see Fig.  \ref{basket}A for
a schematic view of the experimental setup).  According to the
representationist view, as both groups are being exposed to the same visual
input, the development of their visual system should be at least comparable
in their performances following the restricting rearing period.  However
the restrained cats were severely impaired in their basic perceptual and
motor skills. Their results showed that passive observation of the outside
world was not sufficient for generating basic sensory-motor capacities used
to solve simple tasks.

Much more recently the FeelSpace\textsuperscript{\textregistered}
\citep{nagel2005a} project conducted in the Neurobiopsychology Department
in Osnabr\"uck is based on a very similar idea. Within this project, humans
learn how to make use of a mechanical belt by carrying it during their
everyday life (Fig. \ref{basket}B). By means of a compass, the belt
vibrates always at the North direction independent of body direction. This
results in the vibration of different parts of the body in contact with the
belt, depending on the orientation of the subject in the North-South axis.
Initially the stimulation by the belt is rather simple and meaningless,
however the tactile stimulation gains a biological meaning within a
relatively short temporal interval (in the order of weeks) through the
building up of a sensory-motor coupling history resulting from the
interaction between the subject and the effect of the externally plugged
device.  Not only that, subjects acquire the capacity of incorporating this
"new sense" on navigational tasks they are required to solve, additionally
their phenomenal perception of the space becomes different, if not
extended.  For example, they report awareness of the orientation of their
dining table with respect to the position of the orientation of their
office desk.  This fact lies indeed on a major divergence point between
enactivist and representationist approaches. Whereas the representationist
perspective has serious difficulties with the incorporation of new senses
to an old body, the enactivist view predicts that the emergence of new
senses should not be given a special status considering that this is how we
make use of our biological sense apparatus during development in any case.


Given the importance and necessity of active bodily participation for the
normal development of a visual system, one might still ask how humans solve
visual tasks while they are not bodily active: for example, in simple
laboratory conditions. Suppose we are given the task of predicting the
success (whether it's ``in" or  ``not") of a basketball player's shot and
provided necessary video material recorded under controlled conditions
while players are aiming the ball to the basket (Fig. \ref{basket}C). How
do humans solve this task. Given that we are presented pixel-based video
data, one approach would be to follow a strategy of finding some heuristics
based on the visual data. For example initial ball trajectory, fore-arm
angle etc. would be the best candidate parameters with predictive value
regarding the ball's trajectory. If this were true, this task could
exclusively be solved by involvement of visual areas in the central nervous
system. 

Using transcranial-magnetic stimulation and measuring the muscular evoked
potentials, \cite{aglioti2008a} tackled this issue experimentally. What
they found was evidence for the large-scale involvement of the central
nervous system, including motor cortex, during this task. The activity of
neurons in the motor cortex corresponding to the arm area (the only area
tested) were found to be involved during this task, suggesting that
subjects were using the parts of their brain usually involved during such
tasks in real-life. Importantly, people who were not experienced were not
able to solve this task, suggesting that what is to be perceived is not a
process which relies on symbolic inference but rather depends on previous
bodily experience. This interpretation directly supports an embodied action
view of the nervous system, which is valid even during conditions that do
not require the usage of a body.


\begin{figure}[htp]
\centerline{\includegraphics[width=\textwidth]{./intro/figures/IntroPanel.png}}
\caption[Embodied Cognitive Science]{ \textbf{Embodied Cognitive Science.}
\textbf{(A)} Apparatus for equating motion and consequent visual feedback
for an actively moving (A) and a passively moved (P) S[ubject] (Fig. and
caption from \cite{held1963a}, my text is indicated within square brackets.
\textbf{(B)} One of the earliest models of the
FeelSpace\textsuperscript{\textregistered} belt. (Image taken from
www.feelspace.de). 13 vibrators provide tactile information depending on
the positioning of the subject along the North-South axis. \textbf{(C)}
Fig. taken from \cite{aglioti2008a}. Showing the visual information that
human subjects were provided during a task which consisted of predicting
whether a sequence of frames representing a basketball player shooting
would be a success or fail. The frames were cut shortly before or after the
ball left the hand of the player. \textbf{(D)} Making of stereoscopic
natural movies by freely moving cats in a natural habitat. A pair of
cameras carried by a cat was connected to two VCR recorders carried by the
experimenter. (See also Fig. \ref{setup} for more detail.) \textbf{(E)} An
example of two frames extracted from movies are shown together with an
artificial stimulus. These kind of stimuli were used in our physiological
and modelling experiments. } \label{basket} \end{figure}

The dominant view and scientific enterprise in the Cognitive Science was
thus far dominated by representationism. To this view is an intrinsic
difficulty of conceiving the organisms as being embodied agents associated.
Recent advances in the field of Cognitive Science make it unlikely that we
can achieve an understanding of the intelligent behavior without conceiving
the organism's dynamic nature and its situatedness in the environment. In
the future this will necessarily result in a shift of our current
computer-mind metaphor toward a more dynamic environment-body metaphor. 

One point must be considered carefully in the discussion of
representationism and taken care to note that the point here is not to
reject the existence of the representations in the brain. Most certainly
the activity of neurons could match to some external variables faithfully.
And as observers we may assign to these neurons a representative role.
Neurons within the motor cortex could very well represent different
movement directions of the arm \citep{georgopoulos1994a}; the activity of
neurons responsive to visual stimulation may very well correspond to, under
certain restricted conditions, different parameters of the visual
stimulation.  For example, the activity of middle temporal area neurons are
known to match to the speed of the stimulus \citep{liu2006a} and the
stimulation of these neurons can cause perceptual shifts on the monkey's
perception in a predictable way in accordance with the feature selectivity
of these neurons \citep{celebrini1994a}. But we should be aware that given
this knowledge we can not deduce that the nervous system as a whole is
operating on the basis of representations. The main point is to emphasize
the fact that sole observation of a correlation between neuronal activity
and parameters of external world do not allow us to conclude that the
cognition and intelligent behaviour is based on manipulation of symbols
extracted from outside world. It is the representationism which is
problematic and not representations. 


\subsection{About this Thesis}
%Or how this thesis inscribe itself with this new paradigm in the Cognitive
%Sciences? 


How would it be possible to study the physiology of sensory systems within
an embodied paradigm? One strategy would be to study sensory systems under
conditions as close to the real-world operational conditions of the animals
as possible. This requires that we, as experimentalists, need to be aware
of the natural embodiment of the animals, and try to develop an
understanding of the role of sensory neurons within a context that makes
sense for their situatedness. 


For this reason, during physiological investigations where I focused on the
visual system of the cat, I aimed to approximate as closely as possible the
visual signals that cats receive during their real-world conditions (Fig.
\ref{basket}D).  To this end, I recorded stereoscopic natural movies from
their perspective using a pair of micro-cameras carried on the head of
freely moving cats viewing their natural habitat. A large part of the
theoretical and experimental results presented in this thesis relies
heavily on these movies. In order to avoid technical complications due to
eye movements, natural movies were recorded in head centered coordinates.
The fact that cats do not make large saccadic movements during active
behaving \citep{einhauser2008a} makes it likely that the movies recorded by
cameras do represent the input to the sensory system within reasonable
limits.


Since the discovery of simple and complex cells in the primary visual
cortex (V1\footnote{complete list of abbreviations are given at the end of
this thesis.}) \nomenclature{V1}{Primary Visual Cortex} by Hubel and Wiesel
\citep{hubel1959a,hubel1962a}, which brought them the Nobel Prize in 1981
for the work they carried out during early 60's, the scientific
investigations of sensory systems are dominated mainly by the usage of
simplified stimulation protocols in physiological experiments. In visual
experiments usage of drifting gratings, random dots, flashed bars and
moving edges were (and are still today) very common for the
characterization of the receptive fields (RF)\nomenclature{RF}{Receptive
Field}. There is no doubt that the spatial and temporal properties of such
visual images and movies are extremely unusual and disparate with respect
to images encountered in the natural environments (see for example Fig.
\ref{basket}E) from the animals perspective. While a natural stimulus such
as, a photograph taken in a wood, contains many different spatial
frequencies and orientations simultaneously, simplified stimuli are
generally constituted of a single orientation and spatial frequency at a
time. The cortical responses to natural and complex stimuli were marginally
investigated by only a few experimenters \citep{creutzfeldt1978a}. 


It is no doubt that technological restrictions during these early times of
electrophysiology favoured the usage of simplified stimuli during
experiments. However, beside these technological constraints, having an
easily controllable and parameterizable source of stimulation was also seen
as a major benefit. Therefore one can say that conceptual drives had also a
significant contribution in increasing the popularity of simplified
stimuli.  Nevertheless, during this initial period of the sensory
neuroscience, usage of such simple stimuli ensured a constant accumulation
of data, results and ideas.  Different theories (for example related to the
emergence of orientation selectivity) which are debated even today, were
proposed during these initial times.  Therefore, the very basics of our
current knowledge on sensory processing relies largely on these experiments
realized under reduced stimulation settings. Consequently little is known
about how sensory neurons process complex natural inputs. Furthermore we
currently are only partially starting to understand whether and how
cortical neurons are evolutionary adapted to process their most common
input patterns. Therefore natural stimuli form the best choice of stimuli
if the search is for the existence of more complex and sophisticated
cortical mechanisms related to the processing of real-world input. 


However, this view does not constitute a consensus among specialists in the
field. \cite{rust2005a} argue that the physiological systems can be studied
by simplified, parameterized stimuli forming the constitutive parts of more
complex signals such as natural images. This view however has one implicit
assumption: namely, it asserts that the knowledge gathered regarding the
functioning of neurons with experiments where simple stimuli is used, can
be transferred without any complication to the cases involving more complex
situations such as processing of real-world inputs. As of today, we do not
have enough data to clearly answer whether this position is wrong or not.
I think however that the results presented in this thesis (Chapter
\ref{part_oi}) contributes to this topic. 


As noted at the very beginning, the sensory input that animals are exposed
to are constrained in many respects. First of all the natural habitat
enormously restricts the possible set of input patterns, that is to say not
all signals are equally likely \citep{simoncelli2001a, barlow1961a,
atick1992a}. Consider for example a set of 10 by 10 pixel-image
representing the light falling within a restricted area of the retina with
255 discrete numbers of grays. It is trivial to show that the total number
of all possible inputs is simply astronomical. However, studying
statistical properties of natural images, one realizes that real-world
input spans only a very limited subspace \citep{chandler2007a}. It is
therefore reasonable to consider the possibility that sensory neurons
developed adaptational skills in order to react optimally to such stimuli.
Such an adaptational strategy has many advantages. First of all,
restricting the processing power to a very limited subspace may increase
reaction times in face of life-threatening situations where the survival is
important.  Moreover, recently it has been proposed that the energy
consumption may be an important constraint for the development and
functioning of the nervous system \citep{laughlin1998a}. From this
perspective RF which minimize the total number of spikes by coupling to the
environment as efficiently as possible are of biological interest as they
reduce the required energy consumption for the transmission of information.


The fact that natural images spans only a limited portion of all possible
stimuli means that the real-world input that the animal receives contains
certain regularities. A milestone study by \cite{laughlin1981a} showed that
the contrast sensitivity of neurons in large monopolar cells in the fly
visual system follows the distribution of contrast values encountered in
the habitat of the fly. The incremental change in spiking activity
following an incremental change in the contrast of the stimulus was not
random nor constant. The change in number of spikes was minimal for
contrast values which were rare in the environment and maximal for the
contrast levels which were most common. This specific contrast sensitivity
curve has the consequence that the neurons in the visual system of the fly
are most sensitive to resolving the most common signals. Additionally from
an information theoretical point of view, this leads to a distribution of
spiking activity having the maximum rate of information because all
possible spike counts occur with equal probability in the long run. It is
thought that this specific way of translating luminance contrast values
into spiking activity is the result of an adaptive process leading to the
efficient processing of natural input. In the visual system of mammals such
adaptive phenomena has also been observed. In a now classical study,
\cite{dan1996a} recorded responses of neurons located in the lateral
geniculate nucleus (LGN) \nomenclature{LGN}{Lateral Geniculus Nucleus} of
cats. This nucleus is the target of the retinal ganglion cells and is
located only one synapse before V1; it therefore stands in the mid-way
between retina and cortex. They showed that the spatio-temporal RF
properties of these neurons are specifically designed to process natural
input efficiently so as to maximize the information content of the spike
trains.  Using another type of stimuli with properties deviating
significantly from the statistics of natural images, such as white-noise
patterns, they observed that the responses of neurons were much less
equally distributed: meaning that they were inefficiently conveying the
information. They interpreted their results as being a consequence of an
adaptational strategy of neurons to natural real-world input. 


The statistical properties of the input signals that reach neurons in the
early cortical areas are not constrained solely by external world.  They
also depend on the animal's body shape, head movements, eye movements and
on the coordination of the same. Therefore in addition to spatial
properties of natural images, temporal properties also may in principle be
equally important. That is to say that the input received from the
beginning on by the sensory neurons about the external world is dynamic and
very idiosyncratic with respect to the animal in question. In the first
part of this thesis (Chapter \ref{part_rf}), I aim to understand how
spatial properties of binocular RFs located in area 17/18
(A17/18)\nomenclature{A17/18}{Area 17/18} of cats and V1 of monkeys are
related to the spatio-temporal statistics of natural signals received by
the visual system. The spatio-temporal properties of the natural movies I
used here are constrained by, both the spatial properties of natural images
and the bodily motion of cats recording these movies. I employed an
unsupervised learning scheme in an artificial neuronal network in order to
derive short-hand mathematical descriptions of learning principles that
lead to the observed binocular complex and simple cells RFs. The learnt RFs
do possess striking similarities to the neuronal RFs observed in
physiological experiments.

In the second part (Chapter \ref{part_oi}), I use a state-of-the-art neuronal
recording method, namely voltage-sensitive dye imaging (VSDI),
\nomenclature{VSDI}{Voltage-Sensitive Dye Imaging} in order to unravel the
dynamics of large numbers of neuronal populations during exposure to
naturalistic movies and I compare these dynamics to the activity levels
evoked by simplified laboratory stimuli (Section \ref{fullfield}). This is
the first time that a large-scale recording method directly measuring
neuronal activity has been used in conjunction with natural stimuli to
investigate neuronal dynamics at the mesoscopic level. My results
demonstrate that the cortical processing structures are indeed adapted to
their real-world input and the artificial stimulation paradigms, using for
example, moving bar stimuli, lead the system towards an operating regime
which is different than natural movies. Moreover, in the last part of the
same Chapter (Section \ref{local}), by presenting natural movies locally
through either one or two apertures, I analyzed how the presence of
contextual information influences the local processing. My results show
that the dense intra-cortical connectivity patterns are at work under
conditions of complex dynamics stimulation.


In the third part of this thesis (Chapter \ref{part_eeg}), inspired by the
dynamic nature of the natural movies, I investigate the processing of
dynamic stimuli and the effect of dynamic extramodal sensory signals in
humans.  Previous results in our working group \citep{kayser2004b} showed
that dynamic stimuli such as movies recorded by cats induces modulations in
the local field potentials (LFP) \nomenclature{LFP}{Local Field Potentials}
at different frequency bands, a phenomenon called motion locking. Due to
the fact that the VSDI is not well-suited for recording of high frequency
oscillations, we recoursed to the EEG which is a well-known and established
recording method.\nomenclature{EEG}{Electroencephalogram} I investigated
how such dynamic stimuli is processed in humans and the effect of
contextual auditory information that was either congruent or incongruent
with the visual input. Our results showed, for the first time in humans,
the existence of motion locking using the method of EEG thus extending the
results obtained in the cat cortex. Moreover I provided evidence that
motion locking in the early visual areas is subject to modification in the
presence of contextual auditory signal if it is congruent to the motion of
the visual stimulus. 


In the fourth and last part of this thesis (Chapter \ref{part_oa}), the
active behavior of humans during real-world input was in my focus. In order
to complement the physiological findings reported above concerning the
existence of short and local integrative mecanisms at the cortical level, I
measured eye movements in human subjects as they were freely observing
naturalistic photographs and investigated how a localized auditory signal
influenced their behavior under natural conditions. My results showed that
the presence of localized sound stimuli shifted the eye movements toward
the side of the auditory stimulus; however, this was more than a simple
orientation behavior and reflected the outcome of an integrative
phenomenoun taking place during the planning of eye movements.
