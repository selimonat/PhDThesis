The main axis of this thesis focused on approximating real-world
operational conditions within the laboratory environment by using complex
and rich natural stimuli. To this end, a considerable amount of energy was
devoted to gathering these stimuli, which were subsequently used in a
variety of experimental and theoretical settings. Together with many recent
reports published during the last few years (see \cite{felsen2005a} for a
review), this thesis provided supporting evidence that the usage of natural
stimuli for the study of central nervous system is both possible and
productive in empirical and theoretical terms, thereby speaking against a
large number of neuroscientists, who are rather reluctant for the entrance
of natural stimuli into experimental fields due to their complex and
uncontrolled nature \citep{rust2005a}. 


The central nervous system of mammals contains many more synapses than the
number of genes that exist in their genome. For example, in humans the
number of genes is estimated to be in the order of 50,000 and no estimation
returns a number larger than 100,000 \citep{venter2001a}. These estimations
are 10 orders of magnitude smaller than the number of synapses in the human
brain \citep{Herculano-Houzel2009a}. Assuming this ratio as a valid rule of
thumb for other species, a genetic determination of the fine connectivity
patterns between specific neurons seems to be implausible. Therefore, a
generic versatile learning rule which does not depend on explicit genetic
definitions, is extremely useful for achieving a given level of selectivity
(e.g. edge orientation) with respect to the external world. Moreover,
absence of the necessity of a hard-wiring strategy via genetic information
enables considerable plasticity with respect to relevant features that are
specifically needed by different animals. 


Learning stable features from the input signals is a good candidate for
achieving this goal. We reported that the binocular disparity coding,
observed in the primary visual area of monkeys and cats can be understood
in terms of extracting stable features from input signals, given that these
are rich in content and represent the natural input. As this learning
scheme involves the temporal dimension, the result of slow feature analysis
depends crucially on the properties of the body and motor repertoire of the
animal in question and enables neurons to be selective for different
features which are relevant for a given animal. For example, the neurons of
animals with fast versus slow locomotion can extract features suitable for
the range of behaviors that their bodies allow without any explicit
definitions in their genomes being necessary. 


In addition to low-level feature selectivity that neurons in the early
sensory cortices express, the operations carried out across the visual
ventral pathway involve considerably complex computations. For example,
neurons located in the temporal cortical areas of monkeys have RFs that are
selective for complex patterns representing specific objects. Moreover,
this selectivity is to a large extent invariant to the viewing angle of
objects and their position in the visual field \citep{gross1972a,
rolls2006a}. Even higher in the hierarchy across the visual ventral
pathway, the hippocampus occupies the final point of neuronal convergence
of a large-number of cortically interconnected areas. Some of the
hippocampal neurons called place cells \citep{okeefe1971a} fire when an
animal is at a given specific location in the space irrespective of the
precise viewing angle of the animal in that portion of the space. This
means that visual input is processed in such a way that these neurons
extract the view-angle-invariant information from the input signals. In
simulation studies such non-trivial RF properties were obtained using the
same temporal stability based learning rule. For example, concerning the
temporal cortical neurons, \cite{einhauser2005a} showed that the
translation invariant object recognition can be achieved by extracting
stable features from a series of object pictures taken from different angle
of view. Furthermore, \cite{wyss2006a,franzius2007a} provided evidence that
extracting stable features in an hierarchical manner from the visual
signals recorded by a robot exploring either a real-world \citep{wyss2006a}
or a virtual environment \citep{franzius2007a}, place-selective neurons
emerge at the top layers. It is therefore likely that the functioning of
the ventral visual pathway, however complex the computations carried by
these neurons are, may indeed be based on a simple learning rule iterated
over and over across cortical areas. 


Cortical neurons also exhibit many different other specializations and
there is no doubt that the complete operation of the cortical network
constituting the visual areas cannot be totally captured by a simple formal
expression. For example, the hierarchical organization of visual areas
reflects only one side of the facts, because it is well known that the
bidirectionality in the cortico-cortical connections is rather a general
rule of organization \citep{felleman1991a}. These connections are also
distributed across different modalities, effectively multiplying the
integrative power of neurons. Moreover today, research on crossmodal
integration provides experimental evidence suggesting that different
sensory pathways classically thought to be segregated \textemdash at least
in the early sensory areas \textemdash may actually overlap considerably,
resulting in the integration of different sensory signals very early in the
hierarchy \citep{dehay1988a,driver2008a,macaluso2004a}. Adding to the
complexity, cortical areas also contain a dense web of intra-areal
connections, which are highly organized \citep{rockland1982a}. According to
our current knowledge, as early as in the primary visual area, the
organization of these local connections are thought to reflect the general
statistical properties of the input signals founding the basis of local
integrative processes \citep{betsch2004a}. There is no doubt that more
experimental research is needed in order to understand better the cortical
function.


Using the physiological recording technique of VSDI \citep{grinvald2004b},
we aimed to understand the processing of natural input in the visual area.
We were able to analyze the activity of a relatively large cortical area
with reasonably good spatial and temporal resolution. Even though the
spatial resolution of this technique is not strong enough to resolve single
neurons, it however provides the possibility of recording a large number of
neuronal masses directly and simultaneously. Importantly, it captures the
global cortical state spanning many columns and pinwheels, and therefore it
is much less susceptible with respect to any bias that the experimenter may
introduce inadvertently during electrophysiological recordings. For
example, whereas it is simply not possible to record electrophysiologically
from a silent neuron, VSDI would not be blind to their presence as it
faithfully captures the big picture.

We found major differences in the cortical activity levels evoked by
natural versus simple laboratory stimuli. The physiological measurements
indicated that during the processing of commonly used simple drifting edge
stimuli, the visual cortex is driven toward a state characterized by an
excess of excitation, leading the system into a strong hyperexcitatory
state. The activity levels in response to natural stimuli were much more
moderate in comparison. We interpreted these results as being the
consequence of an evolutionary adaptational process which leads to the
efficient processing of naturally occurring sensory inputs in terms of
energy consumption.  Within this framework, the fact that the processing of
natural stimuli does not require high levels of net excitation can be
conceived as the result of specialization of the cortical circuits.


Beside these observed differences in responses to artificial and natural
stimuli, it is also important to emphasize that these discrepancies may
create difficulties for interpreting the conclusions derived from studies
where simple, artificial stimuli are used. Let's take as an example, the
emergence of orientation selectivity in the primary visual cortex. It is by
far the most investigated case study in cortical computation since the
early work of Hubel and Wiesel \citep{priebe2008a}. However, many of these
studies are based on the results obtained by laboratory stimuli. Taken
together with the results presented here, it is reasonable to doubt the
generalizability of these results. For example, the orientation selectivity
measured under artificial conditions, may well be an underestimated
approximation of the real selectivity of neurons. The excess of excitation
in response to grating stimuli (or/and the excess of inhibition in response
to natural movies) could in principle lead to such differences in the
tuning of neurons. These considerations should lead us to be more prudent
for the interpretations of results.


In the experimental paradigms which are currently used, the final cortical
response is obtained after averaging a large number of repeated trials,
thus averaging out the noise component. By analyzing the data in a trial by
trial basis, \citep{arieli1996a} has shown that a considerable part of the
cortical variability in response to stimulation is due to the spontaneous
activity levels, rather than the external stimulation. This means that the
complexity of the signals can only be underestimated with the experimental
paradigms we are currently using. We showed that even following averaging a
large number of trials, complex spatio-temporal activity dynamics
constitute a fundamental characteristic of the responses of early sensory
cortex to input signals, be it complex or simple in nature. Propagating
waves were consistently observed during both grating and natural
conditions. This fact raises the question of how a system characterized by
complex dynamics can self-coordinate in order to integrate different
sources of information. We found that these complex dynamics representing
the sensory input were, besides being locked to the dynamics of the
external stimuli, also very sensitive to internal signals conveying
contextual intra- or extramodal information. The presence of contextual
visual information was found to support the activity levels of distant
neuronal populations. In the same vein, an additional auditory signal was
found to increase the strength of the motion locking of the distant visual
cortical areas. Therefore it is tempting to state that the cortical
machinery is extremely prone to integrate different sources of information.
Given this strong integrative power, it is however remarkably robust in
conserving its stability under normal operating conditions.



As the presence of integrative phenomenon at the neuronal level does not
determine unambiguously the nature of the motor output at the behavioral
level, the effectiveness of the integrative processes can be best studied
in the behavioral level. Interestingly, the results of my psychophysical
experiments paralleled the physiological results. Instead of having
different sensory modalities in competition with each other for the control
of the behavior, we reported that different signals were integrated before
the production of the motor decision. Remarkably, the integration scheme
was well predicted by a linear combination of different sources of
information. It is surprising that the presumably complex dynamics
underlying the functioning of neuronal populations, results in interactions
which are predicted by simple linear models on the behavioral level. 


This thesis coincided with a transition period in systems neuroscience, a
transition period consisting of the introduction of more and more
elaborate, complex and clever experimental conditions within the laboratory
settings. In my opinion the usage of complex natural stimuli for the
investigation of sensory systems represents only the starting point of this
global trend. Yet many exciting findings have already been brought to light
concerning the adaptational mechanisms used by cortical neurons in sensory
cortices. As the central nervous system of mammals is evolved for
maintaining of the survival in face of extremely uncertain environmental
conditions, these findings may not be utterly surprising from an
evolutionary perspective. For example, the presence of a prey, once
detected by sensory neurons, must be reliably communicated to other regions
of the brain, and in order to avoid unnecessary energy consumptions and to
guaranty the survival, the probability of a prey or a predator at a given
location needs to be evaluated pertinently based on very limited number of
trials \citep{glimcher2003a}. Therefore very robust neuronal mechanisms
need to be at play to evaluate the risks of actions and their outcomes.
However, such mechanisms can not be uncovered when the complexity of the
our experimental setups do not match to the complexity of our subjects.
Therefore  conducting better and richer experiments in the future will be
our chance to learn new insights about the way the brain functions.
