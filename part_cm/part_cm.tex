
\section{Context} 

In everyday life, our brains decide about the relevance of huge amounts of
sensory input. Further complicating this situation, the input is
distributed over different modalities. This raises the question of how
different sources of information interact for the control of overt
attention during free exploration of the environment under natural
conditions.

Previous Chapters dealt with the question of how the presence of different
sources of information are integrated at the cortical level. In the first
Chapter we were interested in the integration of binocular information. The
second Chapter dealt with the question of how cortical connectivity
underpins the integration of spatially separated visual information. In the
third Chapter we extended this question to different modalities and
investigated how signals originating from different modalities are
integrated in the case of dynamic input. During these Chapters we ignored
to a large extent the behavioral component of these underlying neuronal
processes. In this Chapter we investigate how behavior is influenced by the
presence of different sources of input. The integration we observed at the
cortical level leaves open the question of how the behavior is influenced.
In the behavioral level, different modalities may work independently or
interact to determine the resulting motor output. 


In order to investigate this question, we focused on human overt attention
by means of eye tracking methodology. To collect information from the
environment, humans extensively use their visual system by deploying their
attentional resources to different regions of a scene. During the
exploration of a scene, fast eye movements called saccades intermingling
with relatively stable fixation periods form the basis of this process. By
measuring the density of the fixated points, one can obtain an empirical
estimate of the saliency of different regions on an image. Such empirical
saliency maps can later be used in order to evaluate different integration
mechanisms.

We presented natural images and lateralized natural sounds in a variety of
conditions, and measured the eye movements of human subjects. We show that,
in multimodal conditions, fixation probabilities increase on the side of
the image where the sound originates showing that, at a coarser scale,
lateralized auditory stimulation topographically increases the salience of
the visual field. However, this shift of attention is specific because the
probability of fixation of a given location on the side of the sound scales
with the saliency of the visual stimulus, meaning that the selection of
fixation points during multimodal conditions is dependent on the saliencies
of both auditory and visual stimuli. Further analysis shows that a linear
combination of both unimodal saliencies provides a good model for this
integration process, which is optimal according to information theoretic
criteria. 

Our results support a functional joint saliency map, which integrates
different unimodal saliencies before any decision is taken about the
subsequent fixation point. These results provide guidelines for the
performance and architecture of any model of overt attention that deals
with more than one modality. 

\section{Introduction}

How are different sources of information integrated in the brain while we
overtly explore natural multimodal scenes? It is well established that the
speed and accuracy of eye movements in performance tasks improve
significantly with congruent multimodal stimulation \citep{corneil1996a,
corneil2002a, arndt2003a}. This supports the claim that sensory evidence is
integrated before a motor response. Indeed, recent findings indicate that
areas in the brain may interact in many different ways \citep{driver2000a,
macaluso2005a}. The convergence of unimodal information creates multimodal
functionality \citep{meredith1986a, beauchamp2004a} even at low-level areas
traditionally conceived as unimodal \citep{macaluso2000a, calvert1997a,
ghazanfar2005a}; evidence is also currently mounting for early feedforward
convergence of unimodal signals \citep{molholm2002a, fu2003a, foxe2005a,
kayser2005b}. Little is known, however, about integration processes under
the relevant operational \textemdash i.e. natural \textemdash conditions.
Most importantly, we lack a formal description of the integration process
during overt attention. It is important to develop formalizations using
behavioral data as it reflects the final outcome of the processes within
the CNS. \nomenclature{CNS}{Central Nervous System} 

Current models of overt attention are based on the concept of a saliency
map: A given stimulus is first separated into different feature channels;
after the local contrast within each feature space is computed, these
channels are then combined, possibly incorporating task-specific biases
\citep{koch1985a, itti2001a, parkhurst2002a, peters2005a}. The selection of
the next fixation point from a saliency map involves a strong
non-linearity. This is typically implemented as a winner-takes-all
mechanism. The timing of this non-linearity with respect to the integration
of multiple feature channels crucially influences both the performance and
structure of the resulting system. In principle, three idealized multimodal
integration schemes can be considered. 

\textit{Early interaction} \textemdash The information from different
modalities could be integrated early, before the computation of a saliency
measure and the selection of a fixation point. Indeed, many studies provide
evidence for an interaction between signals in early sensory cortices
\citep{calvert2000a, kayser2005b}. An integration in the form of
interaction may be the result of modulatory effects of extra-modal signals
during computation of saliency maps within a given modality. Or
alternatively such an interaction may be caused by multiplicative
integration of unimodal saliencies after the saliency maps for each
modality are computed. The selection of the most salient point after
crossmodal interaction has taken place imposes an expansive nonlinearity.
Consequently, sensory signals from different modalities should interact
supralinearly for the control of gaze movements. 

\textit{Linear integration} \textemdash Alternatively, saliency could be
computed separately for different modalities and subsequently be combined
linearly before fixation selection. Recent research shows that the bulk of
neuronal operation underlying multisensory integration in superior
colliculus can be well described by a summation of unimodal channels
\citep{stanford2005a}. Within this scheme multimodal saliency would be the
result of the linear combination of unimodal saliencies. From an
information-theoretic perspective, this type of linear summation is
optimal, as is the resulting multimodal map, in the sense that the final
information gain is equal to the sums of the unimodal information gains. 

\textit{Late combination} \textemdash No true integration between
modalities occurs. Instead, overt behavior results from competition between
candidate fixation points in the independent unimodal saliency maps. The
implementation of such a max operator results in a sub-linear integration
of the unimodal saliency maps. Although improvements in saccade latencies
and fixation accuracies in (non-natural) multimodal stimulus conditions
have been used to support the counter-claim that crossmodal integration
does take place \citep{corneil2002a, arndt2003a}, this hypothesis still
warrants investigation under natural free-viewing conditions.

We presented human subjects with lateralized natural sounds and natural
images in a variety of conditions and tracked their eye movements as a
measure of overt attentional allocation. These measurements were used to
compute empirically determined saliency maps, thus allowing us to
investigate the above hypotheses. 

\section{Materials \& Methods} 

\label{part_cm_mm} 

\subsection{Participants and recording}

Forty-two subjects (19 males, mean age 23) participated in the experiment.
All subjects gave informed written consent but were naive to the purpose of
the experiment. All experimental procedures were in compliance with
guidelines described in Declaration of Helsinki. Each subject performed
only one session.

The experiments were conducted in a small dimly lit room. The subjects sat
in a stable chair with back support, facing a monitor. A video-based
head-mounted eye tracker (Eye Link 2, SR Research, Ontario, Canada) with
sampling rate of 250 Hz and nominal spatial resolution of $0.01^\circ$ was
used for recording eye movements.

For calibration purposes, subjects were asked to fixate points appearing in
a random sequence on a $3 \times 3$ grid by using built-in programs
provided with Eye Link (similar to \cite{tatler2006a}). This procedure was
repeated several times to obtain optimal accuracy of calibration. It lasted
for several minutes, thus allowing subjects to adapt to the conditions in
the experimental room. All the data analyzed in the present article were
obtained from recordings with an average absolute global error of less then
$0.3^\circ$.

During the experiment, a fixation point appeared in the center of the
screen before each stimulus presentation. The experimenter triggered the
stimulus display only after the subject had fixated this point. The data
obtained during this control fixation were used to correct for slow drifts
of the eye tracker; that is, if drift errors were high a new calibration
protocol was started again. Subjects could take a break and remove the
headset at any time. In those instances, which occurred rarely, the
continuation of the experiment began with the calibration procedure
described above.

\subsection{Stimuli}

The multimodal stimuli consisted of images and sounds. Images depicted
natural scenes like forests, bushes, branches, hills, open landscapes,
close ups of grasses and stones, but also to a limited extent human
artifacts, for example, roads, house parts. Some of these stimuli were used
in a previous study \citep{einhaeuser2003a}. The photographs were taken
using a 3.3 megapixel digital camera (Nikon Coolpix 995, Tokyo, Japan),
down-sampled to a resolution of 1024 \texttimes 768 pixels, and converted
to grayscale. The images were displayed on a 21-in. CRT monitor (SyncMaster
1100DF, Samsung Electronics, Suwon, South Korea), at a resolution of 1024
\texttimes 768 pixels and with a refresh rate of 120 Hz. The distance of
the monitor from the subject's eyes was 80 cm. The stimuli covered
$28^\circ \times 21^\circ$ of visual angle on the horizontal and vertical
axes, respectively. The screen was calibrated for optimal contrast and
brightness using a commercial colormeter (EyeOne Display, GretagMacBeth,
Regensburg, Switzerland). The natural auditory stimuli were taken from free
samples of a commercial Internet Source (Askland Technologies Inc.,
Victorville, Canada). Overall, 32 different sound tracks were used. All of
these were songs of different birds, thus in accordance with the semantic
content of the images presented. The auditory stimuli were generated using
a sound card (Sound Blaster Audigy EX ZE Platinium Pro, Creative Labs,
Singapore) and loudspeakers (Logitech 2.1 Z-3, CA, USA). Loudspeakers
flanked both sides of the monitor at a distance of 20 cm, at the same depth
plane as the screen. In order to avoid the speakers attracting the
attention of the subjects, they were hidden behind black curtains. Sounds
were played from the left or right speaker, depending on the experimental
condition. The auditory signal amplitude was in the range of 50-70 and
55-75 dB for left and right conditions, respectively. The slight increase
was due to the acoustic conditions in the experimental room.

\subsection{Experimental paradigm}

We employed two unimodal conditions (visual and auditory) and one
multimodal condition (audiovisual) (\ref{fig1}A). During the visual
condition (V), 32 natural images were presented. The auditory condition
used $16 \times 2$ presentations of natural sounds, originating from left
(AL) and right (AR) side relative to the subject's visual field.

\begin{figure}[!htbp] \centerline{
\includegraphics[width=\textwidth]{./part_cm/figures/figure01.png}} 

\caption[Experimental Paradigm and Subject Selection.]{
\textbf{Experimental Paradigm and Subject Selection.} \textbf{(A)} Five
different experimental conditions are used to analyze the effect of
unimodal (visual and auditory) and crossmodal (audiovisual) stimuli on
overt behavior. 42 subjects studied 32 natural images in three different
conditions. In the visual condition (V), natural images were presented. In
multimodal conditions, the images were paired with localized natural sounds
originating from either the left (AVL) or right (AVR) side of the monitor.
In the remaining conditions (AL and AR), the sole effect of the localized
auditory stimulus was characterized. The duration of stimulus presentation
in all conditions was 6 seconds. \textbf{(B)} Distribution of values of
each subject's $\sigma$, which characterizes the horizontal spread of the
pdf $p_{s,v}(x,y)$ of a given subject's at condition V. $p_{s,v}(x,y)$ from
two subjects are shown in the inset with corresponding horizontal marginal
distributions shown above. Arrows mark the histogram bins to which these
subjects belong. The vertical dashed line indicates the threshold $\sigma$
value required for a given subject to be included in further analysis.}
\label{fig1}
\end{figure}


During auditory conditions, we presented auditory stimuli jointly with
white noise images. These were constructed by shuffling the pixel
coordinates of the original images. They lack any spatial structure and as
a result do not bias the fixation behavior of the subjects. Obtaining a
truly unimodal saliency map for auditory conditions adds some undesirable
technical issues. Firstly, due to the operation of the eye tracker and
monitor truly zero-light conditions are hard to achieve. Secondly,
presenting a dark stimulus leads to more fixations outside the dynamics
range of the monitor and eye tracker. Finally, a sudden drastic change in
mean luminance of the visual stimulus would introduce non-stationarities in
the form of dark adaptation and create a potential confound. To avoid these
problems, we presented white noise images of identical mean luminance as
the natural pictures.

The multimodal conditions (AVL and AVR) each comprised the simultaneous
presentation of 32 auditory and visual stimuli pairs, without any onset or
offset asynchrony. In order to balance the stimulus set, a new pairing of
audiovisual stimuli was presented on each side to each subject. Stimuli
were shown in a pseudorandom order, with a different permutation used for
each subject. Each stimulus was presented for 6 s. A given session
contained 128 stimulus presentations and lasted in total for up to 50 min.
The only instruction given to the subjects was to watch and listen
carefully to the images and sounds. No information about the presence of
the speakers at both sides of the monitor or the lateralization of the
auditory stimuli was provided to the subjects.

\subsection{Data analysis}

We defined fixation points and intervening saccades using a set of
heuristics. A saccade was characterized by an acceleration exceeding 8000
deg/s\textsuperscript{2}, a velocity above 30 deg/s, a motion threshold of
$0.1^\circ$, and a duration of more than 4 ms. The intervening episodes
were defined as fixation events. The result of applying these parameters
was plotted and was visually assessed to check that they produce reasonable
results.

\subsection{Probability distributions}

From the fixation points of the individual subjects, we built probability
density functions (pdf). The first fixation on each stimulus was discarded,
as it was always located at the position of the preceding fixation cross in
the center of the image. A pdf, $p_{s,i,c}(x, y)$, for a given subject
\textit{s}, image \textit{i}, and condition \textit{c} was calculated as in
equation \ref{cm_eq1}.


\begin{equation}
p_{s,i,c}(x,y,t)=\frac{1}{F}\sum_{f=1}^{F}\delta(x-x_f)\delta(y-y_f)\delta(t-t_f)
\label{cm_eq1} \end{equation} 




with $\delta(x)$ as the discrete Dirac function (the Dirac function is
equal to zero unless its argument is zero, and it has a unit integral).
$x_f$, $y_f$, and $t_f$ are the coordinates and time of the
f\textsuperscript{th} fixation. $F$ is the total number of fixations. We
distinguish three different pdfs for a given condition with respect to how
these individual pdfs were averaged: subject, image, and spatio-temporal
pdfs. Subject pdfs $p_{s,c}(x,y)$ for a given subject $s$ and condition $c$
were built by averaging all the pdfs obtained from a given subject over the
images, without mixing the conditions, according to \ref{cm_eq2}

\begin{equation} p_{s,c}(x, y)=\frac{1}{I}\sum_{i}^{I}p_{s,i,c}(x,y)
\label{cm_eq2} \end{equation} 


Image pdfs ($p(x, y)$) and spatio-temporal pdfs ($p(x, t)$) were similarly
computed by averaging over the appropriate dimensions. Image pdfs inform us
about consistent biases that influence subjects' gaze and are therefore
empirically determined saliency maps specific to a given image. Raw pdfs
are matrices of the same size as an image and store fixation counts in
their entries. In all cases, these raw pdfs were smoothed for further
analysis by convolution. A circular two-dimensional Gaussian kernel was
used, which had a unit integral and a width parameter $\sigma$ with a value
of $0.6^\circ$ unless otherwise stated. This spatial scale is twice as
large as the maximal calibration error and maintains sufficient spatial
structure for data analysis.

\subsection{PDF parameterization}


Several parameters were extracted from the pdfs,


\begin{equation} \mu_c = \frac{\sum_x{p_c(x)x }}{\sum_x{p_c(x)}} \label{cm_eq4} \end{equation} 


The center of gravity is measured according to Equation \ref{cm_eq4} in
order to quantify the global shift of subject and image pdfs. $\mu_c$ is
the center of gravity along the X-axis for condition $c$, $p_c(x)$ is the
marginal distribution of a pdf along the X-axis. In condition V, fixation
probabilities were usually distributed symmetrically on both sides of the
visual field with centralized center of gravity values. This simple
statistic successfully quantifies any bias toward the sound location,

\begin{equation} \sigma = [ \sum_x{(x-\mu_V)^2 p_V(x)}  ]^{1/2}  \label{cm_eq5} \end{equation} 

The spread of a given pdf was measured from a subject pdf under condition V
using Equation  \ref{cm_eq5} in order to quantify how explorative that
subject's scanning behavior was. $\sigma$ is the spread along the X-axis,
$p_V(x)$ is the marginal distribution along the X-axis of the subject pdf,
and $\mu_V$ is the center of gravity computed as in Equation \ref{cm_eq4}.
The marginal distributions arising from condition V were well-behaved
(\ref{fig1} inset), thus allowing us to examine the explorative behavior of
subjects. Seven of 42 subjects did not engage in explorative viewing
during the analysis of the images, resulting in small spread values. These
subjects were excluded from further analysis.


The spatio-temporal pdfs, $p_c(x, t)$, contain information about how the
fixation density along the horizontal axis varies over time. We assessed
the statistical differences of pdfs from different conditions in a
time-localized manner; that is, we obtained a p-value as a function of time
for three pairs of conditions (AVL and V; AVR and V; AL and AR). A
two-sided Kolmogorov-Smirnov goodness-of-fit hypothesis test in
corresponding temporal portions of the pdfs was used with significance
level $\alpha$ set to .001. This was done after binning the probability
distribution $p_c(x, t)$ over the time axis with a bin size of 240 ms,
yielding 25 time intervals for comparison per pair of conditions. A
temporal interval over which the null hypothesis was rejected in at least
in two of the condition pairs was considered as a temporal interval of
interest.

The similarity between image pdfs obtained from the same image under
different conditions \textemdash for example, $p_{i,V}(x, y)$ of image $i$
and condition V and $p_{i,AV}(x, y)$ of the same image and AV conditions
(i.e., either AVL or AVR) \textemdash was evaluated by computing
$r_{V−AV}^2$. Before the coefficients were calculated, the AV image pdfs
were first normalized to $p_{i,AV}^N$ according to Equation \ref{cm_eq6}

\begin{equation} p_{i,AV}^{N} =
\frac{<p_V(x)>_i}{<p_{AV}(x)>_i}p_{i,AV}(x,y) \label{cm_eq6} \end{equation} 


This normalization corrects the image pdfs of a given bimodal condition for
the global effect of the sound location, so that the expected number of
fixations over the horizontal axis is the same over different conditions.
The resulting distribution of $r_{V,AV}^2$ values was then compared to a
control distribution of $r^2$ values, which measure the baseline
correlation between image pdfs coming from the same condition pairs (i.e.,
V and AVL, or V and AVR) but differing images. The Kullback-Leibler (KL)
divergence, which does not assume any a priori relationship between two
distributions, was used to quantify the similarity of the different image
pdfs obtained from different conditions. This measure was evaluated
according to the following formula \ref{cm_eq7}, where $D_{KL}(p_{i,c_1},
p_{i,c_2})$ denotes the KL divergence measure between two pdfs,
$p_{i,c_1}(x, y), p_{i,c_2} (x,y)$ in bits, 

\begin{equation} D_{KL}[p_{i,c_1}(x,y),p_{i,c_2}(x,y)] =
H[p_{i,c_1}(x,y),p_{i,c_2}(x,y)] -H[p_{i,c1}(x,y)] \label{cm_eq7}
\end{equation} 

\nomenclature{KL}{Kullback-Leibler}

The KL divergence measures the difference between the
cross-entropy of two probability distributions and the entropy of one of
them. The cross-entropy is always greater than or equal to the entropy;
therefore, the KL divergence is always greater than or equal
to zero, which allows its usage as a distance measure between two different
pdfs. However, unlike other distance measurements, it is not symmetric.
Therefore, it is used to measure the distance between the prior and
posterior distributions. In comparing different image pdfs, we used the
condition V as the prior distribution. One problem we encountered was zero
entries in the probability distributions. As the logarithm of zero is not
defined, a small constant ($c = 10^{-9}$) was added to all entries in the
pdf. The precise choice of this constant did not make a difference to the
results of our analysis.

\subsection{Modeling the cross-modal interaction}


In order to quantify the cross-modal interaction, we carried out a multiple
regression analysis. We devised a model (Equation \ref{cm_eq8}) with a
cross-product interaction term using smoothed unimodal and multimodal pdfs
as independent and dependent variables, respectively, 



\begin{equation} \label{cm_eq8}
p_{i,AV}=\beta_{i,0}+\beta_{i,1}\cdotp_{i,V}+\beta_{i,2} \cdot p_{i,A} +
\beta_{i,3} \cdot p_{i,V-A}  \end{equation} 

In Equation \ref{cm_eq8}, $p_{i,AV}$, $p_{i,V}$, and $p_{i,A}$ are the
image pdfs of image $i$ at audiovisual, visual, and auditory conditions,
respectively. The interaction term $p_{i,VA}$ is supposed to approximate
the image pdf that would arise from a multiplicative cross-modal
interaction. It is created by the element-wise multiplication of both
unimodal image pdfs and renormalized to a unit integral.


The integrative process was further characterized by constructing
integration plots. The probability of fixation at each x-y location was
extracted from the 32 image pdfs of visual, auditory, and bimodal
conditions, yielding a triplet of values representing the saliency. A given
triplet defines the saliency of a given image location in the multimodal
condition as a function of the saliency of the same location in both
unimodal conditions, represented by the point ($p_V(x, y), p_A(x, y),
p_{AV}(x, y)$) in the integration plot. These points were irregularly
distributed and filled the three-dimensional space unevenly. We discarded
the 15\% of the values which lay in sparsely distributed regions, and we
concentrated instead on the region where most of the data were located. The
data points inside this region of interest were then binned, yielding an
expected value and variance for each bin. Weighted least square analysis
was carried out to approximate the distribution by estimating the
coefficients of the following equation:

\begin{equation} \frac{p_{AV}}{g(p_{AV})} = \beta_0 + \beta_1 \cdot
\frac{p_V}{g(p_V)} + \beta_2 \cdot \frac{p_A}{g(p_A)} + \beta_3 \cdot
\frac{p_{V\cdot A}} {g(p_{V\cdot A})} \label{cm_eq9} \end{equation} 


The difference between the above equation and Equation \ref{cm_eq8} is that
Equation \ref{cm_eq9} does not take different images into consideration and
pools the data over images and space. Additionally, each individual
probability value is normalized by its geometric mean ($g(pc)$), which
normalizes for its individual range thus allowing a direct comparison of
the regression coefficients.


\subsection{Luminance contrast measurements}


LC was computed as the standard deviation of the luminance values of the
pixels inside a square patch of about $1^\circ$ centered at fixation
positions. The luminance contrast computed at the fixation points over
images and subjects yielded the actual contrast distribution. This
distribution was compared to a control distribution to evaluate a potential
bias at the fixation points. An unbiased pool of fixation coordinates
served as the control distribution \textemdash for a given image, this was
constructed by taking all fixations from all images other than the image
under consideration. This control distribution takes the center bias of the
subjects' fixations into account, as well as any potential systematic
effect in our stimulus database \citep{baddeley2006a, tatler2005a}. The
contrast effect at fixation points was computed by taking the ratio of the
average contrast values at control and actual fixations. In order to
evaluate the luminance contrast effect over time, the actual and control
fixation points were separated according to their occurrences in time. The
analysis was carried out using different temporal bin sizes ranging from
100 to 1000 ms. All analysis was carried out using MatLab (Mathworks,
Natick, MA, USA).




\section{Results}

First, we analyze the effect of the lateralized auditory stimuli on the
fixation behavior of subjects during the study of natural images
Fig. \ref{fig1}A. Second, we characterize the temporal interval during which the
influence of auditory stimuli is strongest. Third, we demonstrate the
specific interaction of visual and auditory information. And finally, we
address the predictions derived from the three hypotheses of crossmodal
interaction stated above. 


\subsection{Subjects' Gaze Is Biased towards the Sound Location.} 


In Fig. \ref{fig2}, the spatial distribution of fixation points
averaged over images in all 5 conditions is shown for 2 subjects. In the
visual condition (V), the fixation density covers a large area and is
equally distributed over the left and right half of the screen, with
neither of the subjects showing a consistent horizontal bias. The center of
gravity ($\mu_{V}$) is located at 505 and 541 pixels respectively (white
crosses) for these two subjects, in the close vicinity of the center of the
screen (located at 512 pixels). In the multimodal conditions (AVL and AVR),
both subjects show a change in their fixation behavior. The horizontal
distance between $\mu_{AVL}$ and $\mu_{AVR}$ is 221 and 90 pixels for the
two subjects respectively. Thus, in these two subjects, combined visual and
auditory stimulation introduces a robust bias of fixation towards the side
of sound presentation. 

Auditory unimodal conditions (AL and AR) induce different patterns of
fixation (Fig. \ref{fig2} right hand columns, note different scales of color
bars). First, despite lateralized sound presentation, a non-negligible
proportion of fixations are located close to the center. Nevertheless the
lateralized sound stimulus induces a shift toward the sound location, even
in the absence of any structured, meaningful visual stimulation. In most
cases, the shift had an upwards component. Furthermore, the off-center
fixations are less homogeneously distributed and their distribution does not
qualitatively resemble the distribution of fixations in the visual
condition. 

\begin{figure}[!htbp] 
\centerline{ \includegraphics[width=\textwidth]{./part_cm/figures/figure02.png}} 

\caption[Bias of Subjects' Gaze towards the Sound Location.] {\textbf{Bias
of Subjects' Gaze towards the Sound Location.} \textbf{(A, B)}
$p_{s,c}(x,y)$ for two subjects in each condition. Each colorbar shows the
scale of the pdf images located to its left; notice the differences in
scale. White crosses denote the center of gravity of a given pdf along the
horizontal axis. These pdfs were generated by convolving the original pdfs
with a Gaussian kernel ($\sigma = 0.6^{\circ}$).} \label{fig2} \end{figure}

The complete statistics of all subjects and images are shown in Fig. \ref{fig3}.
The distribution of center of gravity shifts ($\mu_{AVL}-\mu_{V}$ and
$\mu_{AVR}-\mu_{V}$) for all subjects over all images is skewed
(Fig. \ref{fig3}). In the majority of subjects, we observe a moderate effect of
lateralized sound presentation. A small group of subjects show only a small
influence; one subject an extreme effect. The medians of the two
distributions are both significantly different from zero (sign test, $p
\textgreater 10^{-5}$). Crosses indicate the positions of the two example
subjects described above. They represent the $70^{th}$ and $90^{th}$
percentiles of the distributions. Hence, the complete statistics support
the observations reported for the two subjects above. 

An analysis of the influence of auditory stimuli on the selection of
fixation points in individual images (over all subjects) is shown in
Fig. \ref{fig3}B. We see that for all visual stimuli the shift in average
horizontal position of fixation points is towards the sound location
(t-test, $p \textgreater 10^{-5}$). In both of the panels A and B, the
distributions flank both sides of zero, with mean values of -37 and 36
pixels for $\mu_{AVL}-\mu_{V}$ and $\mu_{AVR}-\mu_{V}$ respectively. Thus,
auditory stimulation introduces a robust bias of fixation towards the side
of sound presentation for all natural visual stimuli investigated. 

\begin{figure}[!htbp] \centerline{ \includegraphics[width=0.5\textwidth]{./part_cm/figures/figure03.png}}
\caption[Distribution of Fixation Density Shifts towards Sound Location.]{
\textbf{Distribution of Fixation Density Shifts towards Sound Location.}
\textbf{(A)} Centers of gravity ($\mu_{c}$ for condition \textit{c}) were
calculated for the fixation pdfs of each subject averaged over all images
for multimodal conditions (AVR and AVL) and unimodal visual condition (V).
The distributions of distances between multimodal and unimodal centers of
gravity are shown. The averages are marked with arrowheads and equal -44
pixels for $\mu_{AVL}-\mu_{V}$ (\textit{dark gray}) and 42 pixels for
$\mu_{AVR}-\mu_{V}$ (\textit{light gray}). The two distributions are
statistically different (sign test, $p \textless 10^{-5}$). The plus signs
mark the effect size for the subjects depicted in Fig. \ref{fig2}.
\textbf{(B)} Shows the same measurement, this time calculated for fixation
pdfs of each image averaged over all subjects. The two distributions are
statistically different (t-test, $p \textless 10^{-5}$). The average values
of the distributions are same as in (A). However, the variance of the
shifts of gravity centers is bigger on subject pdfs compared to the image
pdfs therefore resulting in different scales on the abscissa.}
\label{fig3}\end{figure}

\subsection{Effect is More Prominent in the First Half of Presentation}

Next we analyze the temporal evolution of the above-described horizontal
effect of the auditory stimulus. Fig. \ref{fig4} depicts the difference between
the spatio-temporal fixation probability density functions (pdf) for AVR
and V, AVL and V, and AL and AR conditions. Comparing visual and multimodal
conditions shortly after stimulus onset, (Fig. \ref{fig4}A, B lower plots) we
observe a rapid shift in fixation density on the side of the auditory
stimulus, which is sustained for the presentation period. This increase in
fixation density is realized at the expense of fixations in the central
region, and less so at the expense of fixations on the contra-lateral side.
This can be seen by comparing the marginal distributions (averaged over
time) originating from the pdfs used to calculate \newpage the difference
(Fig. \ref{fig4}A, B upper plots). The intervals of time for which the
differences reach significance level (two-sided KS-test, p=0.001) are
indicated by vertical black bars. Comparing conditions AL and AR
(Fig. \ref{fig4} right panel), we observe an increase in fixation probability on
the half of the horizontal axis corresponding to the side of the sound
location. This difference decays only slowly over time. 

\begin{figure}[!htbp] \centerline{ \includegraphics[width=\textwidth]{./part_cm/figures/figure04.png}}
\caption[Effect of Localized Auditory Stimulus is More Prominent in the
First Half of Presentation.]{ \textbf{Effect of Localized Auditory Stimulus
is More Prominent in the First Half of Presentation.} The uppermost plots
show two marginal probability distributions for the following pairs of
conditions: \textbf{(A)} AVR and V, (\textbf{B)} AVL and V, (\textbf{C)} AR
and AL, \textit{dashed} and \textit{solid} lines respectively. The lower
plots depict the difference between the spatio-temporal pdfs of the same
pairs of conditions. Contour lines are drawn at zero values. Along the
times axis, the \textit{black horizontal} lines mark the regions where the
difference between the two pdfs is significant (two-sided KS-test,
$\alpha=0.001$). The \textit{vertical dashed} line limits the temporal
interval of interest, which is used for further analysis. The pdfs were
generated using a Gaussian kernel ($\sigma = 0.4^{\circ}$ and 70 ms). }
\label{fig4}\end{figure}



\subsection{Specific Interaction of Visual and Auditory Information}

As a next step we investigate the integration of auditory and visual
information in the temporal interval of interest. The first column of
Fig. \ref{fig5} depicts examples of the natural \newpage images used in the experiment.
The other columns show the image pdfs computed over all subjects in
conditions V, AVL and AVR. As these pdfs were computed over many subjects'
fixations, they constitute empirically determined saliency maps. For each
image, we observe a characteristic distribution of salient regions, i.e.
regions with high fixation probabilities. It is important to note that the
fixation densities are highly unevenly distributed, suggesting a similarity
between subjects' behaviors. We computed the correlation coefficients
between image pdfs generated from two subsets of 5 randomly selected
subjects (repeating the same analysis 300 times). For all three conditions
the distribution of coefficients over all images and repetitions peaked at
around 0.6. This suggests that different subjects had similar behaviors for
scrutinizing the image during the temporal interval of interest (240 - 2640
ms). It is not clear whether this was the result of the specific image
content or shared search strategies between subjects.

\begin{figure}[!htb] \centerline{ \includegraphics[width=\textwidth]{./part_cm/figures/figure05.png}} 

\caption[Specific Interaction of Visual and Auditory
Information.]{\textbf{Specific Interaction of Visual and Auditory
Information.} Fixation pdfs,$p_{i,c}(x,y)$, for a given image \textit{i}
and conditions \textit{c} (V, AVL and AVR) are shown, along with the
corresponding natural image \textit{i}. Each pdf constitutes an empirically
determined saliency map. The saliency maps for each image are shown along
rows, for each condition along columns. \textit{White crosses} in each
panel depict the centers of gravity of the pdfs. In multimodal conditions,
the center of gravity shifts toward the side of auditory stimulation.
Interestingly however, moving across each row we see that the salient spots
for each image are conserved across conditions as shown by the high
$r\textsuperscript{2}$ and low KL divergence values (right of colorbar).
Fixation pdfs are computed inside the temporal interval of interest and
convolved with a Gaussian kernel ($\sigma = 0.6^{\circ}$). }

\label{fig5}
\end{figure}

The two right-hand columns show the respective distributions obtained in
multimodal conditions. First, as noted earlier, the lateralized stimulus
causes the center of gravity to shift along the horizontal axis
(Fig. \ref{fig5}, white crosses). Importantly, the spatial distributions of the
spots are alike in different conditions, but differ across images. The
regions with high fixation probability in one multimodal condition
(Fig. \ref{fig5}) still effectively attract gaze when the side of auditory
stimulation is switched, as well as in the unimodal condition.

This observation is quantified by measuring the KL divergence and
$r\textsuperscript{2}$ statistic between saliency maps (such as those in
Fig. \ref{fig5}) belonging to unimodal and cross-modal conditions. For the
examples shown in Fig. \ref{fig5} we obtain a KL-divergence of 0.88, 1.32,
0.94, 1.02, 1.09 bits between V and AVL conditions and 1.38, 1.79, 0.87,
1.09, 0.79 bits between V and AVR. $r\textsuperscript{2}$ statistics range
from 0.4 to 0.97, indicating that a substantial part of the total variance
of the multimodal conditions is explained by the distribution of fixation
points in the unimodal visual condition. 

The distribution of KL-divergence values obtained from the complete dataset
(32 images times 2 auditory stimulus locations) is presented in \ref{fig6}.
The more similar the two pdfs are, the closer the KL-divergence values get
to zero; zero being the lower limit in the case of identity. This
distribution is centered at $1.08 \pm 0.03$ bits (mean $\pm$SEM). This is
significantly different to the mean of the control distribution ($3.45 \pm
0.02$ bits), which was created using 3200 randomly selected non-matched
V-AV pairs. The control distribution provides the upper limit for
KL-divergence values given our data set. Hence, given the distribution of
fixation points on an image in the visual condition, the amount of
information necessary to describe the distribution of fixation points on
this image in the multimodal conditions is about one third of the
information necessary to describe the difference in fixation points on
different images in these conditions. 

\begin{SCfigure}[][!htb]
\includegraphics[width=0.5\textwidth]{./part_cm/figures/figure06.png} 
\caption[Auditory and Visual Information are Integrated.]{ \textbf{Auditory
and Visual Information are Integrated.} The distributions of KL divergence
\textbf{(A)} and $r\textsuperscript{2}$ \textbf{(B)} values for control
(textit{dark gray}) and actual (\textit{light gray}) conditions are shown.
The actual distributions are obtained by comparing 64 pairs of multimodal
(AVR and AVL) and unimodal (V) fixation pdfs. Control distributions are
created by computing the same statistics on randomly-paired non-matched
multimodal and unimodal pdfs (n = 3200). The measurements are directly
obtained from pdfs shown in Fig. \ref{fig5}. \textbf{(C)} The logarithm of
the ratios of actual and control luminance contrast values are presented as
a function of time. Almost all values lie above the identity line. This
effect is stable over the time of presentation and for different
conditions. The \textit{gray shaded} area shows the temporal region of
interest.} \label{fig6}\end{SCfigure}

These results are supported by a more conventional linear measure. Fig.
\ref{fig6}B shows the distribution of actual $r\textsuperscript{2}$ values
calculated between image pdfs from multimodal and unimodal conditions
originating from the same image. The distribution is centered at
$0.71\pm0.13$ and the difference between this measure and a control
$r\textsuperscript{2}$ measure calculated from shuffled image pairs is
highly significant (t-test, $p \textgreater 10^{-5}$). This implies that
for the majority of images, the unimodal fixation pdfs account for more
than half of the variance in the observed distribution of fixation points
in multimodal conditions. Hence, the bias of gaze movements towards the
side of the auditory stimulus largely conserves the characteristics of the
visual saliency distribution. Therefore the behavior of the subjects under
the simultaneous presence of auditory and visual stimuli is an integration
of both modalities. 

As a complementary approach, we investigate the effect of multimodal
stimuli on the relationship between visual stimulus properties and the
selection of fixation points. Several previous studies investigating human
eye movements under natural conditions describe a systematic increase of
luminance contrast at fixation points \citep{reinagel1999a, tatler2005a}.
If the auditory stimuli cause an orientation behavior independent of the
visual stimuli, then we can expect the luminance contrast at fixation
points to be reduced. If a true integration occurs, we expect this
correlation between luminance contrast and probability of fixation to be
maintained under multimodal stimulus conditions. Fig. \ref{fig6}C shows the
ratio of luminance contrast at actual fixations and control locations for
the unimodal and both multimodal conditions. Nearly all values
($\log\frac{actual}{control}$) are greater than zero, indicating a positive
correlation between fixation points and luminance contrast. Moreover, the
effect of contrast is constant over the entire presentation, with no
systematic difference during the temporal interval of interest
(Fig. \ref{fig6}C, gray area). Furthermore, the three conditions do not differ
significantly in the size of effect. This holds for temporal bin sizes
ranging from 100 to 1000 ms (data not shown). Therefore, we can conclude
that the additional presence of a lateralized auditory stimulus does not
reduce the correlation between the subjects' fixation points and luminance
contrast. 

\subsection{The Integration Is Linear}

To quantitatively characterize the integration of unimodal saliencies, we
perform a multiple regression analysis. The experiments involved two
unimodal auditory conditions (AL and AR), and two corresponding multimodal
conditions (AVL and AVR). In order to simplify subsequent discussion, we
will use a more general notation of A and AV for unimodal auditory and
multimodal conditions respectively. We model the multimodal (AV)
distributions of fixation points by means of a linear combination of
unimodal (A and V) distributions and the normalized (to unit integral)
product of unimodal distributions (A \texttimes V) as indicated in equation
(\ref{cm_eq7}). Here we assume that the effect of a multiplicative
integration of unimodal saliencies is well approximated by a simple
multiplication of their probability distributions. The fits were computed
separately for left and right conditions, and the results were pooled for
visualization purposes. The distribution of 64 coefficients for each of the
regressors is shown in Fig. \ref{fig7}. On average, over all images, the
contribution of unimodal visual saliency is largest, with a mean at $0.75
\pm0.15$ (mean $\pm$SD; Fig. \ref{fig7}, circles). The contribution of
unimodal auditory saliency is smaller ($0.16 \pm0.12$). The coefficient of
the cross-product interaction term is, however, slightly negative with mean
$0.05 \pm0.10$. We repeated the same analysis for a subset of subjects
(n=14, 40 \%) for whom the lateralized auditory stimulus had the strongest
effect on fixations in terms of gravity center shift in the unimodal
auditory conditions. In these subjects, the contribution of auditory
coefficients was increased ($0.32 \pm0.17$) at the expense of the visual
ones ($0.53 \pm20$), without any apparent effect on the interaction term
($-0.06 \pm0.11$, t-test, $p = 0.59$) (Fig. \ref{fig7}, crosses). In both
cases, the intercept was very close but still significantly different from
zero. These results suggest that biggest contributions to the multimodal
pdfs originate from the linear combinations of the unimodal pdfs. 

\begin{SCfigure}[][!htb] 
\includegraphics[width=0.5\textwidth]{./part_cm/figures/figure07.png}
\caption[Prominent Contribution of Unimodal Saliencies.]{\textbf{Prominent
Contribution of Unimodal Saliencies.} This figure shows the distributions
of the parameters of equation Fig. \ref{cm_eq7} (inset), computed using multiple
regression analysis. Image pdfs used for the regression analysis were
computed either by using all subjects (n = 35) or by selecting a subset of
subjects (n = 14) for whom the lateralized auditory stimulus had the
strongest effect on fixations, quantified in terms of gravity center shift.
The best fits between image pdfs from conditions AV and V, A and A
\texttimes V were calculated for each of the 32 images, in left and right
conditions, yielding 64 fits for each parameter. The distributions of the
coefficients resulting from the regression analysis using all subjects are
shown. \textit{Circles} denote the mean of each distribution of each
coefficient. All means are significantly different from zero (t-test, p
\textless $10^{-3}$). The average value of the visual coefficients
$\beta_{1} (0.75 \pm0.15 ;\pm$SD) is greater than the average of the
auditory coefficients $\beta_{2} (0.16 \pm0.12)$, and these unimodal
coefficient averages are both greater than that of the multimodal
coefficients $\beta_{3} (-0.05 \pm0.1)$. Repeating the same analysis using
the subset of auditorily-driven subjects results in higher average auditory
coefficients $(0.32 \pm0.17)$ and lower visual coefficients $(0.53
\pm0.20)$, while the interaction term does not change significantly $(-0.06
\pm0.11, t-test $p = 0.59$)$. \textit{Crosses} indicate the means of each
and every distribution for this subset of subjects. } \label{fig7}
\end{SCfigure}

In a subsequent analysis, we carried out the regression analysis using a
different combination of dependent variables and evaluated how the
introduction of an additional dependent variable increased the explained
variance. Using only unimodal visual pdfs as one regressor, we obtained
$r\textsuperscript{2}$ values having a median value of 0.72 over all images
\textemdash as expected from the previous section. Additionally including
unimodal auditory pdfs increased the median $r\textsuperscript{2}$ only
slightly, by 3\%. Repeating this analysis using only the subset of subjects
showing the strongest auditory lateralization effect, we obtained a median
value of 0.36 with the sole visual regressor. The subsequent introduction
of the unimodal auditory pdfs as second dependent variable increased the
goodness of fit by 21\% over all images. Further including the cross-
interaction term as the third dependent variable, the goodness of fit
increased slightly, by 5\%. Therefore we can argue that mechanism linearly
combining the unimodal saliencies can well account for the observed
behavior in the multimodal conditions.


As a model-free approach, we compute integration plots using saliencies
obtained from different conditions. It should be noted that no assumptions
are made regarding the calculation of the saliency; that is, these saliency
values are empirically determined by the gaze locations of many subjects.
Integration plots are constructed by plotting the saliency of a given
spatial location in the multimodal pdfs as a function of unimodal
saliencies of the same spatial location. The specific distribution within
this three dimensional space describes the integration process. In
Fig. \ref{fig8} A, B and C, the height of each surface depicts the corresponding
salience of the same location during the multimodal condition as a function
of the saliency of the same location in unimodal conditions. The three
hypotheses about the integration of auditory and visual information make
different predictions (see introduction): Early interaction leads to a
facilitatory effect and an expansive non-linearity (Fig. \ref{fig8}). The
landscape predicted in the case of linear integration is planar and shown
in Fig. \ref{fig8}. Late combination gives rise to a compressive non-linearity
(Fig. \ref{fig8}C). Applying this approach to our complete dataset leads to a
highly non-uniform distribution of observed unimodal saliencies, when
considered in terms of their frequency of occurrence. In Fig. \ref{fig8}D, the
count matrix of joint occurrences of unimodal saliencies is presented; the
surface is very sparsely filled at regions with high values of salience and
practically no point is present at regions where both unimodal saliencies
are high (top right region of Fig. \ref{fig8}D). Statistically reliable
statements about the data within these regions are thus not possible.
Within this space, we defined a region of interest depicted as a rectangle
in Fig. \ref{fig8}D; this portion of the space contains 85\% of the total number
of samples. Inside this region, we bin the data using a 10 \texttimes 10
grid and calculate the expected value (Fig. \ref{fig8}) and an error estimate
(variance) for each bin. 

\begin{SCfigure}[][!htb]
\includegraphics[width=0.65\textwidth]{./part_cm/figures/figure08.png} 

\caption[The Integration is Linear.]{ \textbf{The Integration is Linear.}
\textbf{(A, B, C)} Three hypothetical frameworks for integration are
presented schematically as integration plots. The X and Y axes represent
the unimodal saliencies associated with a given location on the visual
field. The saliency of the same location in the multimodal condition is
color-coded, i.e. each pixel represents the saliency of a given point in
multimodal condition (p(AV)) as a function of the saliency of the same
point in unimodal conditions (p(A), p(V)). The specific distribution of the
points generating this landscape unravels the integration process. Please
note that the inherent topology of underlying images is no longer contained
in integration plots. The three integration schemes mentioned in the text
(see Introduction) predict different probability landscapes. If the
unimodal probabilities were interacting this would generate a landscape
with an expansive non-linearity \textbf{(A)}; however if the multimodal
saliencies were combined linearly, the resulting landscape is expected to
be planar \textbf{(B)}. The absence of an integration in a scenario where
the maximum of the unimodal saliencies determines the multimodal saliency
results in a compressive non-linearity \textbf{(C)}. \textbf{(D)} Joint
count matrix obtained by using all subjects. All pairs of unimodal
saliencies are plotted against each other. Grayscale level logarithmically
codes for the number of occurrences inside each bin. The marked rectangular
region contains 85\% of all points. \textbf{(E)} Integration plot
calculated for points lying in the rectangular region of (D) using a 10
\texttimes 10 binning. The color codes for the saliency in the multimodal
condition as in (A-C). Image pdfs used to compute this plot are obtained by
using all subjects. \textbf{(F)} Same as in (E), however only subjects with
the strongest auditory response are used. \textbf{(G)} Integration plot,
calculated using difference maps (see Results for details).}
\label{fig8}\end{SCfigure}

The relationship between these unimodal and multimodal saliencies is
further analyzed using a weighted regression analysis with unimodal
saliencies as dependent variables. This yielded $0.54 \pm0.04$ ($\pm$ 95
\%CI) and $0.59 \pm0.09$ for the linear contribution of visual and auditory
saliency respectively. Both coefficients were highly significant (t-test, p
\textless $10^{-6}$)except for the intercept coefficient (t-test, $p =
0.23$). $r\textsuperscript{2}$ is equal to 0.89 suggesting a good fit. We
repeated the same analysis with the interaction term included after
normalizing each regressor with its geometric mean in order to have the
same exponent range, thus permitting an evaluation of the contribution of
different regressors to the multimodal saliency. This yielded $0.57
\pm0.08$, $0.29 \pm0.08$ and $0.029 \pm0.05$ for visual, auditory and
interaction terms respectively. The linear contributions of unimodal
saliencies were highly significant whereas the intercept and the
interaction terms were not statistically different to zero.

Using such large bins increases the statistical power within each bin at
the expense of detailed structure. We therefore conducted the same analysis
using up to 50 bins covering the same region of interest. The
$r\textsuperscript{2}$ of the fitted data at this resolution was 0.87,
ensuring that the fit was still reasonably good. The values of coefficients
were practically the same, and the only noticeable change during the
incremental increase of the resolution was that the interaction term
reached the significance level ($p \textgreater 0.05$) at the resolution of
20 \texttimes 20, thus demonstrating a slight facilitatory effect. These
results support the conclusion that linear integration is the dominating
factor in crossmodal integration during overt attention, with an additional
small facilitatory component. 

The above analysis is influenced by a particular property of the auditory
saliency maps. Many fixation densities in the AL and AR conditions are
located at the center of the screen (see Fig. \ref{fig4}C). We tried to avoid
this problem in two different ways. In the first method, we performed the
same analysis on the subset of subjects mentioned earlier who were most
influenced by the lateralized auditory stimulus, thus minimizing the
central bias. Restricting the analysis allowed us to define a new region of
interest, which included 90\% of the total data points (Fig. \ref{fig8}E) and
discarded only those points that were very sparsely distributed in high
saliency regions. $r\textsuperscript{2}$ values varied within the range of
0.81 and 0.9, decreasing with higher binning resolutions. As above,
increasing the number of bins revealed a slight but significant
facilitatory effect. Within this subset of subjects, the contribution of
auditory saliency ($0.36 \pm0.08$) was again shown to increase at the
expense of the visual contribution ($0.50 \pm0.08$). Removing the
interaction term from the regression analysis caused a maximum drop of only
2.5\% in the goodness of fit for all tested bin resolutions within this
subset of subjects. 

In the second method used to remove the central bias artifact of unimodal
auditory pdfs, we took the differences between the left and right auditory
conditions, i.e. subtracted the two empirically determined auditory
saliency maps to yield difference maps. In each case, the saliency map for
the condition where the sound was presented contra-laterally was subtracted
from the saliency map of the congruent side (i.e. AL \textendash AR for
auditory stimulus presented on the left, AR \textendash AL for auditory
stimulation from the right). These newly generated maps are well-behaved
and allow the analysis of a larger region of saliency space (90\% of total
samples). The above analysis was repeated using the difference maps, and is
shown in Fig. \ref{fig8}G. It should be noted that the positive values on the
y-axis are the data points originating from the region of the screen from
which the sound emanates, during the temporal interval of interest. We
performed separate regression analyses for these two halves of the
resulting interaction map. In the lower part ($p(A) \textgreater 0$), the
best predictor was the visual saliencies, as can be seen from the contour
lines. In the upper part ( $p(A) \textless 0$), a linear additive model
incorporating auditory and visual saliencies well approximates the surface.
The results derived in an analysis of the effects of different bin sizes
were comparable to the above results; that is, a model combining linearly
unimodal saliencies along with a slight facilitatory component was
sufficient to explain a major extent of the observed data. 

We repeated the last analysis with image pdfs obtained with varying degrees
of smoothing. Decreasing the width of the convolution kernel systematically
reduced the explained variance of the fits on integration plots built
within probability ranges containing comparable amount of points. In a
large interval of tested parameters (0.4$^{\circ}$ \textendash
0.8$^{\circ}$), the main result was conserved i.e. the saliency surface
was, to a large extent, captured by a linear combination of unimodal
saliencies, with a slight multiplicative effect also evident. 

\section{Discussion}

In this study, we investigated the nature of the multimodal integration
during overt attention under natural conditions. We first showed that
humans do orient their overt attention towards the part of the scene where
the sound originates. This effect lasted for the entire period of the
presentation of the stimuli, but had a stronger bias during the first half
of presentation. More interestingly, this shift was far from a simple
orientation behavior \textemdash overt behavior during multimodal stimuli
was found to be dependent on the saliency of both visual and auditory
unimodal stimuli. Although subjects' fixation points were biased towards
the localized auditory stimuli, this bias was found to be dependent on
visual information. Our analysis suggests that a predominantly linear
combination of unimodal saliencies accounts for the crossmodal integration
process. 

We quantified the saliency associated with a given image region by analysis
of the measured overt eye movements of a large amount of subjects.
Subjects' behavior was similar within the temporal interval where the
effect of the lateralized sound was strongest. However we do not know
whether this was the result of a search strategy shared between subjects or
whether it originates purely from the bottom-up content present in the
image. The results presented here do not depend on the precise determinants
of the saliency associated to different image regions. Similarly, we
evaluated the saliency of different parts of the visual field associated
with the lateralized auditory stimulation. In many subjects this resulted
in a shift of fixation toward the side of the sound, in accord with
previous studies showing that sound source location is an important
parameter. 

Prior studies \citep{corneil1996a, corneil2002a, arndt2003a} have shown
that congruent multimodal stimulation during tasks where subjects were
required to move their gaze to targets as fast as possible results in
faster saccadic reaction times together with an increase in the accuracy of
saccades. Here we are extending these results to more operationally
relevant conditions by using natural stimuli under free-viewing conditions
where the subjects are not constrained in their behavior. Moreover, we are
formally describing the crossmodal behavior in terms of unimodal behavior. 

Concerning the temporal dynamics of the integration process, we found that
the localized sound stimuli attracts subjects' attention more strongly
during the first half of presentation, corresponding to an interval of
approximately 2.5 seconds. Although it is observed that the lateralization
of fixation density continues throughout the whole presentation time (Fig.
\ref{fig4}), the effects are much weaker and do not reach the significance
level. This effect can be understood as a consequence of inhibition of
return, the subject losing interest in that side of the image, or
alternatively due to an increasing efficiency of top-down signals over
time, resulting in a superior efficiency of the sensory signals to attract
attention during early periods of exposure only. 

One interesting point is to know whether the present results \textemdash a linear
integration of auditory and visual saliencies \textemdash generalize to situations
with a combination of complex visual and complex auditory scenes. In the
proposed computational scheme the origin of visual and auditory saliency
maps is not constrained, but measured experimentally. The spatial structure
of the auditory salience map is more complex, but presumably does not much
the spatial acuity of the visual system. As a consequence, in the case
several auditory stimuli would contribute no fundamental property in the
integration process needs to be changed and we expect the same integration
scheme to hold. 

In our study, majority of the natural images we have presented to the
subjects were devoid of human artifacts. It could be argued that our
auditory stimuli were semantically more congruent with natural scenes where
there was no human artifact visible and therefore the crossmodal
integration would be stronger. Although some arbitrary decisions has to be
taken, we separated our visual stimuli into two classes depending on
whether human artefacts were present or not, and conducted the regression
analysis with these two sets separately. We have not found stronger
integration in the case of natural images without human artefacts compared
to the case where human artefacts were visible. 

How do these results fit with current neurophysiological knowledge? One of
the most studied structures in the context of cross-modal integration is
the superior colliculus, a deep brain structure (Meredith and Stein, 1986;
Stein et al. 2004). It has long been known that superior colliculus
contains neurons that receive inputs from different modalities. Neurons
fire more strongly with simultaneous congruent spatial stimulation in
different modalities, compared to unimodal firing rates. A recent report
(Stanford et al., 2005) which attempted to quantify this integration
process occurring in the superior colliculus, has pointed out that a great
deal of the integration can be described by the linear summation of the
unimodal channels, thereby providing supporting evidence that it is
possible for linear integration to be implemented in the brain. At the
cortical level, we are far from obtaining a final clear-cut consensus on
how saliency is computed and integrated. In order for a cortical area to
fulfill the requirements of a saliency map, the activity of neurons must
predict the next location of attentional allocation. A number of such
cortical areas have been proposed. Primary visual cortex, the largest of
all topographically organized visual areas, may contain a saliency map (Li,
2002). Simulations inspired by the local connectivity of V1 generate
results compatible with human psychophysical data, thus linking the
activity of neurons in early visual areas to the computation of salience.
By recording single unit activity in monkey cortex during the exploration
of natural visual stimuli,Mazer and Gallant (2003) found that the activity
of neurons in V4 predicted whether a saccade would be made to their
receptive fields. Based on these findings, they argue that V4, a higher
level area located in the ventral visual stream, contains a topographic map
of visual saliency. It is likely that these considerations may be
generalized to other areas located in the ventral stream, but presumably
also to cortical areas responsible for auditory processing. In addition,
areas in the dorsal visual pathway and the frontal lobe \textemdash lateral
intraparietal (LIP) \nomenclature{LIP}{Lateral Intraparietal Area}area and
frontal eye field (FEF) \nomenclature{FEF}{Frontal Eye Field}respectively
\textemdash have been associated with saliency. The activity of FEF neurons
can be effectively modulated by the intrinsic saliency of the stimuli and
further modulated by the current requirements of the task (Thompson and
Bishot, 2004; Thompson et al., 2005). In the dorsal pathway, Bisley and
Goldberg (2006) propose that LIP displays the crucial properties of a
saliency map. Since saliency related activity in the brain seems to be
widely distributed over many areas these areas could in theory be in the
position to independently compete for the control of overt attention.
However, our results support the existence of a joint functional saliency
map, in which the information from different modalities converges before
the non-linearities involved in the process of fixation point selection are
applied. 

It should be noted, however, that our results can not unravel the neuronal
mechanisms underlying integration, as the exact cellular computations could
in principle be carried out by operations other than linear summation of
local variables. This depends on how saliency is represented \textemdash for example,
if saliency were represented logarithmically, the linear summation would
create a multiplicative effect. What we have shown is that at the
behavioral level the information converges before motor decisions are taken
and that this integration is mostly linear. We thus provide boundary
constraints on the computations involved in the control of overt attention. 

\cite{renninger2005a} use information theoretical tools to provide a new
framework for the investigation of human overt attention. According to this
hypothesis, the information gain is causally related to the selection of
fixation points, i.e. we look where we gain the most information.
Considered within this framework, it is tempting to speculate that a linear
integration scheme of unimodal saliencies is compatible with the optimal
information gain, in the sense that the linear integration of information
gains originating from different modalities provides the optimal
combination strategy, as the information gain is the sum of the information
quantities that each modality provides.

The integration of multiple sources of information is also a central issue
in models of attention operating in unimodal conditions. As already
mentioned, modality-specific information is separated into different
feature channels, and the subsequent integration of these different sources
is usually subject to arbitrary decisions on the part of the modeler due to
the lack of biologically relevant data arising from natural conditions.
Whether unimodal feature channels are also linearly integrated is a
testable hypothesis and needs further experimental research. 

One problem we encountered was the centralized fixation density present in
the fixation probability distributions of unimodal auditory conditions.
Although subjects were effectively oriented by the lateralized sound
source, most of their fixations were concentrated at the center of the
monitor. We avoided this problem in our analysis by taking the difference
of the probability distributions obtained in unimodal auditory left and
right conditions, and also by constraining analysis to the subset of
subjects whose behavior was most influenced by auditory stimulation.
However, we believe that this problem may be alleviated by using multiple
sounds simulated to originate from different parts of the image. 

It is common for complex systems, composed of non-linear units, to function
in a linear way. Neurons are the basic functional constituents of nervous
systems, and may express highly non-linear behaviors; for example, the
Hodgkin-Huxley equations describing the relation between membrane potential
and ionic currents are highly non- linear. Furthermore, the excitatory and
inhibitory recurrent connections within and between cortical areas allow
for complex non-linear interactions. However, irrespective of these
underlying non-linear aspects, many neuronal functions are still well
described by linear models. Neurons in early sensory cortices
\citep{schnupp2001a} such as simple cells \citep{Carandini1997b}, for
example, are well approximated when considered as linear filters operating
on input signals. A recent study involving micro-stimulation in motor
cortex showed that signals for movement direction and muscle activation
also combine linearly. \citep{ethier2006a}. We have shown that the
cross-modal integration during overt attention process is best described as
a linear integration of sensory information, possibly originating from
different brain areas. In doing so, we have provided an important
constraint for any model of cross-modal interaction. This raises an
important challenge for any biologically plausible model of human overt
attention operating in environments with multiple source of information. 

