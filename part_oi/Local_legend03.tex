\textbf{Effects of Contextual Stimulation on the Latency of Activation.}
\textbf{(A)} Latency maps derived from presentation of single and coherent
movie patches. Color codes for the time point at which each pixel reached
significant activation (non-responsive pixels in \textit{black}). Left column:
Latency maps during single movie conditions exhibit circular spots of
latency values discernible from a background of non-activated pixels.
Upper and lower rows show two different experiments. Middle column:
Coherent conditions produce two spots in latency maps in which latencies
are decreased compared to single movie presentations. Vertical bar marks 1
mm cortical distance. Colorbar denotes time after stimulus onset in
seconds.  Right column: To quantify the reduction in latencies as a
function of activity levels, five sets of pixels were selected
corresponding to five non-overlapping percentile intervals ranging from
95\textsuperscript{th} to 100\textsuperscript{th} until
75\textsuperscript{th} to 80\textsuperscript{th} based on average activity
maps (not shown). \textbf{(B)} Average latencies computed across pixels
within the highest percentile interval of activity (95\textsuperscript{th}
to 100\textsuperscript{th}) are plotted for single and coherent conditions.
Plus sign represents mean. Each experiment contributes 4 values (two
regions of interests and two different movies). \textit{Red} and
\textit{green} dots correspond to cases illustrated in
\textbf{(A)}.\textbf{(C)} Reduction in latency as a function of activity
levels.
