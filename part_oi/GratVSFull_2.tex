
\subsection{Gratings push nonspecific response component} 


Could these higher activity levels under grating conditions result from
orientation-specific activation, or would it rather be attributable to
unspecific response components? Examples of single condition orientation
maps computed by temporally averaging the dye signal are shown in
\ref{GVSN}B.  To answer the question, we created two sets of
non-overlapping pixels with orthogonal orientation preference by selecting
the highest activated 10th percentile of the pixels from the horizontal and
vertical single condition maps (\ref{GVSN}B, rightmost panel). We then
evaluated the activity levels of these pixels during their preferred and
orthogonal grating conditions and compared these to the activity during
natural conditions.


\begin{figure}[!htb] \centerline{
\label{GVSN}\includegraphics[width=4cm]{part_oi/figures/figure03_gratVSFull_b.png}}
\caption{\protect\textbf{Stimulus Locking of EEG Power to Visual Stimuli.} \textbf{(A)} Each plot shows the
topographic distribution of the grand average of the spectrotemporal
analysis correlation results (averaged over bimodal congruent, incongruent
and unimodal visual conditions and over all subjects) within the 20-35 Hz
band, at selected time lags beginning at 0 ms and ending at 200 ms lag.
Dots represent locations of labelled electrodes, with locations below head
centre drawn outside the cartoon head. Colour codes for the magnitude of
the correlation coefficient, with warm colors indicating positive and cold
colors negative correlation, and values are linearly interpolated between
recording sites for visualisation purposes. \textbf{(B)} Grand average
spectrotemporal analysis correlograms (as in a) for individual frequencies
are shown for electrode OZ. Each row represents the results for a single
frequency, with frequencies given on the y-axis and correlation lags on the
x-axis. Colour codes for magnitude of correlation coefficient
}
\end{figure} 


The average activity levels of pixels during stimulation with gratings of
non-preferred orientation was $.82\times10^{-3} \Delta F/F$ which gives us
an estimate of the orientation-unspecific component. Comparing the activity
during stimulation with the preferred grating orientation, we found an
increase of 25\%, corresponding to a value of $1.03 \times10^{-3} \Delta
F/F$. The resulting modulation depth across all experiments (n=11) is
summarized in \ref{GVSN}C (black vs.  gray bars). The average activity
levels reached for the preferred orientation were $1.06 \times 10^{-3} \pm
.078\times10^{-3} \Delta F/F$  (SEM), compared to $.88 \times10^{-3} \pm
.07 \times10^{-3} \Delta F/F$ for non-preferred.  Thus, the dynamic range
of cortical activity encoding the information about stimulus orientation
corresponds to the top ~20\% of overall activation in accord with Sharon
and Grinvald (2002).

For natural stimuli, an average activity level across both sets of pixels
was $.8\times10^{-3} \Delta F/F$  (\ref{GVSN}C, green bar, left plot). This
corresponds to 84\% of the activity generated by neurons stimulated with a
grating of preferred orientation (compare green and black bars in
\ref{GVSN}C).  Importantly, this value was also smaller than the activity
obtained in response to the non-preferred grating (compare green and gray
bars in \ref{GVSN}C) meaning that even sub-optimally activated regions were
more active when compared to activity yielded by natural stimulation.
Therefore, by using average activity levels to derive single conditions
maps, we conclude that the higher levels of activity observed with gratings
are a consequence of an overall increase in the level of activity.

The natural movie dynamics induce considerable modulations of activity over
the entire course of presentation. We next ensured that these modulations
did not influence the previous analysis. We repeated the same analysis this
time based on single condition maps representing the peak activity reached
during the course of the stimulus presentation. Even though these values
are more prone to outliers, they provide a more robust comparison of
activity levels across these different stimulation conditions. We found
that peak activities for the non-preferred orientation remained higher than
the maximal values recorded during natural conditions (\ref{GVSN}D). This
confirms that the comparatively higher activity levels observed in the
grating conditions were not a result of the temporal modulations
encountered during stimulation with natural movies. Rather, this boost in
activity for grating stimuli results from a general, strong activation of
the visual cortex and cannot be solely attributable to strongly activated
neuronal populations receiving their preferred input. 

\begin{figure}
\centerline{
\label{md_exp}
\includegraphics[width=6cm]{part_oi/figures/ModDepth_fft_plot_3_40_0.png}
\label{md_all}
\includegraphics[width=6cm]{part_oi/figures/ModDepth_fft.png} } 

\caption{this figure is momentarily by hand created, it must have its own
function!!}  \end{figure}




