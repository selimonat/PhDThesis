\subsection{Introduction}



In order to understand processing of natural stimuli it is essential to
unravel neuronal interactions within cortical areas. An important
characteristic of early visual cortex is long-range connections linking
distant neurons to each other \citep{rockland1982a, gilbert1989a}. However
their functions, especially under natural stimulation conditions are
largely unknown. Here we investigated the dynamics of long-range cortical
integration by using localized presentation of natural movies.



\subsection[]{Experimental Procedures\footnote{This part complements the
Experimental Procedures of the last Section. Here only the information
specific to the localized stimulation protocol is provided.}}

\label{local_mm}

In order to investigate long-range cortical interactions, natural movies
were also presented locally through a single or a pair of Gaussian
apertures. These were created by modulating the contrast of the movies as a
function of space according to a two dimensional Gaussian function (FWHM
$\approx3^{\circ}$-$4^{\circ}$, size depend on distance of stimuli to area
centralis in the individual experiments). \nomenclature{FWHM}{Full Width at
Half Maximum} The local movies were placed $\approx3^{\circ}$-$4^{\circ}$
relative to the visual field projection of the area centralis.

Local conditions were indexed by two parameters: position (\textbf{A} or
\textbf{B}) on the screen and movie number (1 or 2) specifying which
full-field movies were to be masked. By displaying either one or
simultaneously two local movies, the conditions \textbf{A}1, \textbf{B}1,
\textbf{A}1\textbf{B}1, and \textbf{A}2, \textbf{B}2,
\textbf{A}2\textbf{B}2 were created (Fig. \ref{loc_cond}). For conditions
with a pair of patches the distance between the centers of the two Gaussian
apertures was equal to 2.5\texttimes FWHM. In conditions
\textbf{A}1\textbf{B}1 and \textbf{A}2\textbf{B}2 ,the stimuli played
through both apertures were belonging to the same original movie (movie 1
or 2) and therefore these conditions are called coherent. Additionally we
created incoherent conditions by presenting two different movies within
both apertures. Local movies were corrected for mean luminance so that the
average of the pixels within the Gaussian apertures was always equal to the
brightness of the background they were embedded. However, we did not
equalize the contrast within each apertures as it is not possible to do so
without introducing strong artifacts, particularly in cases where the local
portion of a movie frame contains a zone with homogeneous brightness values
corresponding to a surface-like object in the movie.

For each experiment, a total of 8 local stimulation conditions were created
(Fig. \ref{loc_cond}B). Using two positions (positions \textbf{A} and
\textbf{B}) and two natural movies (movie 1 and 2), single (conditions
\textbf{A}1, \textbf{B}1, \textbf{A}2 and \textbf{B}2), coherent
(conditions \textbf{A}1\textbf{B}1 and \textbf{A}2\textbf{B}2) and
incoherent (conditions \textbf{A}1\textbf{B}2 and \textbf{A}2\textbf{B}1)
conditions were constructed. During stimulation with these localized
stimuli, we observed evoked activity in 7 out of 11 experiments, which were
included in the following analysis. 

\begin{figure}[!htb]
\centerline{\includegraphics[width=\textwidth]{part_oi/figures/Local_figure01.png}}
\caption[Presentation of Natural Movies.]{\protect\textbf{Presentation of Natural Movies.} For each experiment, two base
natural movies (marked \textit{red}/\textit{blue}) were used to create different stimulation
conditions: In full-field conditions (left), movies maximally cove\textit{red} the
visual hemi-field of the cat. In local conditions, movies were presented
through either one or two Gaussian masks ($3^\circ$ or $4^\circ$ FWHM, see
Experimental Procedures) centered at positions ``\textbf{A}" (top position)
or ``\textbf{B}" (bottom position), illustrated by \textit{white} circles.
According to the position label (\textbf{A} or \textbf{B}) and the index of
the natural movies (movie 1 or 2), the following conditions were displayed:
single (\textbf{A}1, \textbf{B}1, \textbf{A}2, \textbf{B}2), in which only
one local patch was shown; coherent (\textbf{A}1\textbf{B}1,
\textbf{A}2\textbf{B}2), in which two local patches belonged to the same
full-field movie; and incoherent (\textbf{A}1\textbf{B}2 and
\textbf{A}2\textbf{B}2), with local patches derived from different movies
(see rightmost column).
} 
\label{loc_cond}
\end{figure} 
	


The data from 8 hemispheres (7 animals) were used for the results reported
in the following (\textit{local set}). The remaining data had to be
discarded because we did not observe any activity in response to localized
presentation of the movies. Within this dataset local stimulation did not
always activate the cortex strongly. We therefore had to distinguish
between experiments with high activity levels from those where local
stimulation did not yield a strong activity. Therefore the analysis were
occasionally restricted  to the experiments with high activity
(\textit{selected local set}, \ref{local_mm}). The selection was done by
applying a threshold on the peak average activity levels of the evoked data
representing the average activity across presentation time and local
conditions ($\textgreater 0.2 \times 10^{-3} \Delta$ F/F).

\subsection{Results}


\subsubsection{Locally Presented Natural Movies Evoke Localized Dynamic
Activity Patterns}



Upon localized stimulation by natural movie patches, activity emerged from
baseline level with variable delays among conditions (Fig. \ref{loc_st}
shows responses to two different movies: movie one, rows 1-3; movie two,
rows 4-6). The cortical responses were located along the cortical
antero-posterior axis, reflecting the vertical positioning of the movie
patches in the visual field. The localized stimuli gave rise to different
spatial extents and peak values of cortical activation at different frames.
Across the time averaged maps (Fig. \ref{loc_st}, second column from
right), spatially localized spots of activity with approximately centrally
located peak values were clearly visible. These spot-like activation
patterns observed during localized presentation allowed us to disambiguate
the contributions of subcortical input from those of other sources. We were
thus able to choose non-overlapping regions of interests (ROI) that were
specifically driven by individual movie patches presented at the two
different locations. The direct input to these ROIs was  the same under
coherent and single patch conditions, thus any differences in activity can
be attributed to impact of long-range cortical interactions. By selecting
the most active pixels corresponding to the highest percentiles during
single patch conditions (\textbf{A}1, \textbf{B}1, \textbf{A}2 and \textbf{B}2), we chose two pairs of
different ROIs (one pair for each movie) per experiment (Fig. \ref{loc_st},
rightmost column). This yielded 28 ROIs for comparison (local set).
Occasionally we focused on a subset of 19 comparisons corresponding to the
ROIs with the strongest responses to the stimuli (selected local set). In
the following, the latencies and activity levels within these selected ROI
will be compared across conditions to identify contextual interactions.

\begin{sidewaysfigure}[!hp] 
\centerline{
\includegraphics[width=\textwidth]{part_oi/figures/Local_figure02.png}}
\caption[Impact of Coherent Context on Cortical Activation Patterns.]
{\protect\textbf{Impact of Coherent Context on Cortical Activation Patterns.}
Spatio-temporal activity patterns produced by locally presented natural
movies. Leftmost column schematically shows the corresponding stimulus
condition and the movie used (\textit{blue} and \textit{red}). Each single frame represents
the average activity within 100 ms of non-overlapping segments of recorded
data. Please note the differences in activity patterns between top and
bottom triplets of rows where two different movies were used. The
similarity between rows in a given triplet suggests that these differences
are specifically caused by the individual dynamics and content of the
movies. The column to the left of the colorbar shows the average activity
maps computed across the stimulus presentation period. Rightmost column
depicts pixels constituting highest 5\textsuperscript{th} percentile of the
entire activity distributions. Scale bar 1 mm.
} 
\label{loc_st}
\end{sidewaysfigure} 


