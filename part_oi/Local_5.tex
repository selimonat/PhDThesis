\subsubsection{Facilitatory Effects are Sensitive to the Spatio-Temporal
Coherence of the Stimulus}

In how far does the observed facilitation depend on the specific
spatio-temporal properties inherent to the natural movies? As the movies
were recorded from the viewpoint of freely moving cats, each natural movie
had a distinctive flow field profile. Furthermore, motion was widely
distributed across the visual field. During coherent conditions the same
original natural movie was presented through both of the Gaussian
apertures, and consequently the spatio-temporal characteristics of the
natural movie were left intact. In contrast, in the case of incoherent
conditions, movie patches stemmed from different natural movies, leading to
an evident dissonance due to the elimination of naturally occurring
correlations between apertures. The total input
received by cortical neurons was identical across the sum of coherent and
incoherent conditions
(\textbf{A}1\textbf{B}1+\textbf{A}2\textbf{B}2=\textbf{A}1\textbf{B}2+\textbf{A}2\textbf{B}1),
and therefore any difference encountered in activity levels found in this
comparison can be attributed to contextual facilitatory effects arising
from the spatio-temporal coherence of the stimulus.

\newpage

\begin{SCfigure}[50][!htb]

\includegraphics[width=0.35\textwidth]{part_oi/figures/Local_figure05.png}
\caption[Magnitude of Facilitation Depends on Stimulus Coherence.]{\protect\textbf{Magnitude of Facilitation Depends on Stimulus Coherence.}
\textbf{(A)} Cortical activation patterns evoked by coherent (left) and
incoherent (right) movie patches shown for 2 different experiments
(upper/lower row).  Averages were computed across the first 900 ms in
accord with the occurrence of long-range facilitation. Note that besides
differences in peak values, the spatial profiles of activity in response to
coherent conditions were enlarged; in particular cortical regions between
the two activation spots showed elevated activity levels.\textbf{(B)}
Summary across all experiments.
}
\label{loc_d2i}\end{SCfigure} 

As facilitation was prominent during the early phase of stimulus
presentation, we here focused on the first 900 ms. \ref{loc_d2i}A shows average
activity maps evoked by coherent (left) and incoherent (right) stimulation
for two different experiments. As can be seen, the local movie patches
evoked stronger and more spatially extended responses when the inherent
spatio-temporal statistical properties of the natural movies were left
intact. To quantify the differences in activity levels across all
experiments we again compared average activity within those pixels that
were most active (local). The location of these pixels corresponded to the
peak position observed during coherent and incoherent conditions thus
yielded two regions of interests to be compared per experiment. The average
activity was in most cases higher for coherent conditions. This finding was
consistent across the majority of experiments (\ref{loc_d2i}B), with 71\%
of experiments revealing higher activation levels for coherent conditions
(Wilcoxon signed-rank test, $p < 0.05$). We thus conclude that the
facilitatory effects are stronger when the provided context is coherent
with respect to the spatio-temporal characteristics of the remote
stimulation.

\subsection{Discussion}

Center-surround interactions are intrinsically tied to the integrative
operationality of cortical function (for review \cite{allman1985a,
series2003a, albright2002a}). In early visual areas, a fundamental property
of neurons is their ability to sense regions of the visual space which is
beyond their RF borders \citep{fitzpatrick2000a}. The major characteristic
of this ability is that isolated stimulation of surrounding regions is not
sufficient \textit{per se} to drive cortical neurons above firing
thresholds. However strong modulatory effects are exerted on responses to a
stimulus presented centrally, to the point of effectively changing tuning
characteristics \citep{sillito1995a}. Previous experiments demonstrated a
variety of facilitatory and/or inhibitory contextual effects but the final
outcome crucially depends on the precise configuration of the parametrized
stimulus used to stimulate center and surround regions. While the effect by
surround was found to be mainly inhibitory \citep{Hubel1965a, Maffei1976a,
Jones2001a} and spatially asymmetrically organized \citep{Walker1999a}, it
has been shown that the precise nature of the effect depends on the
contrast of contextual stimuli relative to the contrast threshold of the
recorded neuron \citep{toth1996a, sengpiel1997a, kapadia1999a}. Although
these neuronal phenomena are usually interpreted from the viewpoint of
their benefits on naturally occurring visual tasks such as contour forming,
figure-ground separation, object segmentation, or perceptual completion of
occluded objects, it is not clear whether \textemdash and if so, how
\textemdash these center-surround interactions studied with simple stimulus
configurations extrapolate to complex dynamic conditions that mimic natural
input.

By using "keyhole-like" presentations of the original natural movies
through either one or two distant Gaussian masks, we quantified the effect
of surrounding stimulation on local activity levels. We provide evidence
that contextual integrative mechanisms are indeed operative under natural
stimulus conditions. The local movies evoked spatially confined activation
spots with approximately circular shapes. Contextual stimulation led to
higher activity levels within these regions compared to when the movies
were shown in isolation. This facilitatory impact of the distant movie
started within the first hundreds of milliseconds and lasted for nearly one
second. The changes in activation levels were accompanied by reduced
response latencies, most effective at cortical regions that were activated
only slightly when single apertures were presented.

There are three idealized mechanisms, each based on different anatomical
substrates, which could mediate the observed facilitatory long-range
interactions. Overlapping feed-forward thalamo-cortical input could be a
trivial explanation for increased cortical drive during stimulation with
adjacent movie patches. However, there are a number of counter-arguments
against this explanation. First, the pixels analyzed were separated by
distances larger than the spread of direct thalamo-cortical projections
\citep{humphrey1985a, bringuier1999a}. This is additionally supported by
the fact that our latency measurements revealed spatially separated spots
of equal response onsets for each movie patch, thus reflecting synchronous
early thalamic input at two distinct cortical regions. Furthermore, the
pixels included in our calculations were closely located around electrode
penetration sites corresponding to RFs that were clearly
offset in visual space. We therefore exclude a major direct feed-forward
contribution as an account for our results.

Rather, the dense network of unmyelinated intra-laminar horizontal
connections, linking distant neurons across several millimeters of cortical
space, is a potential substrate for the observed long-range effects. It has
been shown that neurons within the early visual cortex of cats receive
diverse of subthreshold input combinations \citep{monier2003a}. As we
observed subthreshold cortical activity dynamics far beyond the retinotopic
representations of the individual movie patches, local activity may
therefore be influenced by horizontal lateral input. In addition, feedback
signals originating from higher visual areas with larger RF sizes than in
the primary visual cortex may play another, additionally important role.
In voltage-sensitive dye experiments, back-propagating waves of activity
have been shown to be initiated in further downstream cortical areas as
early as ~100 ms after stimulus onset \citep{roland2006a, xu2007a}. Thus,
these connections act fast \citep{lamme2000a, hupe2001a} and are likely to
mediate surround modulations spanning considerable distances in visual
space, whereas lateral intra-laminar connectivity may account for
modulations within shorter distances \citep{angelucci2002a}.


An important cornerstone of long-range facilitation is its dependence on
the precise spatial configuration of the surrounding context
\citep{nelson1985a}. It has been shown that facilitatory effects decrease
with decreasing congruency of the contextual stimuli with respect to the
center stimulus \citep{polat1998a, kapadia1995a}. Using static stimuli,
such coherence is generally controlled parametrically by changing the
orientation difference between center and surround patches
\citep{levitt1997a, polat1998a, sillito1995a, chisum2003a}. As our stimuli
were dynamic and complex, to control the coherency of the stimuli across
two Gaussian apertures, we adapted a non-parametric method by exploiting
the unique spatio-temporal characteristics of each original movie. When the
same movie was presented through both apertures they were perceptually
bound without effort, and appeared to belong to a single scene. On the
other hand, the content within both apertures appeared to be immediately
incompatible when two differing movies were used. There are a number of
factors that determine coherence between patches taken from the same movie.
First, stimulus motion was similar between the two distant apertures. This
was due to the ego-motion of the recording cat, which induced large and
equal motion fields across the visual scene captured by the camera. It has
been earlier noted that such temporal phase relationships across distant
regions are perceptually salient and enable object segmentation even in the
absence of any spatial information \citep{Lee1999a}. Second, natural images
tend to possess large spatial correlations because of the dominance of low
spatial frequencies in their spectrum \citep{simoncelli2001a}. Moreover,
auto-correlations of orientations may cover large portions of visual field
reaching up to 8 degrees \citep{Kayser2003a}.

We observed that the facilitatory contextual effects were slightly stronger
when coherently moving movie patches were present in the context. Since the
total input across coherent and incoherent stimulation received by the
recorded cortical area was constant, these results cannot be solely
explained by the local properties of movie patches. Rather, this
facilitation necessarily reflects the outcome of an integrative phenomenon
sensitive to the content of both local movie patches when presented
simultaneously. Therefore, we suggest that the functional architecture of
early visual cortical circuits may have empirically internalized the
typical contextual relationships found in natural visual scenes.

