\textbf{Motion Locking of Average Activity.} \textbf{(A)} The temporal
progression of spatially averaged activity levels is depicted (\textit{red}/\textit{blue}
traces, left y-axis) together with the absolute amplitude of the motion
vector (\textit{black} lines) computed between consecutive frames evaluated within
the local portion of movies directly stimulating the recorded cortical
tissue. In order to discard the transient part of the response only the
period of 200 to 1800 ms after stimulus onset is used. The thickness of
lines represents bootstrap confidence intervals (p = $10^{-5}$) computed by
resampling all the pixels that belong to given frame.  \textbf{(B)} The
correlation between the absolute flow field and the time-course of activity
computed at different lags of time. The two examples in (A) are depicted
with \textit{red}/\textit{blue} traces. \textit{Green} trace represents the
mean taken over all experiments and movies (11 experiments with 2 movies
each). \textit{Shaded} area represents 95\% bootstrap confidence intervals.
The \textit{dotted} line is drawn at 90 ms where the correlation peaks. The
\textit{shaded gray} horizontal bar represents the 95\% bootstrap
confidence interval of the peak location.


