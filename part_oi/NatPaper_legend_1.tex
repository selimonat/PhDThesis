\textbf{Recording Cortical Responses to Natural Stimuli and Gratings.}
\textbf{(A)} Two natural movies (within the \textit{blue} and \textit{red} boxes) and
vertical gratings (within the \textit{gray} box) used as stimulation are depicted
together with evoked optical responses. Visual stimuli are depicted in the
upper row within each box. Leftmost image represents an example movie frame
covering approximately a visual angle of $\approx
40^{\circ}\times30^{\circ}$. The scale bar represents $5^{\circ}$ of angle
of view. The local portion that directly stimulates the recorded cortical
region is delineated with a white rectangle. The temporal evolution of the
movie within the delineated region is shown as a succession of frames. The
second row within each box is space-time representation of the evoked
optical signals recorded during two seconds including the prestimulus
period. Each frame represents the average activity during intervals of
non-overlapping 100 ms. The rightmost image shows the average activity
computed over the entire stimulus presentation. Vascular image of the
recorded cortical area is shown in the top leftmost frame (P=posterior,
A=anterior, M=medial, L=lateral; scale bar represents 1 mm). Colorbar
indicates (bottom box) activity levels as fractional fluorescence change
relative to blank.  \textbf{(B)} Time courses of global activity computed
by taking the average across all pixels of a given frame. Shaded
\textit{gray} area symbolizes the prestimulus period. Line colors are
matched to the boxes shown in (A); \textit{black} = grating,
\textit{blue}/\textit{red} = natural conditions.  The thickness of lines
represents confidence intervals computed by resampling all the pixels that
belong of a given frame (p = $10^{-5}$).  Right panel: Amplitudes of
activity averaged over the presentation time.  Error bars represent the
standard deviation of activity levels depicted in the time-course.

