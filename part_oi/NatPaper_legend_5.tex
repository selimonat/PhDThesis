\textbf{Transient Part of the Response to Stimulus Onset.} \textbf{(A)} Four
plots illustrate the deceleration-acceleration notch occurring during the
transient part of the responses to gratings (\textit{black} traces) and natural
movies (\textit{green} traces) onsets. The time runs from 0 to 200 ms. Deceleration
was not strictly followed by acceleration during natural conditions.
\textbf{(B)} Average across all experiments and 95\% bootstrap confidence
intervals. The time sample where the second derivative (shown below
schematically) of the grating response crosses zero point was detected
(zero point in x-axis and \textit{red} arrows in (A) and (B)) and used to
align in time different time courses prior to averaging. The area under the
second derivative curve located in between the alignment point and the
previous and next zero-crossings were used to quantify the strength of the
deceleration and acceleration, respectively.  \textbf{(C-D)} The
relationship between the orientation selectivity quantified as the
modulation depth and the strength of different DA notch components
(deceleration in the left panel; acceleration in the right panel). 2
experiments out of 11 were excluded because the notch was not detected.
Shaded areas represent the 95\% confidence bands for the regression line.
While the orientation selectivitiy was not found to be correlated with
deceleration strength (r = 0.25, -0.41/0.68, p = 0.05), it was
anti-correlated significantly with the acceleration component (r = -0.6,
-0.96/-0.003, p = 0.05).


