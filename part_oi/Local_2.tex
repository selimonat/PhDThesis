	\subsection{Activated Area Analysis} 
 
\label{oi_local_area}

In this part, I am interested in answering the question of how many pixels
exceeds a given treshold of activity in different conditions.  This is
important because even if we have no effect on the activity levels, there
could be more pixels activated during double conditions than in the sum of
the mono conditions due to a larger area of activation.  



In this part I am simply thresholding the activity levels after
z-transforming the data. I use as threshold a z-value of 2 with the
condition that a two consecutive samples are above the trehsold to be
counted as significantly active. As a consequence I map the continous
activity values to a binary activity levels, 1 meaning above the threshold
and 0 below the threshold. After having transformed the data to a logical
array I sum over the temporal dimension and divide it my the total number
of samples so that I get an estimate of the percentage a pixel was above
the threshold value.


in figure \ref{meta} for each of the experiments, 4 subplots are presented.
each of the subplots compares the double conditions (on the left) to mono
conditions (on the right). The color codes for the percentage of the
activated samples a given pixel had to the total number of samples present
in the interval of iterest. The interval of interest start from the
stimulus onset and goes up to the 1 second after stimulus onset. It
completely ignores the second half of the presentation.



First of all, in only few of the experiments the percentage values reaches
high values. in experiments 4.1, 5.1, 5.2 and 6.1 we have numbers reaching
up to 50 percents. In other experiments these values are smaller. With the
exception of very few subplots the number of activated pixels is always
example aree worth commenting on. for example the first movie of the
experiment 5.2, you can see that in the sums of the mono conditions you do
not see the top blob meaning that when the stimulus was shown individually
it did not activate the cortex in a noticible way. However you see that it
is present in the double condition meaning that the presence of the
secondary blob below it has an effect on the processing of the first blob.
Second example worth noting is the second movie of experiment 4.1 together
with the first movie of the experiment 6.1 where you see the region between
two blobs gets also activated.

\begin{figure}[!hp] \centerline{
\label{loc_countmap}\includegraphics[width=4cm]{part_oi/figures/Local_figure03b.png}}
\caption{} \end{figure}

Another way of looking at those data would be to sum the two mono
conditions and compare it directly to the double conditions. This is done
in figure \ref{meta2}. Here what we see is that in the big majority of the
experiments, the double condition maps have always higher values than the
sum of the mono condition maps. What is also visible in these figures is
that, whenever the data has the perfect shape of two blobs of activity
during double condition spreads to the regions which are in between two
blobs of activity. This is best seen in experiments 4.1 and 6.1 although
counter examples can also be seen in for example the second movie of 6.1


The easy way to visualize this data in a summarizing way is to take the
average of the percentile maps separetely for double and sums of mono
conditions and plot them against each other. This is shown in figure
\ref{meta_sum} in a logaritmic scale. Most of the points lye below the
diagonal meaning that the number of samples activated above the threshold
during double conditions is bigger than the sum of the samples during two
mono conditions. The leftmost point is coming from the experiment 3.1's one
of the movies where the visual stim does not activate any how the cortex,
therefore it is simply noise and can be deleted.

