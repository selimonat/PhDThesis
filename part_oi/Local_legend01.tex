\textbf{Presentation of Natural Movies.} For each experiment, two base
natural movies (marked \textit{red}/\textit{blue}) were used to create different stimulation
conditions: In full-field conditions (left), movies maximally cove\textit{red} the
visual hemi-field of the cat. In local conditions, movies were presented
through either one or two Gaussian masks ($3^\circ$ or $4^\circ$ FWHM, see
Experimental Procedures) centered at positions ``\textbf{A}" (top position)
or ``\textbf{B}" (bottom position), illustrated by \textit{white} circles.
According to the position label (\textbf{A} or \textbf{B}) and the index of
the natural movies (movie 1 or 2), the following conditions were displayed:
single (\textbf{A}1, \textbf{B}1, \textbf{A}2, \textbf{B}2), in which only
one local patch was shown; coherent (\textbf{A}1\textbf{B}1,
\textbf{A}2\textbf{B}2), in which two local patches belonged to the same
full-field movie; and incoherent (\textbf{A}1\textbf{B}2 and
\textbf{A}2\textbf{B}2), with local patches derived from different movies
(see rightmost column).
