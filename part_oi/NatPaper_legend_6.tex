\textbf{Second and Higher Order Characteristics of Response Distributions.}
\textbf{(A)} Temporal evolution of spatial inhomogeneity measured as the
standard deviation of activity levels with a given frame. The values at
each time point were averaged across different experiments after the
standard deviation was computed. Grating stimuli induces slightly stronger
spatial inhomogeneity, yet differences were not significant.  \textbf{(B)}
Histogram of activity levels of all recorded samples excluding the
prestimulus and transient part of the responses for grating (\textit{gray} shade)
and natural (\textit{green}). The data of different experiments are pooled
together. \textit{Black} curve depicts the difference between both
histograms. \textbf{(C)} Same as in (B) but after removing from each frame
its mean value. Both distributions deviate from a Gaussian distribution in
terms of their kurtosis. However distribution during natural condition is
more leptokurtotic with a kurtosis value of 3.91 in comparison to
distribution obtained in response gratings which had a kurtosis of 3.35. 


