\subsection{Experimental Procedures}


\subsubsection{Stimulus Acquisition and Presentation}

Natural movies were recorded by freely moving cats with a head-mounted pair
of cameras (DFM-5303, The Imaging Source Europe, Bremen, Germany) while
exploring a forest (Fig. \ref{basket}D and \ref{setup}). In this study we
used the data recorded from one of the cameras only, hence no binocular
depth cues were given. The cameras and the supporting frame ($\approx$70
gr) were reversibly attached to an electrophysiological implant.

The gain of the camera was set to a constant value and the frames were
recorded with a sampling rate of 25 Hz. The output of the camera was
recorded with a VCR (Roadstar, VDR-6205K, Novazzano, Switzerland) carried
by the experimenter. Recorded analogue movies were later digitalized and
transformed to grayscale with at a size of 640\texttimes480 pixels. The
average and RMS \nomenclature{RMS}{Root Mean Square}value of brightness was
then equalized to the respective average values of the movies in our
database so that each frame in our database had globally the same average
luminance and RMS value. We excluded movies that were solely characterized
by immobile fixation behavior of the cats.

A 24" Sony monitor (Sony Triniton GDM-FW900, Tokyo, Japan) covering a
visual field of $30^{\circ}\times40^{\circ}$ was used to present the
stimuli with refresh rate of 100 Hz (thus each cat-cam movie frame was
repeated 4 times). The stimuli were shown using Matlab (Mathworks, Natick,
MA, USA) and Psychophysics Toolbox extensions
\citep{brainard1997a,pelli1997a}. The duration of the presentation was 2
seconds including 200 ms prestimulus period. Stimuli were presented
binocularly on the monitor located 50 cm distant from the cat's eyes. Eyes
were converged using a prism in front of the eye that was ipsilateral to
the recorded hemisphere. The mean achromatic luminance of the stimuli was
11 cd/m\textsuperscript{2} with 54 cd/m\textsuperscript{2} for brightest,
and 1.6 cd/m\textsuperscript{2} for darkest pixels. The average brightness
of the screen during stimulation and interstimulus intervals (15 s) was
kept constant during the course of the experiment. Stimuli were presented
in pseudo-random order in blocks.

\subsubsection{Stimulus Conditions}

During full-field conditions natural movies and gratings were shown across
the entire monitor screen. In each experiment, two different natural movies
(movie 1 and movie 2) and gratings (horizontal and vertical) were used.
Square wave gratings with spatial frequency of 0.2 cycles/$^{\circ}$ and
drifting velocity of 6 Hz were shown in horizontal and vertical
orientation. Their mean luminance and RMS contrast were adjusted to match
the natural movies.

\subsubsection{Experimental Setup}

All surgical and experimental procedures were approved by the German Animal
Care and Use Committee (AZ 9.93.2.10.32.07.032) in accordance with the
Deutsche Tierschutzgesetz and the NIH guidelines. In brief, animals were
initially anesthetized with Ketamine (15 mg/kg i.m.) and Xylazine (1 mg/kg
i.m.), supplemented with Atropine (0.05 mg/kg i.m.). After tracheotomy,
animals were artificially respirated, continuously anaesthetized with
0.8-1.5\% Isoflurane in a 1:1 mixture of O$_{{2}}$/N$_{{2}}$O, and fed
intravenously. Heart rate, intratracheal pressure, expired CO$_{{2}}$, body
temperature, and EEG were monitored during the entire experiment. The skull
was opened above A18 and the dura was resected.
Paralysis was induced and maintained by
Alloferin\textsuperscript{\textregistered}. Eyes were covered with
zero-power contact lenses as protectives. External lenses were used to
focus the eyes on the screen. To control for eye drift, the position of the
area centralis and RF positions were repeatedly measured. A
stainless steel chamber was mounted and the cortex was stained for 2-3
hours with voltage-sensitive dye (RH-1691), and subsequently washed out
with artificial CSF \nomenclature{CSF}{Cerebrospinal Fluid}. 

\subsubsection{Retinotopical Mappings}

Prior to optical recordings, the topographic mapping between the cortical
surface and the visual field were scrutinized by means of several electrode
penetrations (Fig. \ref{loc_retinotopic}; penetration sites and
corresponding RFs are color coded). Stimuli were positioned so that the
upper movie patch matched the RF position of the simultaneously recorded
multiunit activity (Fig. \ref{loc_retinotopic}, red circle and rectangle).
This ensured that the distance between the Gaussian masks were separated
beyond the size of classical RFs.

\begin{SCfigure}[50][!t]
\includegraphics[width=0.35\textwidth]{part_oi/figures/Local_figure02b.png}

\caption[Retinotopical
Mapping.]{\protect\textbf{Retinotopical Mapping.} Retinotopy across the imaged cortical
region was evaluated by hand mapping of RF locations (colored rectangles)
at various penetration sites (see corresponding colored dots in vascular
image). Parallel to image acquisition, multiunit activity was recorded at a
position (\textit{red} dot) in which the neurons' RF approximately fitted
the center of the upper stimulus (\textit{red} rectangle; \textit{black} dots mark
center of each movie patch).
}

\label{loc_retinotopic} 

\end{SCfigure} 


\subsubsection{Data Acquisition}

Optical imaging was accomplished using an Imager 3001 (Optical Imaging Inc,
Mountainside, NY) and a tandem lens macroscope (Ratzlaff \& Grinvald, 1991),
85 mm/1.2 toward camera and 50 mm/1.2 toward subject, attached to a CCD
camera (DalStar, Dalsa, Colorado Springs). The camera was focused
$\approx$400 nm below cortical surface. For detection of changes in
fluorescence the cortex was illuminated with light of 630\textpm10 nm
wavelength and emitted light was high-pass filtered with cutoff at 665 nm
using a dichroic filter system. Cortical images were acquired at a frame
rate of 220 Hz covering regions of approximately 10\texttimes5 mm of A18. 

Our experimental data that we base our analysis contains 11 hemispheres (10
animals) for the analysis reported in the Section \ref{fullfield}. For
electrophysiological recordings before and in parallel with VSDI, an in
house-build device was used that allowed targeted penetrations at different
locations without opening the sealed recording chamber.

\subsubsection{Preprocessing}

The raw data was processed in two steps. First, in order to remove
differences in illumination across different pixels, divisive normalization
is performed on all the recorded raw samples of a given pixel by its DC
level during prestimulus period. Thus the average value of each pixel
during this period is exactly equal to one for all pixels. Divisive
normalization is preferred over subtractive normalization as it also
equalizes for standard deviation of different pixels. This is due to the
linear relationship between the prestimulus DC amplitude and standard
deviation of the pixels' activities. Second, the removal of heart-beat and
respiration related artifacts were realized by subtracting the average
blank signal recorded in the absence of stimulation. The differences are
later normalized by the blank signal in order to be independent of the
global activity level fluctuations which occurs during the course of an
experiment. As our recordings were synchronized with the heart-beat cycle
of the animal, this blank subtraction step effectively removes these
artifacts. Moreover this method is preferred over the cocktail blank
correction because our conditions were not composed of orthogonal stimuli.
These steps were applied for each trial separately and the outcome was
averaged across trials. The number of trials ranged from 20 to 37 for
different experiments.

\subsubsection{Region of Interest Selection}

In cases where we restricted the analysis to a limited ROI
\nomenclature{ROI}{Region of Interest} of the recorded area, selection
heuristics were based on either time averages or peak activity levels. From
these average or peak activity maps, ROIs were created according to the
percentile ranges of the pixel's activities. For example a given set of
pixels could be assigned to belong to the percentile interval of 95-100\%,
90-95\% and so on. This method of ROI selection has many advantages. As the
selection merely depends on the ranking of the pixels based on their
activity levels and not their absolute value, it facilitates the usage of
the same heuristics for different experiments with varying absolute
activity levels.  A further benefit is that different ROIs originating from
different experiments are all composed of same number of pixels. For
comparison of evoked activity levels during localized presentation of
natural movies, we focused on the highest 5th percentile of the pixels.
However the same results were obtained also with a selection based on the
10\textsuperscript{th} percentile.

\subsubsection{Latency Detection}

Evaluation of the response times was realized on z-score transformed data.
Z-score directly translates a given activity level to its distance from
baseline in units of the baseline process' standard deviation. It is
therefore invariant to changes on absolute level of activities which occurs
during different experiments. Baseline activity was characterized by
temporarily averaging activity levels during prestimulus period. Standard
deviation of the baseline activity for a given pixel was computed by
pooling all its prestimulus samples over all conditions; this increased the
sample size and therefore reduced the uncertainty of the approximation.
Because of the slight spatial dependencies of average and standard
deviation statistics, we computed each pixels' average and standard
deviations for the baseline period separately. After z-score
transformation, latency was determined by detecting for each pixel the time
point at which activity exceeded 2 times the prestimulus time standard
deviation in two consecutive time frames. 


\subsubsection{Singular Value Decomposition} 

We applied SVD \nomenclature{SVD}{Singular Value Decomposition} to the
evoked activity by using \textit{svd} command of Matlab software
(Mathworks, Natick, MA, USA) after transforming each time frame into a
vector. Singular values increased linearly in the logarithmic scale and
significant components were detected by identifying the first component
which had a significantly different increase in its weight with respect to
the previous components.


\subsubsection{Detection of Confidence Intervals}

Due to somewhat large variability across experiments, in some cases it was
preferable to compare first natural and grating conditions in a pair-wise
fashion within each experiment and then evaluate the significance of the
deviations across all experiments. In order to make the pair-wise
comparison possible we first averaged across both natural movie and grating
conditions. In these cases the statistical tests were carried on 11
observations. In all other cases, all observations corresponding to a given
condition across all experiments were used for statistical tests. This
amounted to 22 observations per condition.

In order to estimate the confidence intervals we used the statistical
bootstrap method. We derived the distribution of the statistic of interest
by resampling the observations with replacement $10^5$ times. The
confidence intervals are given in the text within square brackets following
the mean values.
