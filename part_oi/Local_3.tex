	\subsubsection{Long-range Cortical Connections Reduce Response Time}

\label{oi_local_latency} 

We investigated the extent to which contextual stimulation reduced the time
required for neuronal populations to reach a significant activation level.
We defined this as the time it took for a recorded pixel to exceed two
standard deviations of baseline in two consecutive frames, and visualized
these values topographically using latency maps (see \ref{loc_latmap}A,
left and middle columns for examples).  In the case of single conditions
(left column), latency maps exhibited a localized circular spot embedded in
a background that did not reach significant activation levels (black
pixels). We also found a common tendency for the fastest reacting pixels to
be centrally located, with an increase in latencies towards the periphery.
To quantify the effect of contextual modulation on response times, changes
in latency due to the additional movie patch were evaluated within a given
ROI. To avoid problems with determining latencies of weak responses, the
analysis was carried on the selected local set.  
	
	\begin{SCfigure}[50][!h]
	\includegraphics[width=0.35\textwidth]{part_oi/figures/Local_figure03.png}
	\caption[Effects of Contextual Stimulation on the Latency of Activation.]{\protect\textbf{Effects of Contextual Stimulation on the Latency of Activation.}
\textbf{(A)} Latency maps derived from presentation of single and coherent
movie patches. Color codes for the time point at which each pixel reached
significant activation (non-responsive pixels in \textit{black}). Left column:
Latency maps during single movie conditions exhibit circular spots of
latency values discernible from a background of non-activated pixels.
Upper and lower rows show two different experiments. Middle column:
Coherent conditions produce two spots in latency maps in which latencies
are decreased compared to single movie presentations. Vertical bar marks 1
mm cortical distance. Colorbar denotes time after stimulus onset in
seconds.  Right column: To quantify the reduction in latencies as a
function of activity levels, five sets of pixels were selected
corresponding to five non-overlapping percentile intervals ranging from
95\textsuperscript{th} to 100\textsuperscript{th} until
75\textsuperscript{th} to 80\textsuperscript{th} based on average activity
maps (not shown). \textbf{(B)} Average latencies computed across pixels
within the highest percentile interval of activity (95\textsuperscript{th}
to 100\textsuperscript{th}) are plotted for single and coherent conditions.
Plus sign represents mean. Each experiment contributes 4 values (two
regions of interests and two different movies). \textit{Red} and
\textit{green} dots correspond to cases illustrated in
\textbf{(A)}.\textbf{(C)} Reduction in latency as a function of activity
levels.
}	
	\label{loc_latmap}
	\end{SCfigure} 


Within the highest percentile of activation (\ref{loc_latmap}A, red pixels
in the right panel correspond to the 5th percentile) latencies exhibited a
large variability, ranging between 100 and 700 ms (\ref{loc_latmap}B). The
mean latency decreased in the presence of an additional distant movie
patch, with 76\% of cases displaying shorter latencies. The average
latencies were about 340 ms and 268 ms for single and coherent conditions,
respectively (t-test, $p \textless 0.05$). Furthermore, we found that the
effect on latencies increased approximately linearly with decreasing
activity percentiles (\ref{loc_latmap}C). For the least active pixels,
corresponding to the interval of 75-80 \%, the average change in latency
was about 140 ms. This was twice the size of the effect found for pixels
belonging to the highest percentile interval. Hence we conclude that the
presence of a contextual natural stimulus reduced the response times of
distant neurons to their direct subcortical input. Furthermore the least
activated pixels benefited most from this effect.

