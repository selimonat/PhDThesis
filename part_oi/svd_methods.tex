Supplementary Methods

Animal preparation. Animals were initially anesthetized with ketamine (15 mg kg-1 i.m.) 
and xylazine (1 mg kg-1 i.m.), supplemented with atropine (0.05 mg kg-1 i.m.). After 
tracheotomy, animals were artificially respirated, continuously anaesthetized with 0.8 - 1.5 % 
isoflurane in a 1:1 mixture of O2/N2O, and fed intravenously. Heart rate, intratracheal 
pressure, expired CO2, body temperature, and EEG were monitored during the entire 
experiment. The skull was opened above area 18 and the dura was resected. Paralysis was 
induced and maintained by Alloferin�. Eyes were covered with zero-power contact lenses as 
protectives. External lenses were used to focus the eyes on the screen. To control for eye drift, 
the position of the area centralis and receptive field positions were repeatedly measured. A 
stainless steel chamber was mounted and the cortex was stained for 2-3 hours with voltage-
sensitive dye (RH-1691), and unbound dye was subsequently washed out with artificial CSF.

SVD analysis. Generally, SVD and related statistical procedures as Karhunen�Lo�ve 
decomposition and principal component analysis, are helpful mathematical tools to separate 
spatial patterns given by eigenvalues of the autocorrelation matrix from temporal modes in 
complex physical systems28,29. Such procedures were also successfully applied in waveform 
analysis of evoked neuronal potentials30 and spike response patterns31. A first attempt to 
model optical imaging data by Karhunen�Lo�ve decomposition identified different modes 
that resembled the layout of intrinsically recorded maps of ocular dominance, orientation, and 
direction in primary visual cortex32. By exploiting the high spatial and temporal resolution of 
VSDI frequency SVD analysis of recordings in the turtle visual brain led to discovery of 
various forms of widespread traveling activity in different low frequency bands <5, 10, and 20 
Hz33. Such large-scale timing differences of coherently propagating waves across the brain 
were speculated to encode independent visual features by multiple phases of neural 
activity33,34.

SVD Analysis. Evoked signals were computed by averaging 25 to 35 stimulus repetitions. 
We applied SVD to the evoked activity by using svd command of Matlab software 
(Mathworks, Natick, MA, USA) after transforming each frame into a vector. SVD transforms 
the evoked signal,  , into a weighted sum of  separable functions,  , such that 
 ; where   represents the weight of the   component and N, the number 
of frames. Singular values,  , increased linearly in logarithmic scale and significant 
components were detected by identifying the first component which had a significantly 
different increase in its weight with respect to the previous components.
