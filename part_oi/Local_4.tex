	\subsubsection{Lateral Connections Exert Facilitatory Influences under Natural Conditions}

\label{oi_local_fac}


We next evaluated the effect of contextual movies on activity levels. The
time course of activity within the pixels corresponding to the highest 5th
percentile is shown for the example experiment introduced previously
(\ref{loc_m2d}A). The evolution of activity during the local presentation of two
different movies (1st and 2nd rows) at two different cortical positions
(left and right columns) is depicted. Note the similarity in temporal
structure within each row and the pronounced differences across different
rows. This points to the fact that each movie has a specific impact on the
recorded cortical region. The effect of contextual movies can be
appreciated by comparing dashed and solid curves within each plot. The
activity evoked by the same input was similar in the presence or absence of
the second movie patch. However, higher activity levels for coherent
conditions were consistently observed during the first half of stimulus
presentation (black lines). Altogether this suggests that the direct input
dominates the dynamics, while long-range interactions lead to facilitation
upon presentation of contextual information.  


\begin{figure}[!htb]
\includegraphics[width=0.6\textwidth]{part_oi/figures/Local_figure04.png}
\caption[Distant Natural Stimulus Exerts Facilitatory Effects on Local
Activity Levels.]{\protect\textbf{Distant Natural Stimulus Exerts Facilitatory Effects on Local
Activity Levels.} \textbf{(A)} Temporal evolution of activity during single
(\textit{dashed}) and coherent (\textit{solid}) conditions for two natural
movies presented locally (\textit{blue}/\textit{red} mark different movies;
same example data as shown in Fig.\ref{loc_st}). \textit{Black} line shows
differences between coherent and single conditions. In all cases activity
was initially higher in coherent conditions. The insets contain the binary
image representing the pixels belonging to the highest
5\textsuperscript{th} percentile used for calculation of the traces.
\textbf{(B)} Time course of the median difference between single and
coherent conditions (\textit{solid black} line) computed across all
experiments (7 experiments, 28 comparisons). For each time sample, Wilcoxon
signed-rank was used to test for deviations from zero (significant values
marked by circles, see legend). Facilitation remained prominent over 900
ms. \textbf{(C)} Mean activity levels during the first 900 ms for single
and coherent conditions shown for the entire dataset. In agreement with
\textbf{(B)}, most of the data points lie below the diagonal indicating
higher amounts of activity in response to coherent conditions than for
single movie patches. \textit{Filled} squares illustrate experiments in
which the activities of the highest pixels during the whole presentation
period and across coherent and single conditions were below $1.6 \times
10^-4$. \textbf{(D)} Same plot as \textbf{B} summarizing experiments that
revealed strong facilitation (\textit{selected local set} see Section
\ref{local_mm}, \textit{filled} squares in \textbf{C}).
} 
\label{loc_m2d}
\end{figure} 


To test whether this reasoning holds for the entire dataset, we took the
median of differences across all 28 comparisons (solid black curve in
\ref{loc_m2d}B). We found that the facilitatory effect started to evolve briefly
after stimulus onset, reached significance after 500 ms (Wilcoxon
signed-rank test, $p < 0.05$), and remained effective for the next 400 ms.
Across all experiments average activities over this first 900 ms of
stimulus presentation time spanned a wide range, starting from no activity
to values reaching $.4\times10^{-3} \Delta \textrm{F/F}$ (\ref{loc_m2d}C).
In 68\% of comparisons, coherent stimulation yielded a higher level of
activity compared to single conditions (\ref{loc_m2d}C). To test how this
facilitation depended on the efficacy of the movies in driving cortical
activity, we restricted the same analysis to the selected local set (19
comparisons, depicted by filled squares in \ref{loc_m2d}C). In the
remaining 9 cases, activity was very low, and did not exceed
$.2\times10^{-3} \Delta \textrm{F/F}$ within the pixels corresponding to
the highest 5th percentile across all local conditions during the whole
period of presentation. Within the selected local set, the activity during
coherent stimulation was greater in 88\% of the cases and the facilitatory
effect reached significance as early as 200 ms ($p < 0.05$, Wilcoxon
signed-rank test), with a peak at around 600 ms before vanishing at
approximately 900 ms (\ref{loc_m2d}D, solid black curve). Our results so
far have provided evidence that upon context, long-range interactions
within early visual cortex are operational under natural stimulus
conditions, and allow for enhanced activity levels and shorter onset
latencies in the presence of contextual input. 

