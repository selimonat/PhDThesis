\subsection{Analysis of Spiking Data} 



\label{oi_local_spike}

\subsection{Effect Of Electrode Location}



During the experiments the placements of the electrode is not randomly choosen. Indeed it is supposed to be in location correponding where the condition A movies are shown. As a next step I compared the number of spikes at locations A to the recorded spikes at location B during only mono conditions. the results are shown in figure \ref{position}. As it can be seen the number of spikes generated during these conditions are very low but yet the A spikes are higher than the B spikes. 

\begin{figure}[!htb]
\centerline{
\subfigure{\label{position}\includegraphics[width=13cm]{part_oi/figures/SpikeAnalysis_A1_A2_B1_B2.png}}}
\caption{ELECTRODE LOCATION} 
\end{figure}


\subsection{Mono VS Blank}

As the total number of spikes are relatively low during mono conditions, it is worth to analyse how different they are if they are, from the number of spikes recorded during blank conditions. The results are shown in \ref{m2b}. Over all experiments there is significant difference of the average number of spikes recorded during mono conditions compared to double conditions.

\begin{figure}[!htb]
\centerline{
\subfigure{\label{m2b}\includegraphics[width=13cm]{part_oi/figures/SpikeAnalysis_Bl2_Bl1_A1_B1_A2_B2.png}}}
\caption{MONO VS BLANK: the accumulated experiments at the lower left corner are 12, 13 and 3. The only experiment which is below the diagonal is number 8.} 
\end{figure}



\subsection{Double VS Mono Conditions}

Do the extra-receptive field stimulation by the other patch of movie influences the number of spikes? that is do the number of spikes during mono and double conditions differ? The results are shown in figure \ref{m2d_spike}.

\begin{figure}[!htb]
\centerline{
\subfigure{\label{m2d_spike}\includegraphics[width=13cm]{part_oi/figures/SpikeAnalysis_A1_B1_A2_B2_A1B1_A2B2.png}}}
\caption{DOUBLE VS MONO} \end{figure}


\subsection{Coherent VS Incoherent}

The last important comparison is coherent vs incoherent. The results are shown in figure \ref{c2i_spike}. There is no significant difference.

\begin{figure}[!htb]
\centerline{
\subfigure{\label{c2i_spike}\includegraphics[width=13cm]{part_oi/figures/SpikeAnalysis_A1B2_A2B1_A1B1_A2B2.png}}}
\caption{COHERENT VS INCOHERENT} \end{figure}


