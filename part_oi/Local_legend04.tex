\textbf{Distant Natural Stimulus Exerts Facilitatory Effects on Local
Activity Levels.} \textbf{(A)} Temporal evolution of activity during single
(\textit{dashed}) and coherent (\textit{solid}) conditions for two natural
movies presented locally (\textit{blue}/\textit{red} mark different movies;
same example data as shown in Fig.\ref{loc_st}). \textit{Black} line shows
differences between coherent and single conditions. In all cases activity
was initially higher in coherent conditions. The insets contain the binary
image representing the pixels belonging to the highest
5\textsuperscript{th} percentile used for calculation of the traces.
\textbf{(B)} Time course of the median difference between single and
coherent conditions (\textit{solid black} line) computed across all
experiments (7 experiments, 28 comparisons). For each time sample, Wilcoxon
signed-rank was used to test for deviations from zero (significant values
marked by circles, see legend). Facilitation remained prominent over 900
ms. \textbf{(C)} Mean activity levels during the first 900 ms for single
and coherent conditions shown for the entire dataset. In agreement with
\textbf{(B)}, most of the data points lie below the diagonal indicating
higher amounts of activity in response to coherent conditions than for
single movie patches. \textit{Filled} squares illustrate experiments in
which the activities of the highest pixels during the whole presentation
period and across coherent and single conditions were below $1.6 \times
10^-4$. \textbf{(D)} Same plot as \textbf{B} summarizing experiments that
revealed strong facilitation (\textit{selected local set} see Section
\ref{local_mm}, \textit{filled} squares in \textbf{C}).
