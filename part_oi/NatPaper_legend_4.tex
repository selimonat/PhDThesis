\textbf{Differences in the First-Order Characteristics of Response Levels.}
\textbf{(A)} Left: Comparison of average activity evoked by natural movies
and drifting gratings as a scatter plot for the complete data set (n = 11).
Two natural movie (movie 1 and movie 2) and two grating conditions
(horizontal and vertical) were averaged for each experiment beforehand.
Plus sign represents average and SEM. Right: The percentage of samples that
lies above a given common threshold for each experiment. The threshold was
set to be 1 standard deviation above the mean activity computed during
grating conditions (\textit{black} dots). Same absolute threshold was used
for natural conditions (\textit{green} dots).  \textbf{(B)} Single
condition maps obtained for V and H gratings (averaged over the whole
period of stimulus presentation). Pixels with orientation preference
matching to the stimulus are delineated with \textit{black} lines; their
activity corresponds to the specific activity (SC). They occupy the highest
5\textsuperscript{th} percentile of the activity distribution. Non-specific
activity (nSC) is represented by \textit{blue} lines which outlines the
pixels with orthogonal orientation selectivity. Scale bar represents 1 mm.
The middle bar plot depicts the SC (\textit{black} bar), nSC (\textit{blue}
bar) activity averaged across all pixels and experiments. The difference
between SC and nSC is termed Modulation Depth (MD). \textit{Green} bar
depicts the average activity within both sets of pixels during presentation
of the natural movies. Rightmost plot shows the same results computed with
peak instead of mean activity. \textbf{(C)} Temporal evolution of activity
during presentation of grating and natural stimuli (\textit{black} and
\textit{green} filled lines). 22 (Two grating and natural movie conditions
per experiment) different individual time-courses were used. As various
natural movies had different dynamics, average responses are necessarily
less structured than individual time-courses. \textit{Shaded} areas
represent 99\% bootstrap confidence intervals. \textit{Dotted} lines
represent the time course of MD (red traces) and nSC (\textit{blue} traces)
making up the response to grating stimuli. \textbf{(D)} Distribution of
luminance contrast values within constricted regions (e.g. \textit{white}
frame in Fig. \ref{np1}A) of movie frames which directly stimulates
recorded cortical area. The values in the x-axis ranging between 65\% and
128\% represent the luminance contrast in percentage of grating stimulus
contrast. The y-axis on the left represents the number of frame with a
given contrast value. Example frames with increasing luminance contrast
from left to right are shown on the top part. The triangle denotes the mean
of the distribution which is slightly higher than 100\% (Wilcoxon-Signed
Rank, p\textless0.05). The distribution was divided into 9 intervals of
equal number of frames, and for each frame the deviation from the activity
evoked by gratings were computed (\textit{green} dots). The median value of
the deviation is depicted by the height of \textit{green} dots. Best
fitting line had an equation of (r\textsuperscript{2} = 0.61, 0.94/0.11, p
< 0.05) and crossed the x-axis at 132\%. 


