\textbf{Separability of Spatio-Temporal Activity Dynamics.} \textbf{(A)}
Reconstructed evoked activity under spatio-temporal separability hypothesis
for the same data shown in Fig. \ref{np1}. The data represents the first
SVD component and thus the outer product of spatial and temporal activity
profiles computed by averaging along temporal and spatial dimensions,
respectively. \textbf{(B)} Mean time course of the spatially averaged
prediction error computed between the reconstructed and recorded evoked
activity is depicted for natural movie (\textit{green}) and drifting
grating (\textit{black}) conditions. The triangles represent overall
average errors for the examples used in (A). Shaded area represents 95\%
bootstrap confidence interval computed by resampling the data of different
experiments and movies. \textbf{(C-D)} Distribution of two different
measures quantifying spatio-temporal separability is shown as boxplots. The
difference between grating and natural conditions are significant when
evaluated in a pair-wise manner (p\textless$10^{-5}$ and p\textless0.01 for
\textbf{(C)} and \textbf{(D)} respectively).


