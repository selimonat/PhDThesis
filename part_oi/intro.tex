\subsection{Introduction}


The early visual cortex comprises an extended, densely interwoven network,
acting on millisecond time scales \citep{callaway1998a}. Across cortical
layers, activity is rapidly distributed by local feedback loops
\citep{tucker2003a,douglas2007a}, tangentially, long horizontal fibers
connect distant neuronal populations to each other \citep{gilbert1989a,
kisvarday1994a, grinvald1994a, bringuier1999a, jancke2004a}. Additionally,
feedback loops involving higher areas add to interactions over large
distances \citep{bullier2001a}, in sum forming a repeating scheme of
connectivity that allows far-reaching, spatio-temporal integration of
information. Furthermore, neuronal activity in early visual cortex covaries
with basic stimulus parameters such as position, orientation, direction of
motion, color, spatial frequency, and binocular disparity. In many mammals,
the large scale organization of early sensory cortices is in the form of
overlaid maps in which neurons with similar tuning properties are spatially
clustered \citep{hubel1974b, blasdel1986a, bonhoeffer1991a, hubener1997a,
coppola1998b}. When stimulation parameters are varied in isolation or in
well-controlled low-dimensional steady-state experimental settings
\citep{jancke2000a, geisler2007a, benucci2007a}, the layout of cortical
maps changes \citep{basole2003a}, but is still predictable using
feedforward models of visual processing \citep{mante2005b}. 
 
As expected from the success of these models in predicting changes in
population activity and related changes in map layouts, individual neuronal
responses stimulated with simple stimuli such as oriented bars and gratings
are well predicted by linear models incorporating non-linearities at the
output stages \citep{chichilnisky2001a, touryan2002a, rust2005b}. However,
the success of the prediction is largely contingent on the precise
properties of the stimuli that are used; performance drops considerably
when non-parametric, complex stimuli, which appear in ecologically relevant
settings, are used to drive cortical neurons \citep{smyth2003a, david2004a,
david2005a}.  The statistical structures of natural images are by and large
strikingly different to simple laboratory stimulation. While the latter
class of stimuli is controlled precisely by the experimenter and thus
tailored for the purpose of the experiment, the former does not allow
parametrical modifications but better represents the ecologically relevant
input to the sensory system. Despite their complexity, natural images and
movies do possess intrinsic regularities and recent research has shown that
cortical \citep{felsen2005c, mante2005a} and subcortical \citep{dan1996a}
neuronal machinery incorporate diverse phenomena adaptive to the
statistical structures of the input signals. While the question of how
individual neurons are adapted to the input statistics has attracted a
great deal of effort, only a very limited number of studies have compared
the large-scale effects of natural and artificial stimulation, and how
real-world input is represented cortex-wide within the early sensory areas
remains largely unknown.
 
Simultaneous recording of the large number of neurons involved in the
processing of dynamic natural stimuli may in fact lead to an understanding
of discrepancies in the results obtained with different classes of stimuli.
Here we used a large-scale recording method to investigate neuronal
activity dynamics. Optical imaging using blue voltage-sensitive dye records
the sum of synaptic currents with high a spatial and temporal resolution
across cortical distances involving large numbers of neurons with an
emphasis on supragranular layers \citep{grinvald1984a,arieli1996a,
shoham1999a, petersen2003b, jancke2004a, sharon2007a, ferezou2007a}. It
therefore reduces the possibility of having a biased sample of neurons and
faithfully provides the state of cortex under different stimulation
conditions.
 
 Our main working hypothesis is that the cortical circuitry does not embody
 a fixed, generic processing structure in which all input signals are
 processed with the same processing characteristics. In particular, the
 balance between inhibitory and excitatory neurons, being subject to
 modification \citep{david2004a}, may allow different processing regimes
 for different categories of stimuli. Thus the processing of different
 stimuli may in principle be realized in different operating regimes,
 defined by the general characteristics of the state of a large population
 of neurons as they process their input. One corollary of these
 considerations is that simple laboratory stimuli tailored for experimental
 purposes, such as bars in the visual domain and sines or pure tones in the
 auditory domain may indeed lead the system into a processing regime that
 is different than the one within which naturalistic stimuli are processed.
 We defined here an operating regime as the first, second and higher order
 properties of activity levels derived from large number of neurons. As
 optical imaging with voltage-sensitive dyes quantifies the net difference
 between excitatory and inhibitory sources of synaptic activity, it is a
 well-suited recording method for evaluating differences that characterizes
 different operating regimes.
 
 In order to shed light onto the cortex-wide large-scale processing of
 simple and natural visual input we presented natural movies and compared a
 variety of response characteristics to the activity levels evoked by
 simple laboratory stimuli. Our results show that the processing of natural
 movies within the superficial layers of early visual cortex occurs under a
 different spatio-temporal context and in a distinct operating regime.


