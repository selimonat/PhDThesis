\subsection{Discussion} 

In the present section we address the question of whether and how sensory
cortex processes simple, artificial stimuli versus complex, natural stimuli.
Such a comparison between the stimuli most widely used in investigations of
sensory cortex and the ecologically relevant stimuli is of major relevance for
system identification. It is crucial to estimate in how far cortical
functioning can be generalized across different stimulus categories. We
investigated several basic measures of population activity: Peak amplitudes,
average activity, stimulus motion locking and temporal characteristics,
spatio-temporal separability and second-order and higher-order statistics. In
order to simultaneously capture changes in cortical activity across several
millimeters we used a state-of-the-art optical recording method that is
sensitive to sub- and supra-threshold activity levels of a \linebreak \newpage
pool of neurons with a millisecond time resolution. We report large
quantitative and qualitative differences in processing regime, indicating
limited generalization across stimulus categories.  

Natural stimulation evoked sparsely distributed population activity
characterized by highly non-separable spatio-temporal dynamics that were
locked to the intrinsic motion signals present in the movies. In contrast,
responses to drifting gratings were characterized by a stereotypic rapid
increase followed by a monotonic adaptational decay of activity. One of the
major findings was the modest net activity level with balanced effects of
inhibition and excitation in response to natural movies. This was in stark
contrast to the vigorous excitation reached shortly after the transient
notch during stimulation with gratings. An increase by almost half of the
luminance contrast of natural movies was found to be necessary in order to
equalize the activity levels between these different conditions. The
cortical state during stimulation with gratings was dominated by a
widespread feature non-specific activity and, importantly, even this
non-specific activity was stronger than the average and peak activity
levels encountered during the processing of natural movies.


In the interpretation of the high level of unspecific activity induced by
grating stimuli and the lack of spatio-temporal separability of dynamics in
response to natural movies we have to consider several methodological
issues. First, due to highly reciprocal interconnectivity patterns within
the superficial layers and the spatial extension of dendritic trees beyond
functionally distinct domains \citep{gilbert1989a,douglas2004a}, the
optically recorded signal, relating partly to subthreshold activity, could
in principle suffer from averaging unspecific inputs. While, in the case of
gratings, this would lead to an overestimation of the unspecific activity,
it would also mask the non-separable dynamics when present during
processing of natural movies. Indeed, it was shown that the majority of
dendritic contacts within a radius of less than 500 microns do not exhibit
an orientation bias \citep{malach1993a, bosking1997a, buzas2006a}. Thus,
each orientation domain receives horizontal input from differently tuned
neighboring neurons. These unspecific response components are evident in
the orientation tuning of depolarizing responses, which is generally wider
than for spike responses \citep{monier2003a}. This, however, applies more
to natural movies that incorporate a rich spectrum of orientations and
spatial frequencies \citep{simoncelli2001a}. In the case of gratings the
moderate bias of functional columns of similar preferred orientation to be
connected \citep{rockland1982a, gilbert1989a} and the properties of the
reduced stimuli limit such an effect. Yet we observe a much higher level of
unspecific activity in response to grating stimuli. This indicates that the
high level of unspecific activity upon stimulation with oriented gratings,
higher than the average activity induced by natural stimuli, is real and a
sign of a qualitatively different processing regime. 

Second, fibers of passage could in principle add to the measured activity
at a given pixel, and therefore may lead to an erroneous overestimation of
the amplitude of the unspecific response. However, it has been shown that
the dye signal strongly correlates with the changes in membrane potential
at the soma \citep{sterkin1998a,petersen2003c}. Therefore it seems unlikely
that activity at unspecific pixels is particularly infiltrated by specific
signals originating from remote cross-orientation columns. Moreover our
data reveal a large difference in the first order statistics of activity
induced by gratings versus natural stimuli, yet the differences in second
order statistics are smaller in comparison. These observations cannot be
explained by contributions of fibers of passage. Instead voltage sensitive
dye imaging constitutes a well-suited method for the detailed analysis of
specific versus unspecific activation. 

Along the same line, the fact that we observed highly non-separable
spatio-temporal activity dynamics supports the view that optical imaging
does no suffer from the aforementioned problems. The non-separability of
activity during processing of natural movies resulted mainly from emerging
spatial patterns that propagated across time and space and to a lesser
extent by global changes in the activity levels such as adaptation.
Notably, despite their simple statistical structure, the processing of
gratings was also, albeit to a lesser extent, governed by non-separable
spatio-temporal dynamics. The main source of non-separability in this case
was the slight increase in feature selectivity observed across time
accompanying the strong adaptational decay of unspecific activity.
Additionally, we also observed propagating waves retinotopically
representing the motion and the direction of gratings. In fact the
existence of non-separable activity patterns has previously been
demonstrated by in vitro measurements of neocortical patches
\citep{xu2007a} and during in vivo recordings of the visual cortex in
response to simple stimuli \citep{prechtl1997a,benucci2007a}. Here we
report that similar dynamics occur during the processing of natural movies.
Therefore we conclude that the non-separable spatio-temporal dynamics is an
important general quality of cortical responses to both simple and complex
visual stimulation.

By means of relatively long recording duration of about 2 seconds, we were
able to show that processing of continuously moving gratings takes place
within a non-stationary cortical state. The monotonic decay in total
activity during grating presentation relates to the well-known phenomenon
of adaptation found in electrophysiological investigations
\citep{maffei1973a, vautin1977a, movshon1979a, ohzawa1982a, kohn2003a}.
Furthermore, also psychophysically the effects of long-term exposure to
constant stimulation are well-studied \citep{gibson1937a, blakemore1969a,
pestilli2007a}. Moreover, adaptation is frequently used in functional
magnetic resonance imaging experiments in order to reveal the selectivity
of brain areas \citep{krekelberg2006a}. In all these experimental paradigms
adaptation is commonly induced by constant stimulation lasting longer than
20 seconds, sometimes even many minutes. In our experiment, total activity
was reduced by as much as 50\% after only one second of stimulus
presentation. This shows that within neuronal circuitries, major changes
occur already very early under conditions of constant stimulation. This
points to the fact that the cortical processing of the most commonly
employed stimulation method takes place within a highly non-stationary
cortical state. Interestingly, the decay in activity was not specific to
neuronal populations receiving the preferred stimulus. Instead, as shown by
the decay in the unspecific cortical activity, the entire recorded cortex,
whether optimally or suboptimally activated, underwent the adaptational
process.

As the major attributes of the activity in response to gratings were found
to match the characteristics of the unspecific activity, it could be argued
that the adaptational decay occurs independent of the specific content in
the presence of ``any" stimulus. For example, it could be a circuitry
phenomenon in response to the sharp stimulus onset. However, for natural
movies the adaptational decay was largely absent and on average only a
small decrease in activity across time was observed. It is tempting to
speculate, that one of the reasons might be, that in the case of natural
stimuli the onset did not lead to a strong acceleration component, but
stayed in a processing regime with balanced inhibition and excitation. 

Alternatively the fluctuations in activity levels induced unceasingly by
the dynamics inherent in natural movies may be held responsible for the
virtual absence of the adaptational decay. The dye recordings revealed
powerful modulations that were locked to the motion profiles of movies, and
which accounted for a relatively large proportion of the variance present
in the global activity time course. This implies that the observed
motion-locking is a population phenomenon simultaneously affecting cortical
regions of several square millimeters. Indeed, electrophysiological
recordings have previously shown that under conditions in which stimulus
velocity is not constant, local field potentials exhibit stimulus-locked
modulations \citep{kayser2004a, schall2009a, mazzoni2008a}, which result in
action potentials that are characterized by high-degrees of reliability
across trials \citep{mainen1995a}. In contrast, responses to stimuli of
constant velocity, such as drifting gratings, are characterized by a
Poissonian relationship between the mean and the variance of spike counts
\citep{ruyter-van-steveninck1997a}. Hence the low reliability of the
spiking might be a result of the non-stationary temporal dynamics
distinguished by the strong adaptational decay, and may therefore be a
consequence of inappropriate stimulation used for identifying neuronal
response characteristics. On the other hand, high reliability in responses
may reflect an outcome of an adaptive phenomenon dealing with the
ever-changing input patterns resulting from the animal's head and eye
movements. The here observed activity levels of a large number of neurons
locked to the motion cues is likely to underpin such a mechanism.

Shortly after onset of the grating stimulus, we observed a transient
suppression of the rapidly rising activity, termed
deceleration/acceleration notch in a previous report \citep{sharon2002a}.
The dye signal reflects net changes in potentials across membranes and
therefore does not allow an isolated inspection of depolarizing and
hyperpolarizing contributions. Thus the deceleration component, could in
principle result either from withdraw of excitation or increase in
inhibitory strength. However, withdraw of excitation is an unlikely
scenario, given that the input was kept constant during both grating and
natural movie conditions. It was previously argued that the notch results
from the establishment of inhibition within the cortical circuitry
following incoming excitatory inputs \citep{sharon2002a}. It is presumably
triggered by a strong decrease in membrane conductance due to shunting
mechanisms \citep{borg-graham1998a, hirsch1998a, sharon2002a}. In fact,
because activity within pixels coding for orthogonal orientation
decelerated stronger than for preferred orientation, it has been
interpreted as a signature of intra-cortical cross-inhibition
\citep{bonds1989a, ben-yishai1995a, somers1995a, mcLaughlin2000a,
shapley2003a}. In response to natural movies we observed a deceleration in
activity with comparable properties, but the acceleration component was
virtually absent. This resulted most probably from the effective inhibition
leading to sharp termination of the increase in net-excitation. This
provides evidence for a qualitatively different processing mode upon
stimulation with gratings and natural movies. In the former case,
excitation overcomes the rising inhibition in the network and activity
reaches rather high levels. In the latter case, processing of the stimuli
is performed in a regime of more balanced excitation and inhibition. 

How does the increased effective inhibition detected in the dye signal
relate to earlier results obtained at the single cell level? Upon
stimulation with natural images, the strength of inhibitory subregions
within RFs of LGN \citep{lesica2007a} and V1
\citep{david2004a} neurons increased significantly compared to artificial
stimuli. In line with these results, intracellular recordings have shown
that for natural stimuli inhibition followed excitation within a much
smaller time window than for gratings. Such rapidly counteracting
inhibition leads to generation of fewer spikes and a more precise timing
compared to gratings}. Along the same lines, increasing the size of a
natural movie such that it extends to the modulatory surround of the
classical RF leads to highly non-linear changes in spiking behavior. The
enlargement in stimulus size creates a net suppressive effect reducing
overall population activity \citep{vinje2000a, vinje2002a}, in line with
our results obtained for movies that covered the entire visual hemifield.
Moreover, in agreement with the modulatory suppression, natural stimuli
produced increased sparseness \citep{vinje2000a, vinje2002a} and non-linear
changes in feature contrast sensitivity \citep{felsen2005c}.  These
observations underline the impact of suppressive mechanisms on cortical
processing capacities and suggest its functional efficacy in a balanced
regime of processing within neuronal circuits that are designed for natural
input.

The hyper-excitatory state of the cortex upon stimulation with gratings may
lead to erroneous overestimation of the bandwidth of visual feature tuning
as the overwhelming synaptic activation in response to gratings that we
have observed may cause wider tuning curves that occur during processing of
oriented features in natural images. Supporting this conclusion we found
the initial excitation wave to be detrimental with respect to cortical
orientation selectivity as captured by modulation depth measure: During the
first few hundreds of milliseconds where the net excitation peaks, the
cortical spatial patterning due to orientation specific activity was weaker
than predicted by the separability hypothesis. This resulted from a steady
increase in the selectivity and in contrast to \citet*{sharon2002a}, no
brief transient increase in selectivity was observed. Instead, the strength
of the acceleration was negatively correlated with orientation selectivity,
indicating that inhibitory intra-cortical mechanisms, which sharpen
orientation selectivity, were less effective with higher boosts of initial
excitation. Therefore, we conclude that higher activity levels in response
to gratings signifies that the operating point of the cortex was initially
situated at a higher, "artificial position". 

The qualitative differences in processing regime might be related to the
differences in the statistical properties between these two classes of
stimuli. Square wave gratings are characterized, as do natural images
\citep{field1987a, ruderman1994a, schaaf1996a, torralba2003a, betsch2004a},
by a 1/f\textsuperscript{2} power spectrum. Furthermore, the total energy
carried by these stimuli was controlled to be the same. However there are
considerable differences between these two stimulus categories. A grating
is characterized by a single pure orientation and by infinitely long
spatial auto-correlations, whereas spatial correlations in natural scenes
are of finite length \citep{betsch2004a}. These regularities might result
in a state where excitation surmounts inhibition, leading to higher levels
of total activation. An alternative, fully compatible viewpoint interprets
the different processing regime as an adaptational strategy to real-world
stimuli. The reduced activity levels during processing of natural stimuli
reduce energy consumption, an exceedingly important factor from an
evolutionary perspective \citep{laughlin1998a, lennie2003a}. 

Our approach contrasted simple and artificial with complex and natural
stimuli.  Although these may be seen as extremes on a continuous scale, we
observed qualitative differences. Gratings proved to be extremely powerful
in driving cortical activity. For this very reason, cortical dynamics
enters a regime, as signified by the deceleration/acceleration, with high
total activity levels including a strong unspecific component. In contrast,
under natural conditions dynamics are more complex, partly locked to
stimulus dynamics and composed of many non-separable components. Whether
the continuous adaptation, including the suboptimally activated regions,
leads to a processing of grating stimuli comparable to the natural
conditions with balanced excitation and inhibition remains to be
investigated. From an evolutionary perspective, these observations might
not be utterly surprising, given the necessity to limit the energy
consumption of the brain to a level sustainable. Indeed, heightened
attention to the processing characteristics in the regime of balanced
excitation and inhibition is dearly needed \citep{felsen2005b}. 
