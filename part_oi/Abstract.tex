In the previous section, we have seen that relatively simple mathematical
expressions governing synaptic connectivity patterns captured many
essential properties of neuronal RFs located in the early sensory areas.
Rather than predicting the responses of complex and simple cells to natural
movies \textit{per se}, we were dealing with an optimization problem where
the responses were forced to satisfy some well-defined constraints. In this
Chapter, we will directly deal with the characterization of neuronal
population responses recorded from the superficial layers of the early
visual cortex of cats. Natural scenes are complex in nature and there are
no simple mathematical formulas with which to describe them; however, they
are far from random and they contain strong regularities
\citep{chandler2007a}.  Given that natural images occupy only a small
fraction of all possible images, it is expected that neurons constrained by
evolutionary, metabolic and computational pressure have developed some
adaptational strategies for the specific input they are exposed to
\citep{graham2009a}.  We will investigate whether or not cortical responses
do exhibit any signs of functional specialization for the processing of
natural real-world input. As stimulation, we will use natural movies from
the previous Chapter in conjunction with simplified laboratory stimuli,
which are by far the most common source of stimulation in physiological
experiments investigating visual processing. We will use a state-of-the-art
neuronal recording method: Voltage-Sensitive Dye Imaging (VSDI), in order
to visualize directly the responses of neuronal populations to stimulation
with our natural movies. Voltage-Sensitive Dye Imaging offers an
unequalized conjoint spatial and temporal resolution \citep{grinvald2004b}.
Whereas it can simultaneously record neurons spread over distances of few
millimeters on the superficial layers of the cortex, it offers an extremely
good temporal resolution allowing the detection of activity changes in
millisecond time-scales.  However, in order to avoid complications
associated with the stereoscopic presentation of natural movies, here we
presented our movies monocularly.  This recording method is used here for
the first time in conjunction with natural movies.


In the Section \ref{fullfield} of this Chapter, we compare the
spatio-temporal dynamic activity characteristics of neurons driven by
simple laboratory stimuli (i.e. drifting square grating) and complex
natural movies under controlled stimulation conditions. Our results
demonstrate that the processing of natural movies is realized by a rich
repertoire of spatio-temporal activity dynamics which are locked to the
motion signals inherent to natural movies (Sections \ref{oi_motionlock} and
\ref{oi_separability}). As expected from the simplified nature of the
artificial grating stimuli, the spatio-temporal dynamics were much simpler
during processing of grating stimuli. Nevertheless we observed a high
degree of spatio-temporal inseparability in the activity levels. Our
results presented in the Section \ref{oi_svdgrat} show that the responses
to gratings were characterized by a linear superposition of an orientation
and retinotopical map simultaneously encoding the orientation and the
position of the displayed grating.

By comparing large-scale activity levels, we show that the processing of
natural movies is characterized by a state that is very different than the
grating stimuli (Section \ref{oi_operationpoint}). This state is mainly
characterized by an excess of net inhibition. We show that these
differences are unlikely to arise from local differences of luminance
contrast levels between these two classes of stimuli, but rather result
from an evolutionary functional adaptation of the cortex in response to the
input patterns it receives under operational conditions.

The last part of this Chapter (Section \ref{local}) is devoted to the
question of contextual modulation under natural conditions. The early
visual cortex contains an extended network of lateral connectivity going
far beyond the spatial extent of the classical RFs
\citep{series2003a,gilbert1989a}. To study the functional role of these
long-range lateral connections under natural conditions, we presented our
natural movies locally through either one or two Gaussian apertures.  Our
results presented in the Section \ref{local} demonstrated that the local
processing of natural movies is influenced by the distant stimulus located
many visual degrees away. This contextual stimulation had a net
facilitatory effect on activity levels (Section \ref{oi_local_fac}) and
shortened the latency of the cortical responses (Section
\ref{oi_local_latency}).
