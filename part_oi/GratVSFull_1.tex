\subsection{The operating mode of cortical processing is different for
natural stimuli} 


We performed voltage-sensitive dye imaging (VSDI) and multi-unit recordings
in cortical area 18 of anesthetized cats. As visual stimuli we used
drifting gratings and natural movies. The former have been extensively used
over the past decades in many electrophysiological and imaging experiments
and allow a well-founded comparison to activity patterns evoked by natural
stimuli. The latter were captured for the present purpose by cats freely
exploring a natural habitat (\ref{conditions}A). In the full-field
condition, both kinds of stimuli were presented, subtending a large part of
the visual field ($40^{\circ}-30^{\circ}$). To investigate the impact of
long-range cortical interactions, natural movies were also locally
presented, through either a single or a pair of Gaussian apertures
($3^{\circ}-4^{\circ}$) \ref{conditions}B, (see also Supplementary Movie
S1).  




We began our analysis with a comparison of the activation patterns evoked
by drifting gratings and full-field natural scene movies.  Natural scenes
produced a stimulus-dependent response within the first 100 ms after
stimulus onset (\ref{recording}A first and second rows and \ref{recording}B
red/blue traces).  Subsequent activity exhibited dynamic modulations such
as the emergence and spatial propagation of activity intermingled with the
appearance of large regions of low activity (see also Supplemental Movie
S2). In contrast, for gratings, orientation-selective activity appeared in
form of complementary ripple-like activity patterns (\ref{recording}A third
and fourth rows and \ref{recording}B black/gray traces), along with a
strong orientation unspecific excitation distributed widely across the
imaged area. This pattern increased until around 300 ms, and then
constantly decayed.  Average activity maps (\ref{recording}A, rightmost
column) show that stimulation by two orthogonal gratings reveals the
columnar structure of complementary orientation domains (Hubel and Wiesel
1974; Grinvald et al., 1986). For natural movie conditions the average
activity integrated over the entire length of stimulus presentation did not
exhibit any readily identifiable blob-like structure even though strong
spatial patterns were observed in individual frames. Overall, the mean
amplitudes generated by natural stimuli were consistently lower as compared
to gratings (\ref{recording}B, right panel). This observation was also
confirmed by multi-unit recordings, which showed lower firing rates for
natural stimuli compared to gratings (\ref{recording}C, right panel).  

\begin{figure}[!htb] \centerline{
\label{recording}\includegraphics[width=10cm]{part_oi/figures/figure02_recording.png} }
 \caption{\protect\textbf{Two Correlation Analysis Approaches.} Spectrotemporal and waveform
correlation methods are illustrated on the left and right, respectively.
Exemplary stimulus trajectory and single-trial EEG waveforms are shown in
the centre, with a gray bar indicating the temporal range used in both
analysis approaches. In the case of spectrotemporal analysis, each trial is
transformed into a time-frequency representation (see text for details) and
then correlated with the speed profile of the corresponding stimulus. The
waveform analysis method involves creating an ERP from all trials
corresponding to a single stimulus, and then correlating this waveform
directly with the stimulus speed. Peaks at positive time lags indicate that
the EEG power or waveform follows the temporal dynamics of the stimulus, in
other words that the EEG has locked to the stimulus. In both approaches,
correlograms are then averaged over stimuli. Grand averages are created by
taking the median over subjects.
}\end{figure} 






As a next step, we compared the synaptic activity levels reported by the
dye signal and multiunit spiking activities across the whole set of
experiments (n=11). In 9 out of 11 experiments, gratings elicited stronger
optical responses than natural stimuli (\ref{GVSN}A). In the grating
condition, the dye signal was increased by up to 58\% and the median
increase (over experiments) was 30\% (Wilcoxon signed-rank test,
$p=6.83\times10^{-3}$).  Analyzing the spike data in the same manner we
observed a median increase of 48\% in spike counts. Next, we compared these
relative differences across the two recording methods. We did not find a
significant difference in the increases seen in dye signal and spike counts
(Wilcoxon signed-rank test, p=0.19). Taken together, these results show
that grating stimuli lead to considerably higher activity levels than
natural movies.


