
\subsection{Results} 

We presented natural movies along with drifting square gratings (horizontal
and vertical) to anesthetized cats (11 hemispheres, 10 cats) and performed
voltage-sensitive dye imaging on cortical area 18 of cat's visual cortex.
Over the past decades during electrophysiological and imaging experiments
this model sensory cortex has been extensively investigated with simple
laboratory stimuli. Our natural movies were captured by cats freely
exploring a natural habitat and contain a rich spatio-temporal structure.
They depict natural scenes from the cat's point of view and incorporate
temporal dynamics mainly due to head and body motion.  In order to improve
the generality of our results we presented a total of 9 different movies
and in cases where the same movies were used across different experiments,
different portions of the movies were used to drive the recorded cortical
area. Prior to and during the experiment, the recordings were complemented
by retinotopical mappings in order to certify which local parts of the
movies were directly received by the cortical tissue that was being imaged.


\subsubsection{Spatio-Temporal Population Dynamics Evoked by Natural and
Grating Stimuli} 

\label{oi_raw}

\begin{figure}
\includegraphics[width=\textwidth]{part_oi/figures/NatPaper_figure01}
 
\caption[Recording Cortical Responses to Natural Stimuli and Gratings.]
{ \protect\textbf{Recording Cortical Responses to Natural Stimuli and Gratings.}
\textbf{(A)} Two natural movies (within the \textit{blue} and \textit{red} boxes) and
vertical gratings (within the \textit{gray} box) used as stimulation are depicted
together with evoked optical responses. Visual stimuli are depicted in the
upper row within each box. Leftmost image represents an example movie frame
covering approximately a visual angle of $\approx
40^{\circ}\times30^{\circ}$. The scale bar represents $5^{\circ}$ of angle
of view. The local portion that directly stimulates the recorded cortical
region is delineated with a white rectangle. The temporal evolution of the
movie within the delineated region is shown as a succession of frames. The
second row within each box is space-time representation of the evoked
optical signals recorded during two seconds including the prestimulus
period. Each frame represents the average activity during intervals of
non-overlapping 100 ms. The rightmost image shows the average activity
computed over the entire stimulus presentation. Vascular image of the
recorded cortical area is shown in the top leftmost frame (P=posterior,
A=anterior, M=medial, L=lateral; scale bar represents 1 mm). Colorbar
indicates (bottom box) activity levels as fractional fluorescence change
relative to blank.  \textbf{(B)} Time courses of global activity computed
by taking the average across all pixels of a given frame. Shaded
\textit{gray} area symbolizes the prestimulus period. Line colors are
matched to the boxes shown in (A); \textit{black} = grating,
\textit{blue}/\textit{red} = natural conditions.  The thickness of lines
represents confidence intervals computed by resampling all the pixels that
belong of a given frame (p = $10^{-5}$).  Right panel: Amplitudes of
activity averaged over the presentation time.  Error bars represent the
standard deviation of activity levels depicted in the time-course.

}
\label{np1}
\end{figure} 

%\begin{figure}[!t]
%\caption[Recording Cortical Responses to Natural Stimuli and Gratings.]
%{ \protect\textbf{Recording Cortical Responses to Natural Stimuli and Gratings.}
\textbf{(A)} Two natural movies (within the \textit{blue} and \textit{red} boxes) and
vertical gratings (within the \textit{gray} box) used as stimulation are depicted
together with evoked optical responses. Visual stimuli are depicted in the
upper row within each box. Leftmost image represents an example movie frame
covering approximately a visual angle of $\approx
40^{\circ}\times30^{\circ}$. The scale bar represents $5^{\circ}$ of angle
of view. The local portion that directly stimulates the recorded cortical
region is delineated with a white rectangle. The temporal evolution of the
movie within the delineated region is shown as a succession of frames. The
second row within each box is space-time representation of the evoked
optical signals recorded during two seconds including the prestimulus
period. Each frame represents the average activity during intervals of
non-overlapping 100 ms. The rightmost image shows the average activity
computed over the entire stimulus presentation. Vascular image of the
recorded cortical area is shown in the top leftmost frame (P=posterior,
A=anterior, M=medial, L=lateral; scale bar represents 1 mm). Colorbar
indicates (bottom box) activity levels as fractional fluorescence change
relative to blank.  \textbf{(B)} Time courses of global activity computed
by taking the average across all pixels of a given frame. Shaded
\textit{gray} area symbolizes the prestimulus period. Line colors are
matched to the boxes shown in (A); \textit{black} = grating,
\textit{blue}/\textit{red} = natural conditions.  The thickness of lines
represents confidence intervals computed by resampling all the pixels that
belong of a given frame (p = $10^{-5}$).  Right panel: Amplitudes of
activity averaged over the presentation time.  Error bars represent the
standard deviation of activity levels depicted in the time-course.

}
%\end{figure}




Both kinds of stimuli were presented subtending a large part of the visual
field ($\approx 40^{\circ} \times 30^{\circ}$) under equal global luminance
and luminance contrast conditions (Fig. \ref{np1}). However due to the
intrinsic properties of natural movies, local average luminance and
luminance contrast values exhibit some degree of variation across the
presentation time leading to discrepancies between global and local
statistics. For example, a relatively uniform surface of a piece of a wood
or a leaf that is represented with a very peaky distribution of local
brightness values would necessarily have a low luminance contrast. As these
areas in the image cannot be constrained to have a higher luminance
contrast without impairing the quality of the movie we did not control for
these local variations, thus local contrast and average luminance in a
given restricted area of the movie was occasionally higher or lower than
the global target values. Within these restricted regions, the distribution
of the local luminance contrast values were slightly negatively (p
\textless 0.05) skewed (skewness = -0.18) with a standard deviation of
9.7\% (9.3\%/10.1\%\footnote{In the following, bootstrap confidence
intervals are presented in brackets according to the following convention
(Lower Bound/ Upper Bound, $p$ = Confidence Level)}, $p = 0.05$) and an
median luminance contrast sensibly but significantly higher than the
luminance contrast of the grating stimuli (101.25\%, Wilcoxon-Signed Rank,
$p \textless 0.05$, see Fig. \ref{np4} for the distribution of contrast
values of all the frames). There was no systematic bias that caused the
part of the movies that were directly received by the recorded area to be
lower than the global statistics. Therefore we consider that the local
fluctuations do not constitute a major obstacle for the comparison of
activities across stimulus categories and the differences in the activity
levels are not straightforwardly attributable to the local differences in
stimulus power. 

The region which was recorded covered a surface of approximately 1 cm and
it was driven by a portion of $3^\circ$ to $4^\circ$  of visual stimulus
(Fig. \ref{np1}A, white rectangle in leftmost column). It contained a large
number of neuronal components spanning multiple orientation columns and
pinwheels; it thus constituted a representative sample of an early visual
cortex. Fig. \ref{np1}A depicts the evoked activity recorded under
different stimulation conditions along with the local portions of the movie
frames directly received by the recorded area; the cortical response is
represented with 20 frames each representing the average activity during
non-overlapping intervals of 100 ms. With the onset of the natural movies
an excitatory response was observed within the first 100 ms. (Fig.
\ref{np1}A, first and second rows, see also Fig. S1). The average latency
computed from the spatially averaged data (Fig. \ref{np1}B) was 57 ms (SD =
12 ms) and it was not significantly different than the latency of 55 ms (SD
= 11ms, paired t(21) = -0.51, $p = 0.61$) observed in response to grating
stimuli. However, computing the latency separately for each pixel, we found
that not all the pixels responded in the same time to the stimulus onset.
The spatial distribution of latencies was much more variable in response to
natural movies. To evaluate the spatial inhomogeneity we computed the
standard deviation of pixel latencies; the median standard deviation
computed over all experiments (n = 11) was 17 ms and 40 ms ($p \textless
0.001$, signtest) for grating and natural movie stimuli respectively. This
suggests that the onset of natural movies do generate a spatially patterned
and inhomogeneous activity across the cortical surface. 

Following the onset of grating stimuli evoked responses appeared in the
form of ripple-like spatial patterns characteristic of the cortical
columnar architecture for orientation selectivity (Fig.  \ref{np1}A, third row and
black trace). This was accompanied by a strong orientation unspecific
excitation distributed widely across the imaged area. Activity increased
sharply until 300 ms, and then constantly decayed (Fig. \ref{np1}B, black curve).
On the other hand responses to natural movies (Fig. \ref{np1}A, first and second
rows and red/blue traces), exhibited dynamic spatial modulations
characterized by the emergence and propagation of spatial activity
patterns. These spatial patterns co-occurred with temporal fluctuations in
the overall activity levels. The appearance of large regions of low (e.g.
frame 6 in movie 1 and frame 10 in movie 2) activity suggests that
excitatory drives were at times either absent or completely
counter-balanced by inhibition. At these intervals the average activity
within the imaged area approached to prestimulus baseline levels suggesting
that the balance between excitation and inhibition were constantly subject
to modification. 

\subsubsection{Locking of Activity to the Stimulus Motion}

\label{oi_motionlock}

Temporal dynamics of our natural movies were dominated by the body and
head motion of the cat. In parallel, we observed that the time-course of
the global activity i.e. average activity of the recorded cortical area,
during stimulation with natural movies displayed considerable fluctuations.
For each movie within the relevant local portion we computed the amplitude
of the flow-field vector representing the motion between consecutive frames
(Fig. \ref{np2}A black traces; see also Fig. S1). Owing to relatively long
recording interval, we quantified the extent to which the observed
fluctuations in global activity were related to the absolute velocity
profiles of the movies by computing the correlation coefficient at
different lags of time (Fig. \ref{np2}B, right panel). For the two examples shown
in Fig. \ref{np2}A, the cross-correlogram peaked at approximately 100 ms and
reached values up to 0.7 (Fig. \ref{np2}B, blue and red traces). The correlation
curve averaged across experiments peaked at 90 ms with a correlation value
of 0.41 (0.6/0.15, $p = 0.05$) and was significantly different than zero at
the peak position (Wilcoxon-Signed Rank, $p \textless 0.001$; Fig.
\ref{np2}B, green trace). In order to evaluate the variability of the peak
positions across different experiments and movies we detected the peak
position of the correlation curves between the interval of 0 to 300 ms. The
average peak was located at 91.3 ms (71/122 ms, $p = 0.05$, Fig.
\ref{np2}B, gray horizontal bar). These results show that motion cues
within dynamic natural stimuli entail spatially widespread activity
fluctuations by simultaneously driving large numbers of neurons and that a
considerable part of the variance present in the global activity
time-courses is explained by the velocity profile of natural movies. 


\begin{figure}[!t] \centerline{
\includegraphics[width=\textwidth]{part_oi/figures/NatPaper_figure02}
} 
\caption[Motion Locking of Average Activity.]{ \protect\textbf{Motion Locking of Average Activity.} \textbf{(A)} The temporal
progression of spatially averaged activity levels is depicted (\textit{red}/\textit{blue}
traces, left y-axis) together with the absolute amplitude of the motion
vector (\textit{black} lines) computed between consecutive frames evaluated within
the local portion of movies directly stimulating the recorded cortical
tissue. In order to discard the transient part of the response only the
period of 200 to 1800 ms after stimulus onset is used. The thickness of
lines represents bootstrap confidence intervals (p = $10^{-5}$) computed by
resampling all the pixels that belong to given frame.  \textbf{(B)} The
correlation between the absolute flow field and the time-course of activity
computed at different lags of time. The two examples in (A) are depicted
with \textit{red}/\textit{blue} traces. \textit{Green} trace represents the
mean taken over all experiments and movies (11 experiments with 2 movies
each). \textit{Shaded} area represents 95\% bootstrap confidence intervals.
The \textit{dotted} line is drawn at 90 ms where the correlation peaks. The
\textit{shaded gray} horizontal bar represents the 95\% bootstrap
confidence interval of the peak location.


} 
\label{np2}
\end{figure} 

\subsubsection{Spatio-Temporal Separability of Activity Dynamics}


\label{oi_separability}


We used SVD analysis in order to evaluate the separability of the
spatio-temporal dynamics accompanying the temporal modulations reported
above. A function $f(x,t)$ describing spatio-temporal dynamics is said to
be separable if it can be decomposed into \newpage the outer product of
its single dimensional constituents, i.e. $<f(x)>_{t}$ and $<f(t)>_{x}$,
which represent signals averaged across the temporal or spatial dimension,
respectively. Thus, under separability hypothesis the effects of spatial
and temporal dynamics are assumed to interact independently. 


SVD transforms the original signal $f(x,t)$ into a weighted sum of
separable functions $\sum_{i}\gamma_{i}g_{i}(x,t)$ where $\gamma_{i}$
represents the weight of the component \textit{i} in decreasing order of
magnitudes. Following SVD transformation the first component, $g_{1}(x,t)$,
represents the outer product of the average spatial and temporal profiles.
If the signal is separable, this results in all singular values but the
first one to be equal to null. Hence the first components totally capture
the dynamics of the recorded data. 

\begin{figure}[!t] \centerline{
\includegraphics[width=\textwidth]{part_oi/figures/NatPaper_figure03}
} \caption[Separability of Spatio-Temporal Activity Dynamics.]
{ \protect\textbf{Separability of Spatio-Temporal Activity Dynamics.} \textbf{(A)}
Reconstructed evoked activity under spatio-temporal separability hypothesis
for the same data shown in Fig. \ref{np1}. The data represents the first
SVD component and thus the outer product of spatial and temporal activity
profiles computed by averaging along temporal and spatial dimensions,
respectively. \textbf{(B)} Mean time course of the spatially averaged
prediction error computed between the reconstructed and recorded evoked
activity is depicted for natural movie (\textit{green}) and drifting
grating (\textit{black}) conditions. The triangles represent overall
average errors for the examples used in (A). Shaded area represents 95\%
bootstrap confidence interval computed by resampling the data of different
experiments and movies. \textbf{(C-D)} Distribution of two different
measures quantifying spatio-temporal separability is shown as boxplots. The
difference between grating and natural conditions are significant when
evaluated in a pair-wise manner (p\textless$10^{-5}$ and p\textless0.01 for
\textbf{(C)} and \textbf{(D)} respectively).


} 
\label{np3}
\end{figure} 



The signal reconstructed under separability hypothesis (i.e. the first SVD
component), \newpage for the data shown previously (in Fig. \ref{np1}) is
depicted in Fig. \ref{np3}A. In order to get a measure of the separability
of spatio-temporal dynamics, we computed the residuals between the
predicted activity and the evoked activity; where larger residuals would be
an indication of less separable activity dynamics. The time-course of the
prediction error averaged across all experiments is shown in Fig.
\ref{np3}B (see also color matched triangles). During stimulation with
natural movies, the residuals (Fig. \ref{np3}B, green traces) reached a
plateau shortly after the stimulus onset and stayed relatively constant
thereafter. The error profile under grating stimulation (black curve)
started with a phasic strong peak and quickly dropped toward values lower
than the observed for natural stimulation leading to overall smaller error
values (Fig. \ref{np3}C). The median amplitude of prediction errors was
lower in the grating conditions and corresponded to less than 8.8 \%
(7.72/10.5, $p = 0.05$) of the average activity (Fig. \ref{np3}C) this was
significantly smaller than the natural conditions ($p \textless 10^{-5}$,
pair-wise bootstrap) which had a mean error of 13 \% (11.1/15.4 \%, $p =
0.05$). 

The reason for the strong initial peak in residuals can be understood by
comparing the initial frames of the reconstructed (Fig. \ref{np3}A, third row)
and evoked data (Fig. \ref{np1}A, third row); one can see that the spatial
patterning due to cortical columnar organization was much weaker in the
recorded evoked activity, rather a widespread unspecific activity at frames
between 300 and 500 ms was clearly evident. The separable prediction
overestimated the fine-grained columnar structure observed shortly after
stimulus onset and thus led to higher prediction errors. This change in the
overall spatial pattern was the main source of non-separability during
grating conditions. 

We also computed the separability index, a previously used measure
\citep{escabi2003a} which quantifies the ratio of the first singular value
to the sum of all singular values; it thus represents the variance
accounted by the first SVD component and is equal to 1 in case of a
separable function. Median separability index was 0.81 and 0.64 for
gratings and naturals respectively ($p \textless 0.01$, pair-wise
bootstrap).  Accordingly we found that the number of significant singular
values necessary to describe the activity during grating conditions were
much smaller. While in average 5 (4.3/5.6, $p = 0.01$) singular values were
found to be significant, this number was equal to 7.3 (6.4/8.5, $p = 0.01$)
in the case of naturals. We thus conclude that activity within superficial
layers is characterized by non-separable spatio-temporal dynamics during
both natural and grating conditions.


\subsubsection{Singular Value Decomposition Unravels Two Overlapping Maps.
} \label{oi_svdgrat}


The result of the SVD analysis applied to another dataset is shown in Fig.
\ref{svd1}A (upper panel) where the responses evoked by two orthogonal
drifting gratings are depicted together with the time-course of the
activity (Fig. \ref{svd1}A bottom panel) which represents the spatially
averaged response. The SVD was applied to the later part of the evoked
responses (Fig. \ref{svd1}, shaded area) excluding the fast transient
onset. For the data presented, we observed 6 different significant singular
values (\ref{svd1}b, top-left), suggesting inseparable complex
spatio-temporal activity dynamics in the responses to the moving gratings. 

\begin{SCfigure}[][t]
\includegraphics[width=0.55\textwidth]{part_oi/figures/svd_figure01}
\caption[Decomposition of Evoked Cortical Voltage-Sensitive Dye Responses
to Moving Gratings.] { \protect\textbf{Decomposition of Evoked Cortical Voltage-Sensitive Dye Responses to
Moving Gratings.} \textbf{(A)}, Time courses of spatio-temporal activity
patterns (top rows) and spatial averages (bottom traces) expressed as
fractional change in fluorescence relative to blank condition
($\Delta$F/F). Scale bar 1 mm. Here and in all other plots: \textit{green}
= responses to vertical grating, drifting rightwards in visual space;
\textit{blue} = horizontal grating, drifting downwards.  \textbf{(B)}, (Top
left corner) Singular values ranked in order of their contributions.
Components of significant contribution to variance are colored
(\textit{gray} area depicts significance level). The contribution of each
single SVD component to individual recording trials (n = 35) were computed,
their correlations across trials are represented as a matrix. Spatial and
temporal profiles of the first 6 significant SVD components (top row and
left column) were clustered according to their correlation (\textit{red}
and \textit{yellow}).
}
\label{svd1} \end{SCfigure} 


However, the existence of numerous significant SVD components calls into
question the functional relevance of the dynamics described by these
components, as the true underlying biological process is unknown.
Therefore, one cannot directly attribute a biological meaning to the
individual SVD components or to any combination of a subset thereof. We
reasoned that if given sets of components describe an underlying biological
process, their relative contributions across trials should exhibit some
degree of common variation. By measuring the correlation between the
contributions of different SVD components across trials, we detected two
clusters (\ref{svd1}b, correlation matrix). The first cluster contained two
SVD components (Fig.  1b, left column), each displaying non-oscillatory
tonic activity profiles with highest singular values (red circles in
\ref{svd1}b, top-left). The second cluster (\ref{svd1}B, top-row) included
the remaining 4 different SVD components with strongly oscillating
activity. Their relative contributions were similar. However their weight
was two orders of magnitude smaller than the first SVD components,
suggesting weak contribution to the overall evoked activity.  

Fig. \ref{svd2} illustrates the spatio-temporal dynamics described by these
two clusters of non-overlapping SVD components.  Reconstruction of activity
using the first two SVD components displayed the typical patchy structure
of orientation maps characterized by repeating local domains of peak
amplitudes (see contours in red panels and compare to \ref{svd1}). Note
that the gratings evoked the expected orthogonal maps of activation.
Overlaid on these maps, the oscillatory SVD components revealed cyclic
waves of activity that propagated either medial-laterally (rightward in
image frames) or in posterior-anterior direction (downwards) across the
cortex depending on the gratings' drifting direction (contours in yellow
panels). These propagating waves were brought about by oscillations with a
period of $\approx$160 ms. The same patterns were observed in two further
experiments. Thus, responses to each of the gratings' motion direction
showed a clear shift over time indicating stimulus locked retinotopic
propagation. Note that the distance that the wave traveled across the
cortex in one cycle was nearly similar for both conditions. This is due to
the fact that the imaged region matches the part of the cortical
retinotopic map for which cat A18 is approximately isotropic, 5-10 mm below
the representation of the area centralis. In the anterior part of the
image, propagation was less pronounced, most likely due to smaller
amplitudes of oscillatory activity resulting in decreased signal-to-noise
in this region.  

\begin{sidewaysfigure} 
\centerline{
\includegraphics[width=\textwidth]{part_oi/figures/svd_figure02} }
\caption[Propagation of Activity across Stationary Orientation Maps.]
{\protect\textbf{Propagation of Activity across Stationary Orientation Maps.}
Spatio-temporal activity dynamics represented by tonic and oscillatory SVD
components (\textit{red}, \textit{yellow} panels, respectively). Icons at left sketch
stimulus conditions. Time after stimulus onset indicated on top. Each frame
represents average activity within 100 ms (\textit{red} panels) and 10 ms
(\textit{yellow} panels). Contour lines are drawn around
90\textsuperscript{th} percentiles of activity for the tonic components and
at zero crossings for the oscillatory components (multiple cycles of
propagation were averaged in order to increase the signal-to-noise ratio).
Colorbars depict activity levels, $\Delta$F/F. Note the difference of two
orders between the two scales.
}
\label{svd2}
\end{sidewaysfigure} 

Next, to establish the dependency of the observed oscillations on cortical
location, the data were fitted by a harmonic function with a constant
period. Fig. \ref{svd3}A depicts the raw activity over time at two selected
pixel regions (black and gray dots in \ref{svd3}b). We extracted the phase
of the harmonic functions for each recorded pixel and plotted these values
topographically (\ref{svd3}b). In order to estimate the speed of the
propagating waves, we then calculated the change in phase as a function of
cortical distance (\ref{svd3}c). A value of $\sim$34 mm/s was derived that
matched the drifting speed of the gratings (34 $^{\circ}$/s) in visual
space. Consequently, the oscillatory SVD components reflected the shifts of
the gratings' stripes by coherently propagating waves of activity with high
spatial and temporal accuracy.


\begin{figure}[!htb] 
\centerline{
\includegraphics[width=\textwidth]{part_oi/figures/svd_figure03}
} \caption[Propagation Speed of Cortical Activity.] {
\protect\textbf{Propagation Speed of Cortical Activity.} \textbf{(A)} Oscillatory
activity around two pixels separated by 1 mm along the medial-lateral axis
(\textit{gray} and \textit{black} dots in panel \textbf{B}). Oscillatory
activity was best described by a harmonic function with a period of
$\approx$160 ms (\textit{black} and \textit{gray} fitted curves).
\textbf{(B)} The phase of each pixel is shown topographically for both
grating conditions. \textbf{(C)}, Change of phase as a function of space
within circumscribed regions (see vertical and horizontal rectangles in
(B). The slope of the fitting line was $\approx$1.2 rad/mm.
} \label{svd3}\end{figure} 



