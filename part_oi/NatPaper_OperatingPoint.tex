\subsubsection{Comparison of Evoked Activity Levels} 

\label{oi_operationpoint}

We next aimed to characterize the operating regime in which these
spatio-temporal dynamics took place. We define an operating regime as the
first, second and higher order characteristics of the distribution of
activity levels. Fig. \ref{np4}A depicts the activity levels averaged
across the whole recorded area and stimulus presentation period, it thus
illustrates differences in the first order statistics of the activity
levels recorded under these two stimulation conditions. The overall average
activity during presentation of different natural movies was equal to $0.57
\times 10^{-3} \Delta$F/F ($0.46\times 10^{-3} / 0.68\times 10^{-3}
\Delta$F/F, $p = 0.05$). A value of $0.7 \times 10^{-3} \Delta$F/F
($0.58\times 10^{-3} / 0.821\times 10^{-3} \Delta$F/F) was observed with
grating stimuli. Due to relatively large variability in the overall
activity levels across experiments, we first compared grating and natural
conditions within each experiment by computing their differences and later
averaged these differences. In 9 out of 11 experiments grating stimuli
elicited stronger responses (Wilcoxon signed-rank test, $p = 0.004$). The
difference between average activity levels reached values as high as 58\%
of the activity evoked by natural movies; the mean increase was 26\%
(9\%/41\%, $p = 0.01$). As the underlying distribution of activity levels
can be of any form, the average value constitutes a rather underdetermined
description of the underlying distributions. Therefore to complement this
analysis, we focused on the higher tail of the distribution of activity
levels and analyzed the percentage of the recorded samples lying above a
given common threshold value (Fig. \ref{np4}A, right panel). A value of
17\% (16\%/18\%, $p = 0.01$) was consistently observed when the threshold
was set to be one standard deviation above the average activity recorded
during grating conditions (Fig. \ref{np4}A, black dots). The median
percentage of pixels lying above the same absolute threshold value was only
5\% in the case of activity distribution evoked by natural movies (Fig.
\ref{np4}A, green dots). This suggests that the net excitation levels
reached commonly during grating conditions were attained much more rarely
during the processing of natural movies. 

\begin{SCfigure}
\includegraphics[width=0.5\textwidth]{part_oi/figures/NatPaper_figure04} 
\caption[Differences in the First-Order Characteristics of Response Levels.]
{ \protect\textbf{Differences in the First-Order Characteristics of Response Levels.}
\textbf{(A)} Left: Comparison of average activity evoked by natural movies
and drifting gratings as a scatter plot for the complete data set (n = 11).
Two natural movie (movie 1 and movie 2) and two grating conditions
(horizontal and vertical) were averaged for each experiment beforehand.
Plus sign represents average and SEM. Right: The percentage of samples that
lies above a given common threshold for each experiment. The threshold was
set to be 1 standard deviation above the mean activity computed during
grating conditions (\textit{black} dots). Same absolute threshold was used
for natural conditions (\textit{green} dots).  \textbf{(B)} Single
condition maps obtained for V and H gratings (averaged over the whole
period of stimulus presentation). Pixels with orientation preference
matching to the stimulus are delineated with \textit{black} lines; their
activity corresponds to the specific activity (SC). They occupy the highest
5\textsuperscript{th} percentile of the activity distribution. Non-specific
activity (nSC) is represented by \textit{blue} lines which outlines the
pixels with orthogonal orientation selectivity. Scale bar represents 1 mm.
The middle bar plot depicts the SC (\textit{black} bar), nSC (\textit{blue}
bar) activity averaged across all pixels and experiments. The difference
between SC and nSC is termed Modulation Depth (MD). \textit{Green} bar
depicts the average activity within both sets of pixels during presentation
of the natural movies. Rightmost plot shows the same results computed with
peak instead of mean activity. \textbf{(C)} Temporal evolution of activity
during presentation of grating and natural stimuli (\textit{black} and
\textit{green} filled lines). 22 (Two grating and natural movie conditions
per experiment) different individual time-courses were used. As various
natural movies had different dynamics, average responses are necessarily
less structured than individual time-courses. \textit{Shaded} areas
represent 99\% bootstrap confidence intervals. \textit{Dotted} lines
represent the time course of MD (red traces) and nSC (\textit{blue} traces)
making up the response to grating stimuli. \textbf{(D)} Distribution of
luminance contrast values within constricted regions (e.g. \textit{white}
frame in Fig. \ref{np1}A) of movie frames which directly stimulates
recorded cortical area. The values in the x-axis ranging between 65\% and
128\% represent the luminance contrast in percentage of grating stimulus
contrast. The y-axis on the left represents the number of frame with a
given contrast value. Example frames with increasing luminance contrast
from left to right are shown on the top part. The triangle denotes the mean
of the distribution which is slightly higher than 100\% (Wilcoxon-Signed
Rank, p\textless0.05). The distribution was divided into 9 intervals of
equal number of frames, and for each frame the deviation from the activity
evoked by gratings were computed (\textit{green} dots). The median value of
the deviation is depicted by the height of \textit{green} dots. Best
fitting line had an equation of (r\textsuperscript{2} = 0.61, 0.94/0.11, p
< 0.05) and crossed the x-axis at 132\%. 


}
\label{np4}\end{SCfigure} 


\subsubsection{Specific vs. non-Specific Activity}

As we were simultaneously recording neurons spanning the whole orientation
space, we were able to dissociate the activity into two components, namely
the specific (SC) \nomenclature{SC}{Specific Activity}and non-specific
(nSC) \nomenclature{nSC}{non-Specific Activity}activity. We divided the
recorded area into two sets by choosing the most active pixels occupying
the highest 5th percentile during presentation of horizontal and vertical
gratings (Fig. \ref{np4}B, pixels delineated with black/blue contours). nSC
activity represents neuronal activity evoked by a grating that is
orthogonal to the preference of the neurons \citep{sharon2002a}. The
incremental change in activity that is due to the change of orientation to
the preferred one is termed the modulation depth (MD). We found SC and nSC
activity to be equal to $1.06\times 10^{-3} \Delta$F/F ($1.21\times
10^{-3}/0.91\times 10^{-3}$, $p = 0.05$) and $0.88\times 10^{-3} \Delta$F/F
($1.02\times 10^{-3}/0.75\times 10^{-3}$, $p = 0.05$) respectively (Fig.
\ref{np4}B, middle panel blue and black bars). This corresponded to a MD of
20\% (18\%/24\%, $p = 0.05$) of the nSC activity. The average activity
during natural conditions across both sets of pixels was found to be equal
to $0.74\times 10^{-3}\Delta$F/F ($0.64\times 10^{-3}/0.88\times 10^{-3}$,
$p = 0.05$) and interestingly this value was smaller than nSC activity
(Fig.  \ref{np4}B, middle panel green bar). In a pair-wise comparison, we
found that on average the nSC activity was 19\% (7\%/30\%) higher than the
activity during natural conditions (Wilcoxon sign-rank test, $p = 0.018$). 

As shown previously natural movies induced continual modifications in the
balance between excitation and inhibition over the entire course of movie
presentation; one  inter- \newpage

esting  question is to know whether the peak values that can be reached
during natural conditions also differ from those reached during grating
conditions. We therefore repeated the same analysis, this time based on the
peak activation levels reached during the course of stimulus presentation
(Fig. \ref{np4}B, rightmost panel). The peak values averaged across
experiments were approximately 2 times higher than the average amplitudes.
Usage of peak activity levels resulted in a slightly reduced MD
\nomenclature{MD}{Modulation Depth} value of 14\% (12\%/16\%, p = 0.05).
This suggests that even during periods of very strong excitatory drive, the
activity of neurons with orthogonal orientation preferences conforms to the
functional cortical domains, albeit with a slightly reduced selectivity
levels. We confirmed that nSC peak activity was higher than maximum values
attained during natural conditions (Wilcoxon sign-rank test, $p = 0.024$)
by an amount of 8\% (2\%/12\%, $p = 0.05$). These results show that the
processing of oriented bar stimuli is concomitant to a widespread vigorous
excitatory drive; even sub-optimally driven neuronal populations were
subject to a strong net excitation resulting in much higher activity levels
encountered during stimulation with naturalistic movies.


\subsubsection{Differences in the Time-Courses of Activities}

In order to analyze the temporal evolution of the average activity, we
computed the characteristic time-courses for both stimulation conditions
(Fig. \ref{np4}C, gray and green traces, shaded area represents 99.99\% CI).
These represent the activity averaged across different experiments and
therefore illustrate systematic effects of the stimulation category rather
than experiment specific influences (such as for example flow field locked
specific temporal dynamics as shown in Fig. \ref{np2}A). No global trends were
observed in the time courses during stimulation with natural movies. On the
other hand, despite their stationary nature as stimulation, the processing
of drifting square-wave gratings was characterized by a strong adaptational
decay in the global activity levels which led to a dramatic decrease of net
excitation levels. To quantify this decay, we fitted a line to the global
activity time courses and in order to exclude the transient part of the
response, we only included samples starting from 300 ms after stimulus
onset. In all experiments the grating stimuli gave rise to stronger decay
values hence to more negative slopes and in no experiment a positive slope
was observed. The slopes were computed after expressing the activity levels
in percentage of peak activity recorded during grating conditions. For the
grating condition, the slope was equal to -48\%/s (-59/-35\%/s, $p =
0.01$), meaning that after only one second of presentation time the
amplitude of the net excitation declined to a level which is half of the
peak activity levels. During stimulation with natural movies the mean
adaptational decay was -24\% (-44/-5\%/s, $p = 0.01$), and the median slope
was found to be only marginally different than zero (sign-test, $p =
0.065$).  This suggests that the processing of natural movies, despite
their complex spatio-temporal dynamics, is characterized by a rather
stationary activity dynamics in the superficial layers of visual cortex.
Given the fact that the natural movies we used were dynamic, these results
are not surprising.  Nevertheless, we would like to emphasize the fact that
grating stimuli, which are the most common form of stimulation used in
experiments, are processed in a regime which is strongly non-stationary
with respect to continual change in the balance of excitation and
inhibition.

We next analyzed the temporal evolution of the nSC activity and MD
separately (Fig. \ref{np4}C red and blue traces). We found that MD kept
increasing during the first second of the presentation and reached a
plateau relatively late and remained relatively constant until the stimulus
offset (Fig. \ref{np4}E). From 200 ms until 1000 ms the value of MD
nearly doubled and reached values close to $0.35\times 10^{-3}$. On the
other hand the time course of the nSC activity (blue trace) was found to
follow a very similar profile as the overall response time course (black
trace) and toward the end of presentation it approached values as small as
modulation depth. This similarity shows that the major characteristics of
the temporal responses to grating stimuli is dominated by nSC activity
whereas the activity responsible for the encoding of the stimulus
properties such as orientation, has a much smaller and but also more stable
temporal evolution.


\subsubsection{Effect of Variability of Local Luminance Contrast}

To what extent could local properties of natural images such as
fluctuations in the luminance contrast (LC) \nomenclature{LC}{Luminance
Contrast} values be responsible for the differences in the average activity
levels observed? As stated above, the LC values within the local region of
the natural movies directly stimulating the recorded cortical area
exhibited some degrees of fluctuations (Fig. \ref{np4}D, top row of
frames). The distribution of local LC values is shown in Fig. \ref{np4}D
where 100\% in the x-axis represents the LC of the grating stimuli which
did not vary across time. LC values ranged between 65\% and 128\%. For each
of the frames, we computed the evoked activity level associated with it and
computed the ratio to the activity evoked by gratings at the same time of
presentation. For example, to find the activity level evoked by a movie
frame occurring at 500th ms, we measured the average activity at 590 ms and
compared it to the activity evoked by grating stimuli recorded at exactly
the same time. The median of the ratios were calculated after pooling all
the data points which belonged to a given LC percentile interval (Fig.
\ref{np4}D, green dots) each containing the same number of frames. Not
surprisingly the biggest difference between naturals and gratings was
observed for the frames that had the lowest LC values corresponding to the
interval of 65\% and 88\%; the median ratio of activity level reached
values as high as 150 \% (Fig. \ref{np4}D, leftmost green dot). For the
interval where the LC of the natural stimuli was approximately equal to the
target contrast (percentile interval of 99\% and 103\%), the median ratio
decreased down to only 120 \% indicating that grating stimuli induced still
stronger responses. The estimated parameters of the best fitting line using
linear regression indicated that a boost of LC corresponding to 132\% of
the baseline LC level would be necessary in order to obtain the same
activation levels between natural movies and gratings.

\begin{SCfigure} 
\includegraphics[width=0.5\textwidth]{part_oi/figures/NatPaper_figure05} 
\caption[Transient Part of the Response to Stimulus Onset.]{ \protect\textbf{Transient Part of the Response to Stimulus Onset.} \textbf{(A)} Four
plots illustrate the deceleration-acceleration notch occurring during the
transient part of the responses to gratings (\textit{black} traces) and natural
movies (\textit{green} traces) onsets. The time runs from 0 to 200 ms. Deceleration
was not strictly followed by acceleration during natural conditions.
\textbf{(B)} Average across all experiments and 95\% bootstrap confidence
intervals. The time sample where the second derivative (shown below
schematically) of the grating response crosses zero point was detected
(zero point in x-axis and \textit{red} arrows in (A) and (B)) and used to
align in time different time courses prior to averaging. The area under the
second derivative curve located in between the alignment point and the
previous and next zero-crossings were used to quantify the strength of the
deceleration and acceleration, respectively.  \textbf{(C-D)} The
relationship between the orientation selectivity quantified as the
modulation depth and the strength of different DA notch components
(deceleration in the left panel; acceleration in the right panel). 2
experiments out of 11 were excluded because the notch was not detected.
Shaded areas represent the 95\% confidence bands for the regression line.
While the orientation selectivitiy was not found to be correlated with
deceleration strength (r = 0.25, -0.41/0.68, p = 0.05), it was
anti-correlated significantly with the acceleration component (r = -0.6,
-0.96/-0.003, p = 0.05).


}
\label{np5}\end{SCfigure} 


\subsubsection{Deceleration/Acceleration Notch}

We next focused on the transient part of the cortical response representing
the arrival of the visual input to superficial layers of the cortical
circuitry. The steep increase of the net excitation immediately following
the stimulus onset was interrupted by a transient slowdown occurring over a
short period of time (Fig. \ref{np5}A, four examples). This has been
referred to as the deceleration-acceleration (DA)
\nomenclature{DA}{Deceleration/Acceleration Notch}notch in VSDI experiments
and evidence suggests that it corresponds to the establishment of
inhibition within the cortical circuits following the arrival of excitatory
input \citep{sharon2002a}. In all but two experiments we observed a notch
occurring approximately 100 ms after stimulus onset. The exact time of DA
notch varied slightly between experiments (Fig. \ref{np5}A, red arrows). In
order to align different experiments in time we detected the zero crossing
of the second derivative of the dye signal (shown schematically in Fig.
\ref{np5}B) recorded during grating conditions (Fig. \ref{np5}A, red
arrows) and proceeded by averaging the time courses (Fig. \ref{np5}B, black
and green curves). The initial deceleration followed by a strong
acceleration was evident in the average transient response in the case of
stimulation with grating stimuli (Fig. \ref{np5}B, black curve). A similar
deceleration of activity occurred also in response to natural stimuli.
However, the following acceleration component was much weaker or even
absent for natural movies (green curve). As a consequence of the pronounced
acceleration during grating stimulation, activation levels peaked at
extremely high values; the activity between 150-450 ms was 48\% higher than
the overall mean, compared to only 23\% in case of natural conditions. We
thus conclude that the onset of grating stimulation caused an initial
overshoot of activity within the first hundreds of milliseconds. On the
other hand, the activity evoked by natural movies never or very rarely
reached such high values. In order to check whether any bias in the
statistics of natural images could be responsible of this outcome, we
computed the luminance contrast within the area of the movies directly
stimulating the recorded cortical area during the first 300 ms. The average
luminance contrast was equal 100.29\% (94.1\%/105.2, $p = 0.001$) of the
target luminance contrast. Therefore any bias in the contrast levels toward
lower values that might be present in the natural movies within the first
300 ms cannot be held responsible for the discrepancies in the activity
levels. These observations support the view that stimulation with gratings
provides an excitatory input which is extremely powerful and affects the
balance between excitation and inhibition in favor of strong excitation. In
contrast, the system processes natural scenes in a more balanced regime
leading to a marked absence of the late acceleration component and high
activity levels. The extremely high activity levels hint into a
qualitatively different state of the underlying cortical circuitries.

\subsubsection{Strength of the Acceleration vs. Orientation Selectivity }

We reasoned that the stronger the acceleration part, thus stronger the
excitation opposing inhibition, weaker might be the selectivity of the
visual cortex. To test this hypothesis we evaluated the selectivity of the
visual cortex by computing the MD and quantified its relationship to the
magnitude of different notch components across different experiments. The
magnitudes of these were computed by integrating the area just neighboring
the zero crossing of the second derivative of the raw dye signal (Fig.
\ref{np5}C-D, red and blue shaded areas). While the deceleration component
was uncorrelated with the cortical selectivity (r = 0.25, -0.41/0.68, $p =
0.05$), the acceleration was strongly anti-correlated (r = -0.6,
-0.96/-0.003, $p = 0.05$). In line with the earlier findings showing higher
selectivity for neurons under natural conditions, our finding supports the
view that the cortex under strong excitatory input looses its selectivity.

The MD represents the difference between activity levels of neurons that
are strongly and weakly driven. Therefore it can in principle be used to
quantify the dynamic range of neuronal populations. However a measure
similar to MD is not easily derivable in the case of activity patterns
evoked by natural movies because the natural stimuli is not clearly
segregated into orthogonal orientation components making it difficult to
track down differences in the spatial inhomogeneities of activity levels.
To circumvent this problem, we computed the standard deviation of the
activity values for each frame separately (20 frames representing each the
average activity during 100 ms) and in order to obtain systematic
differences between these two stimulation conditions we averaged the
results across experiments (Fig. \ref{np6}A). We found that the grating stimuli
induced slightly stronger spatial inhomogeneities, however these
differences were not found to be significantly different. These results
show that the second-order statistics, contrary to first-order properties
of activity levels are not qualitatively different between these two
categories of stimulation. The dynamic range which is made use of during
encoding of visual input is similar under natural and more simplified
stimulation conditions.


\begin{figure} \centerline{
\includegraphics[width=\textwidth]{part_oi/figures/NatPaper_figure06} }
\caption[Second and Higher Order Characteristics of Response Distributions.]{ \protect\textbf{Second and Higher Order Characteristics of Response Distributions.}
\textbf{(A)} Temporal evolution of spatial inhomogeneity measured as the
standard deviation of activity levels with a given frame. The values at
each time point were averaged across different experiments after the
standard deviation was computed. Grating stimuli induces slightly stronger
spatial inhomogeneity, yet differences were not significant.  \textbf{(B)}
Histogram of activity levels of all recorded samples excluding the
prestimulus and transient part of the responses for grating (\textit{gray} shade)
and natural (\textit{green}). The data of different experiments are pooled
together. \textit{Black} curve depicts the difference between both
histograms. \textbf{(C)} Same as in (B) but after removing from each frame
its mean value. Both distributions deviate from a Gaussian distribution in
terms of their kurtosis. However distribution during natural condition is
more leptokurtotic with a kurtosis value of 3.91 in comparison to
distribution obtained in response gratings which had a kurtosis of 3.35. 


}
\label{np6}\end{figure} 

\subsubsection{Population Sparseness}

We constructed histograms of the activity levels by pooling all recorded
samples of all experiments (excluding samples recorded during prestimulus
and the transient part) separately for the grating and natural conditions.
Most remarkable difference between these histograms was the strong
rightward shift seen for the activity distribution originating from the
grating condition corresponding to the first order differences (Fig. \ref{np6}B,
black trace). Moreover differences in the extreme values also were
observed; while lowest values were more abundant during natural conditions,
the opposite was true for high-end values. We quantified the higher order
characteristics of these distributions, namely the kurtosis. The
distribution of activity levels during natural conditions was found to be
slightly more leptokurtotic with a kurtosis of 3.32 (3.31/3.33, $p = 0.05$)
whereas a value of 3.07 (3.07/3.08, $p = 0.05$) was obtained from the
distribution of activity levels under grating conditions. 

These histograms incorporate changes in the activity levels that occur over
the stimulus presentation hence their shapes are to some extent dominated
by average activity fluctuations. For example the decay in activity levels
during stimulation with gratings may have a non negligible contribution to
the shape of the histogram depicted in Fig. \ref{np6}B. In order to
eliminate these confounding effects and to have a better overview on the
spatial distribution characteristics of activity levels, we recomputed the
histograms after removing from each frame its average activity level, thus
centering all the frames around zero level (Fig.  \ref{np6}C). Both of
these distributions deviated significantly from a Gaussian distribution in
terms of their kurtosis values; a kurtosis of 3.35 (3.34/3.37, $p = 0.05$)
was found for the spatial distribution of pixels under stimulation with
gratings. For the natural condition, the distribution was more strongly
leptokurtotic with a value of 3.91 (3.88/3.93, $p = 0.05$). These results
shows that spatial distribution of activity levels evoked by grating and
natural stimuli do have different higher order properties. Moreover high
kurtosis values found under natural conditions is compatible with the view
that neuronal populations have a sparse representation of the visual input.
